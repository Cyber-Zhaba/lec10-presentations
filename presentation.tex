\documentclass[aspectratio=169]{beamer}

% ==============================
% ==  Основные пакеты и тема  ==
% ==============================
\usetheme{metropolis}

\usepackage{fontspec}
\usepackage{amssymb}
\setsansfont{Fira Sans}
\setmonofont{Fira Code}

\usepackage{tikz}
\usetikzlibrary{arrows.meta, positioning, shapes.multipart}

% Ячейка для окружения
\tikzset{
    env-node/.style={
        rectangle split, rectangle split parts=#1,
        draw=green!60!black,
        fill=green!5,
        rounded corners,
        thick,
        align=center
    },
}

% ==============================
% ==        Метаданные        ==
% ==============================
\title{10. Хвостовая рекурсия и замыкание лямбда-процедур в языке Scheme}
\institute{Курс: Основы информатики}
\date{}

% ==============================
% ==      Блоки с кодом       ==
% ==============================

% Настройки для блоков с кодом
\usepackage{minted}
\usepackage[svgnames]{xcolor}
\definecolor{bg}{rgb}{0.95,0.95,0.95}
\newminted{scheme}{
    fontsize=\scriptsize,
    linenos,
    numbersep=8pt,
    gobble=4,
    frame=lines,
    bgcolor=bg,
    framesep=3mm,
    escapeinside=||
}

% Тривиальная реализация факториала
\newsavebox{\codeFactorialSimple}
\begin{lrbox}{\codeFactorialSimple}
    \begin{minipage}{0.4\textwidth}
        \begin{schemecode}
(define (f1 n)
  (if (= n 0)
      1
      (* (f1 (- n 1)) n)))
        \end{schemecode}
    \end{minipage}
\end{lrbox}

% Хвостовая рекурсия для факториала
\newsavebox{\codeFactorialTailRec}
\begin{lrbox}{\codeFactorialTailRec}
    \begin{minipage}{0.4\textwidth}
        \begin{schemecode}
(define (f2 n acc)
  (if (= n 0)
    acc
    (f2 (- n 1) (* n acc))))
        \end{schemecode}
    \end{minipage}
\end{lrbox}

% Пример замыкания лямбда-процедуры
\newsavebox{\codeLambdaExample}
\begin{lrbox}{\codeLambdaExample}
    \begin{minipage}{0.4\textwidth}
        \begin{schemecode}
(define (f x)
  (+ (g (lambda (y)
        (* x y))) 3))

(define (g h)
    (h (h 7)))
        \end{schemecode}
    \end{minipage}
\end{lrbox}

% Пример замыкания для счетчика
\newsavebox{\codeClosureCounter}
\begin{lrbox}{\codeClosureCounter}
    \begin{minipage}{0.4\textwidth}
        \begin{schemecode}
(define counter
  ((lambda (n)
    (lambda ()
      (set! n (+ n 1))
      n))
    0))
        \end{schemecode}
    \end{minipage}
\end{lrbox}

% ==============================
% ==         Документ         ==
% ==============================
\begin{document}

% Титульный слайд
\begin{frame}
    \titlepage
\end{frame}

% ============================
% ==        Глава 1.        ==
% ==                        ==
% == Оптимизация хвостовой  ==
% ==        рекурсии        ==
% ============================
\begin{frame}[fragile]{Глава 1. Оптимизация хвостовой рекурсии}
\centering

Рекурсивный вызов является \textbf{хвостовым}, если он — \alert{последнее действие} в выполняемой функции, и после возврата из рекурсивного вызова функции \alert{ничего больше делать не нужно}.

\end{frame}


\begin{frame}{1.1 Обычная рекурсия}
\centering
Начнём анализ рекурсии в Scheme с процедуры \mintinline{scheme}{f1}.
На её примере мы увидим, как \alert{REPL} вызывает
функцию,как формируются новые фреймы в стеке и как
происходит возврат результатов при
рекурсивных вызовах.

\usebox{\codeFactorialSimple}
\end{frame}

\begin{frame}[fragile]{1.1 Обычная рекурсия: Шаг 1. Начальный вызов}
\begin{columns}[c]

  % ======== левая колонка: код ========
  \column{0.4\textwidth}
  \centering
  \usebox{\codeFactorialSimple}
  \vspace{0.4cm}

  Среда \alert{REPL} вызывает процедуру \mintinline{scheme}{f1} с аргументом \mintinline{scheme}{3}

  \vfill

  % ======= правая колонка: схема ======
  \column{0.6\textwidth}
  \centering
  \begin{tikzpicture}[
      node distance=1.1cm,
      box/.style={
        draw=blue!70!black, fill=blue!5,
        rounded corners, thick, align=center,
        minimum width=3.8cm, minimum height=0.9cm,
        font=\footnotesize
      }
    ]
    \node[box] (repl) {\alert{REPL}};
    \node[box, below=of repl] (f1-1) {\mintinline{scheme}{(f1 n)}}
      edge [<-, thick, blue!70!black] (repl);
    \node[env-node=1, right=of f1-1] (stack1) { n $\mid$ 3 }
      edge [dashed, thick, blue!70!black] (f1-1);
    \end{tikzpicture}

\end{columns}
\end{frame}


\begin{frame}[fragile]{1.1 Обычная рекурсия: Шаг 2. Вызов \mintinline{scheme}{if}}
\begin{columns}[c]

  % ======== левая колонка: код ========
  \column{0.4\textwidth}
  \centering
  \usebox{\codeFactorialSimple}
  \vspace{0.4cm}

  Интерпретатор вызывает особую форму \mintinline{scheme}{if}

  \vfill

  % ======= правая колонка: схема ======
  \column{0.6\textwidth}
  \centering
  \begin{tikzpicture}[
      node distance=0.8cm,
      box/.style={
        draw=blue!70!black, fill=blue!5,
        rounded corners, thick, align=center,
        minimum width=3.8cm, minimum height=0.9cm,
        font=\footnotesize
      }
    ]
    \node[box] (repl) {\alert{REPL}};
    \node[box, below=of repl] (f1-1) {\mintinline{scheme}{(f1 3)}}
      edge [<-, thick, blue!70!black] (repl);
    \node[env-node=1, right=of f1-1] (stack1) { n $\mid$ 3 }
      edge [dashed, thick, blue!70!black] (f1-1);

    \node[box, below=of f1-1] (if-1) {\mintinline{scheme}{(if (= n 0))}}
      edge [<-, thick, blue!70!black] (f1-1);
    \end{tikzpicture}

\end{columns}
\end{frame}


\begin{frame}[fragile]{1.1 Обычная рекурсия: Шаг 3. Вычисление условия}
\begin{columns}[c]

  % ======== левая колонка: код ========
  \column{0.4\textwidth}
  \centering
  \usebox{\codeFactorialSimple}
  \vspace{0.4cm}

Интерпретатор спустился внутрь формы \mintinline{scheme}{if} и теперь вычисляет предикат \mintinline{scheme}{(= n 0)}.

\vfill

  % ======= правая колонка: схема ======
  \column{0.6\textwidth}
  \centering
  \begin{tikzpicture}[
      node distance=0.8cm,
      box/.style={
        draw=blue!70!black, fill=blue!5,
        rounded corners, thick, align=center,
        minimum width=3.8cm, minimum height=0.9cm,
        font=\footnotesize
      }
    ]
    \node[box] (repl) {\alert{REPL}};
    \node[box, below=of repl] (f1-1) {\mintinline{scheme}{(f1 3)}}
      edge [<-, thick, blue!70!black] (repl);
    \node[env-node=1, right=of f1-1] (stack1) { n $\mid$ 3 }
      edge [dashed, thick, blue!70!black] (f1-1);
    \node[box, below=of f1-1] (if-1) {\mintinline{scheme}{(if (= n 0))}}
      edge [<-, thick, blue!70!black] (f1-1);
    \node[box, below=of if-1] (eq-1) {\mintinline{scheme}{(= n 0)}}
      edge [<-, thick, blue!70!black] (if-1)
      edge [dashed, thick, blue!70!black, bend right=30] (stack1);
    \end{tikzpicture}

\end{columns}
\end{frame}


\begin{frame}[fragile]{1.1 Обычная рекурсия: Шаг 4. Условие ложно}
\begin{columns}[c]

  % ======== левая колонка: код ========
  \column{0.4\textwidth}
  \centering
  \usebox{\codeFactorialSimple}
  \vspace{0.4cm}

Вычисление предиката \mintinline{scheme}{(= n 0)} завершилось и вернуло значение \mintinline{scheme}{#f} (ложь)

  \vfill

  % ======= правая колонка: схема ======
  \column{0.6\textwidth}
  \centering
  \begin{tikzpicture}[
      node distance=0.8cm,
      box/.style={
        draw=blue!70!black, fill=blue!5,
        rounded corners, thick, align=center,
        minimum width=3.8cm, minimum height=0.9cm,
        font=\footnotesize
      }
    ]
    \node[box] (repl) {\alert{REPL}};
    \node[box, below=of repl] (f1-1) {\mintinline{scheme}{(f1 3)}}
      edge [<-, thick, blue!70!black] (repl);
    \node[env-node=1, right=of f1-1] (stack1) { n $\mid$ 3 }
      edge [dashed, thick, blue!70!black] (f1-1);
    \node[box, below=of f1-1] (if-1) {\mintinline{scheme}{(if (= n 0))}}
      edge [<-, thick, blue!70!black] (f1-1);
    \node[box, below=of if-1] (eq-1) {\mintinline{scheme}{#f}}
      edge [<-, thick, blue!70!black] (if-1)
      edge [->, thick, blue!70!black, bend left=30] (if-1);
    \end{tikzpicture}

\end{columns}
\end{frame}


\begin{frame}[fragile]{1.1 Обычная рекурсия: Шаг 5. Выбор ветки «иначе»}
\begin{columns}[c]

  % ======== левая колонка: код ========
  \column{0.4\textwidth}
  \centering
  \usebox{\codeFactorialSimple}
  \vspace{0.4cm}

Форма \mintinline{scheme}{if} получила от предиката значение \mintinline{scheme}{#f}.
Управление передаётся ветке c рекурсивным вызовом \mintinline{scheme}{(f1 (- n 1))}.

  \vfill

  % ======= правая колонка: схема ======
  \column{0.6\textwidth}
  \centering
  \begin{tikzpicture}[
      node distance=0.8cm,
      box/.style={
        draw=blue!70!black, fill=blue!5,
        rounded corners, thick, align=center,
        minimum width=3.8cm, minimum height=0.9cm,
        font=\footnotesize
      }
    ]
    \node[box] (repl) {\alert{REPL}};
    \node[box, below=of repl] (f1-1) {\mintinline{scheme}{(f1 3)}}
      edge [<-, thick, blue!70!black] (repl);
    \node[env-node=1, right=of f1-1] (stack1) { n $\mid$ 3 }
      edge [dashed, thick, blue!70!black] (f1-1);
    \node[box, below=of f1-1] (if-1) {\mintinline{scheme}{(if #f)}}
      edge [<-, thick, blue!70!black] (f1-1);
    \end{tikzpicture}

\end{columns}
\end{frame}


\begin{frame}[fragile]{1.1 Обычная рекурсия: Шаг 6. Тело обычной (нехвостовой) рекурсии}
\begin{columns}[c]

  % ======== левая колонка: код ========
  \column{0.4\textwidth}
  \centering
  \usebox{\codeFactorialSimple}
  \vspace{0.4cm}

После выбора ветки «иначе» интерпретатор начал вычислять тело процедуры — выражение \mintinline{scheme}{(* (f1 (- n 1)) n)}.

  \vfill

  % ======= правая колонка: схема ======
  \column{0.6\textwidth}
  \centering
  \begin{tikzpicture}[
      node distance=0.8cm,
      box/.style={
        draw=blue!70!black, fill=blue!5,
        rounded corners, thick, align=center,
        minimum width=3.8cm, minimum height=0.9cm,
        font=\footnotesize
      }
    ]
    \node[box] (repl) {\alert{REPL}};
    \node[box, below=of repl] (f1-1) {\mintinline{scheme}{(f1 3)}}
      edge [<-, thick, blue!70!black] (repl);
    \node[env-node=1, right=of f1-1] (stack1) { n $\mid$ 3 }
      edge [dashed, thick, blue!70!black] (f1-1);
    \node[box, below=of f1-1] (mul-1) {\mintinline{scheme}{(* (f1 (- n 1)) n)}}
      edge [<-, thick, blue!70!black] (f1-1)
      edge [dashed, thick, blue!70!black, bend right=10] (stack1);
    \end{tikzpicture}

\end{columns}
\end{frame}


\begin{frame}[fragile]{1.1 Обычная рекурсия: Шаг 7. Подготовка к рекурсивному вызову}
\begin{columns}[c]

  % ======== левая колонка: код ========
  \column{0.4\textwidth}
  \centering
  \usebox{\codeFactorialSimple}
  \vspace{0.4cm}

  Интерпретатор начал вычислять первый операнд умножения — выражение \mintinline{scheme}{(f1 (- n 1))}.

  \vfill

  % ======= правая колонка: схема ======
  \column{0.6\textwidth}
  \centering
  \begin{tikzpicture}[
      node distance=0.8cm,
      box/.style={
        draw=blue!70!black, fill=blue!5,
        rounded corners, thick, align=center,
        minimum width=3.8cm, minimum height=0.9cm,
        font=\footnotesize
      }
    ]
    \node[box] (repl) {\alert{REPL}};
    \node[box, below=of repl] (f1-1) {\mintinline{scheme}{(f1 3)}}
      edge [<-, thick, blue!70!black] (repl);
    \node[env-node=1, right=of f1-1] (stack1) { n $\mid$ 3 }
      edge [dashed, thick, blue!70!black] (f1-1);
    \node[box, below=of f1-1] (mul-1) {\mintinline{scheme}{(* (f1 (- n 1)) n)}}
      edge [<-, thick, blue!70!black] (f1-1)
      edge [dashed, thick, blue!70!black, bend right=10] (stack1);
    \node[box, below=of mul-1] (f1-2) {\mintinline{scheme}{(f1 (- n 1))}}
      edge [<-, thick, blue!70!black] (mul-1);
    \end{tikzpicture}
\end{columns}
\end{frame}

\begin{frame}[fragile]{1.1 Обычная рекурсия: Шаг 8. Вычисление фактического аргумента}
\begin{columns}[c]

  % ======== левая колонка: код ========
  \column{0.4\textwidth}
  \centering
  \usebox{\codeFactorialSimple}
  \vspace{0.4cm}

В новом вызове \mintinline{scheme}{f1} интерпретатор начал вычислять фактический параметр — выражение \mintinline{scheme}{(- n 1)}.
  \vfill

  % ======= правая колонка: схема ======
  \column{0.6\textwidth}
  \centering
  \begin{tikzpicture}[
      node distance=0.5cm,
      box/.style={
        draw=blue!70!black, fill=blue!5,
        rounded corners, thick, align=center,
        minimum width=3.8cm, minimum height=0.9cm,
        font=\footnotesize
      }
    ]
    \node[box] (repl) {\alert{REPL}};
    \node[box, below=of repl] (f1-1) {\mintinline{scheme}{(f1 3)}}
      edge [<-, thick, blue!70!black] (repl);
    \node[env-node=1, right=of f1-1] (stack1) { n $\mid$ 3 }
      edge [dashed, thick, blue!70!black] (f1-1);
    \node[box, below=of f1-1] (mul-1) {\mintinline{scheme}{(* (f1 (- n 1)) n)}}
      edge [<-, thick, blue!70!black] (f1-1)
      edge [dashed, thick, blue!70!black, bend right=10] (stack1);
    \node[box, below=of mul-1] (f1-2) {\mintinline{scheme}{(f1 (- n 1))}}
      edge [<-, thick, blue!70!black] (mul-1);
    \node[box, below=of f1-2] (sub-2) {\mintinline{scheme}{(- n 1)}}
      edge [<-, thick, blue!70!black] (f1-2)
      edge [dashed, thick, blue!70!black, bend right=40] (stack1);
    \end{tikzpicture}

\end{columns}
\end{frame}

  \begin{frame}[fragile]{1.1 Обычная рекурсия: Шаг 9. Аргумент вычислен — появляется второй кадр}
\begin{columns}[c]

  % ======== левая колонка: код ========
  \column{0.4\textwidth}
  \centering
  \usebox{\codeFactorialSimple}
  \vspace{0.4cm}

Выражение \mintinline{scheme}{(- n 1)} вернуло значение \mintinline{scheme}{  2}.
Интерпретатор связал его с параметром \mintinline{scheme}{ n } во втором вызове процедуры \mintinline{scheme}{ f1}.

  \vfill

  % ======= правая колонка: схема ======
  \column{0.6\textwidth}
  \centering
  \begin{tikzpicture}[
      node distance=0.5cm,
      box/.style={
        draw=blue!70!black, fill=blue!5,
        rounded corners, thick, align=center,
        minimum width=3.8cm, minimum height=0.9cm,
        font=\footnotesize
      }
    ]
    \node[box] (repl) {\alert{REPL}};
    \node[box, below=of repl] (f1-1) {\mintinline{scheme}{(f1 3)}}
      edge [<-, thick, blue!70!black] (repl);
    \node[env-node=1, right=of f1-1] (stack1) { n $\mid$ 3 }
      edge [dashed, thick, blue!70!black] (f1-1);
    \node[box, below=of f1-1] (mul-1) {\mintinline{scheme}{(* (f1 (- n 1)) n)}}
      edge [<-, thick, blue!70!black] (f1-1)
      edge [dashed, thick, blue!70!black, bend right=10] (stack1);
    \node[box, below=of mul-1] (f1-2) {\mintinline{scheme}{(f1 (- n 1))}}
      edge [<-, thick, blue!70!black] (mul-1);
    \node[box, below=of f1-2] (sub-2) {\mintinline{scheme}{2}}
      edge [<-, thick, blue!70!black] (f1-2)
      edge [dashed, thick, blue!70!black, bend right=40] (stack1)
      edge [->, thick, blue!70!black, bend left=30] (f1-2);
    \end{tikzpicture}

\end{columns}
\end{frame}

\begin{frame}[fragile]{1.1 Обычная рекурсия: Шаг 10. Второй уровень рекурсии активирован}
\begin{columns}[c]

  % ======== левая колонка: код ========
  \column{0.4\textwidth}
  \centering
  \usebox{\codeFactorialSimple}
  \vspace{0.4cm}

  Значение аргумента \mintinline{scheme}{  2} подставлено, второй кадр процедуры \mintinline{scheme}{ f1 } стал полноценным: теперь выполняется вызов \mintinline{scheme}{  (f1 2) }.

  \vfill

  % ======= правая колонка: схема ======
  \column{0.6\textwidth}
  \centering
  \begin{tikzpicture}[
      node distance=0.6cm,
      box/.style={
        draw=blue!70!black, fill=blue!5,
        rounded corners, thick, align=center,
        minimum width=3.8cm, minimum height=0.9cm,
        font=\footnotesize
      }
    ]
    \node[box] (repl) {\alert{REPL}};
    \node[box, below=of repl] (f1-1) {\mintinline{scheme}{(f1 3)}}
      edge [<-, thick, blue!70!black] (repl);
    \node[env-node=1, right=of f1-1] (stack1) { n $\mid$ 3 }
      edge [dashed, thick, blue!70!black] (f1-1);
    \node[box, below=of f1-1] (mul-1) {\mintinline{scheme}{(* (f1 (- n 1)) n)}}
      edge [<-, thick, blue!70!black] (f1-1)
      edge [dashed, thick, blue!70!black, bend right=10] (stack1);
    \node[box, below=of mul-1] (f1-2) {\mintinline{scheme}{(f1 2)}}
      edge [<-, thick, blue!70!black] (mul-1);
    \end{tikzpicture}

\end{columns}
\end{frame}

  \begin{frame}[fragile]{1.1 Обычная рекурсия: Шаг 11. Стек растёт}
\begin{columns}[c]

  % ======== левая колонка: код ========
  \column{0.4\textwidth}
  \centering
  \usebox{\codeFactorialSimple}
  \vspace{0.4cm}

Второй вызов \mintinline{scheme}{(f1 2)} полностью активирован — в стеке появился новый фрейм окружения с локальной переменной \mintinline{scheme}{n} $\mapsto$ \mintinline{scheme}{2}

  \vfill

  % ======= правая колонка: схема ======
  \column{0.6\textwidth}
  \centering
  \begin{tikzpicture}[
      node distance=0.6cm,
      box/.style={
        draw=blue!70!black, fill=blue!5,
        rounded corners, thick, align=center,
        minimum width=3.8cm, minimum height=0.9cm,
        font=\footnotesize
      }
    ]
    \node[box] (repl) {\alert{REPL}};
    \node[box, below=of repl] (f1-1) {\mintinline{scheme}{(f1 3)}}
      edge [<-, thick, blue!70!black] (repl);
    \node[env-node=1, right=of f1-1] (stack1) { n $\mid$ 3 }
      edge [dashed, thick, blue!70!black] (f1-1);
    \node[box, below=of f1-1] (mul-1) {\mintinline{scheme}{(* (f1 (- n 1)) n)}}
      edge [<-, thick, blue!70!black] (f1-1)
      edge [dashed, thick, blue!70!black, bend right=10] (stack1);
    \node[box, below=of mul-1] (f1-2) {\mintinline{scheme}{(f1 2)}}
      edge [<-, thick, blue!70!black] (mul-1);
    \node[env-node=1, right=of f1-2] (stack2) { n $\mid$ 2 }
      edge [dashed, thick, blue!70!black] (f1-2);
    \end{tikzpicture}

\end{columns}
\end{frame}

    \begin{frame}[fragile]{1.1 Обычная рекурсия: Шаг 12. Провал нехвостовой рекурсии}
\begin{columns}[c]

  % ======== левая колонка: код ========
  \column{0.4\textwidth}
  \centering
  \usebox{\codeFactorialSimple}
  \vspace{0.4cm}

  Всего за три шага рекурсии (от \mintinline{scheme}{ n = 3 }  до \mintinline{scheme}{ n = 0 }) стек вырос до восьми активных кадров

  \vfill

  % ======= правая колонка: схема ======
  \column{0.6\textwidth}
  \centering
  \begin{tikzpicture}[
      node distance=0.28cm,
      box/.style={
        draw=blue!70!black, fill=blue!5,
        rounded corners, thick, align=center,
        minimum width=3.8cm, minimum height=0.5cm,
        font=\footnotesize
      }
    ]
    \node[box] (repl) {\alert{REPL}};
    \node[box, below=of repl] (f1-1) {\mintinline{scheme}{(f1 3)}}
      edge [<-, thick, blue!70!black] (repl);
    \node[env-node=1, right=of f1-1] (stack1) { n $\mid$ 3 }
      edge [dashed, thick, blue!70!black] (f1-1);
    \node[box, below=of f1-1] (mul-1) {\mintinline{scheme}{(* (f1 (- n 1)) n)}}
      edge [<-, thick, blue!70!black] (f1-1)
      edge [dashed, thick, blue!70!black, bend right=10] (stack1);
    \node[box, below=of mul-1] (f1-2) {\mintinline{scheme}{(f1 2)}}
      edge [<-, thick, blue!70!black] (mul-1);
    \node[env-node=1, right=of f1-2] (stack2) { n $\mid$ 2 }
      edge [dashed, thick, blue!70!black] (f1-2);
    \node[box, below=of f1-2] (mul-2) {\mintinline{scheme}{(* (f1 (- n 1)) n)}}
      edge [<-, thick, blue!70!black] (f1-2)
      edge [dashed, thick, blue!70!black, bend right=10] (stack2);

    \node[box, below=of mul-2] (f1-3) {\mintinline{scheme}{(f1 1)}}
      edge [<-, thick, blue!70!black] (mul-2);
    \node[env-node=1, right=of f1-3] (stack3) { n $\mid$ 1 }
      edge [dashed, thick, blue!70!black] (f1-3);
    \node[box, below=of f1-3] (mul-3) {\mintinline{scheme}{(* (f1 (- n 1)) n)}}
      edge [<-, thick, blue!70!black] (f1-3)
      edge [dashed, thick, blue!70!black, bend right=10] (stack3);

    \node[box, below=of mul-3] (f1-4) {\mintinline{scheme}{(f1 0)}}
      edge [<-, thick, blue!70!black] (mul-3);
    \node[env-node=1, right=of f1-4] (stack3) { n $\mid$ 0 }
      edge [dashed, thick, blue!70!black] (f1-4);
    \end{tikzpicture}

\end{columns}
\end{frame}

    \begin{frame}[fragile]{1.1 Обычная рекурсия: Шаг 13. Начало сворачивания}
\begin{columns}[c]

  % ======== левая колонка: код ========
  \column{0.4\textwidth}
  \centering
  \usebox{\codeFactorialSimple}
  \vspace{0.4cm}

  Самый глубокий вызов \mintinline{scheme}{  (f1 0) } попал в ветку \mintinline{scheme}{  (if (= n 0) 1 …) }  и вернул значение \mintinline{scheme}{ 1  }.
Это первый результат, который пошёл вверх по цепочке.

  \vfill

  % ======= правая колонка: схема ======
  \column{0.6\textwidth}
  \centering
  \begin{tikzpicture}[
      node distance=0.28cm,
      box/.style={
        draw=blue!70!black, fill=blue!5,
        rounded corners, thick, align=center,
        minimum width=3.8cm, minimum height=0.5cm,
        font=\footnotesize
      }
    ]
    \node[box] (repl) {\alert{REPL}};
    \node[box, below=of repl] (f1-1) {\mintinline{scheme}{(f1 3)}}
      edge [<-, thick, blue!70!black] (repl);
    \node[env-node=1, right=of f1-1] (stack1) { n $\mid$ 3 }
      edge [dashed, thick, blue!70!black] (f1-1);
    \node[box, below=of f1-1] (mul-1) {\mintinline{scheme}{(* (f1 (- n 1)) n)}}
      edge [<-, thick, blue!70!black] (f1-1)
      edge [dashed, thick, blue!70!black, bend right=10] (stack1);
    \node[box, below=of mul-1] (f1-2) {\mintinline{scheme}{(f1 2)}}
      edge [<-, thick, blue!70!black] (mul-1);
    \node[env-node=1, right=of f1-2] (stack2) { n $\mid$ 2 }
      edge [dashed, thick, blue!70!black] (f1-2);
    \node[box, below=of f1-2] (mul-2) {\mintinline{scheme}{(* (f1 (- n 1)) n)}}
      edge [<-, thick, blue!70!black] (f1-2)
      edge [dashed, thick, blue!70!black, bend right=10] (stack2);

    \node[box, below=of mul-2] (f1-3) {\mintinline{scheme}{(f1 1)}}
      edge [<-, thick, blue!70!black] (mul-2);
    \node[env-node=1, right=of f1-3] (stack3) { n $\mid$ 1 }
      edge [dashed, thick, blue!70!black] (f1-3);
    \node[box, below=of f1-3] (mul-3) {\mintinline{scheme}{(* (f1 (- n 1)) n)}}
      edge [<-, thick, blue!70!black] (f1-3)
      edge [dashed, thick, blue!70!black, bend right=10] (stack3);

    \node[box, below=of mul-3] (f1-4) {\mintinline{scheme}{1}}
      edge [<-, thick, blue!70!black] (mul-3)
      edge [->, thick, blue!70!black, bend left=40] (mul-3);
    \node[env-node=1, right=of f1-4] (stack3) { n $\mid$ 0 }
      edge [dashed, thick, blue!70!black] (f1-4);
    \end{tikzpicture}

\end{columns}
\end{frame}


    \begin{frame}[fragile]{1.1 Обычная рекурсия: Шаг 14. Сворачивание стека: первое умножение выполнено}
\begin{columns}[c]

  % ======== левая колонка: код ========
  \column{0.4\textwidth}
  \centering
  \usebox{\codeFactorialSimple}
  \vspace{0.4cm}

  Самый глубокий кадр вернул \mintinline{scheme}{ 1 }.

  \vfill

  % ======= правая колонка: схема ======
  \column{0.6\textwidth}
  \centering
  \begin{tikzpicture}[
      node distance=0.32cm,
      box/.style={
        draw=blue!70!black, fill=blue!5,
        rounded corners, thick, align=center,
        minimum width=3.8cm, minimum height=0.5cm,
        font=\footnotesize
      }
    ]
    \node[box] (repl) {\alert{REPL}};
    \node[box, below=of repl] (f1-1) {\mintinline{scheme}{(f1 3)}}
      edge [<-, thick, blue!70!black] (repl);
    \node[env-node=1, right=of f1-1] (stack1) { n $\mid$ 3 }
      edge [dashed, thick, blue!70!black] (f1-1);
    \node[box, below=of f1-1] (mul-1) {\mintinline{scheme}{(* (f1 (- n 1)) n)}}
      edge [<-, thick, blue!70!black] (f1-1)
      edge [dashed, thick, blue!70!black, bend right=10] (stack1);
    \node[box, below=of mul-1] (f1-2) {\mintinline{scheme}{(f1 2)}}
      edge [<-, thick, blue!70!black] (mul-1);
    \node[env-node=1, right=of f1-2] (stack2) { n $\mid$ 2 }
      edge [dashed, thick, blue!70!black] (f1-2);
    \node[box, below=of f1-2] (mul-2) {\mintinline{scheme}{(* (f1 (- n 1)) n)}}
      edge [<-, thick, blue!70!black] (f1-2)
      edge [dashed, thick, blue!70!black, bend right=10] (stack2);

    \node[box, below=of mul-2] (f1-3) {\mintinline{scheme}{(f1 1)}}
      edge [<-, thick, blue!70!black] (mul-2);
    \node[env-node=1, right=of f1-3] (stack3) { n $\mid$ 1 }
      edge [dashed, thick, blue!70!black] (f1-3);
    \node[box, below=of f1-3] (mul-3) {\mintinline{scheme}{(* 1 1)}}
      edge [<-, thick, blue!70!black] (f1-3)
      edge [dashed, thick, blue!70!black, bend right=10] (stack3);

    \end{tikzpicture}

\end{columns}
\end{frame}


    \begin{frame}[fragile]{1.1 Обычная рекурсия: Шаг 15. Умножение на самом глубоком уровне завершено}
\begin{columns}[c]

  % ======== левая колонка: код ========
  \column{0.4\textwidth}
  \centering
  \usebox{\codeFactorialSimple}
  \vspace{0.4cm}

  Выражение \mintinline{scheme}{(* 1 1)} вычислено и вернуло \mintinline{scheme}{1}.
Кадр самого нижнего умножения полностью отработал и передал результат \mintinline{scheme}{1} вверх — в кадр процедуры \mintinline{scheme}{(f1 1)}.

  \vfill

  % ======= правая колонка: схема ======
  \column{0.6\textwidth}
  \centering
  \begin{tikzpicture}[
      node distance=0.32cm,
      box/.style={
        draw=blue!70!black, fill=blue!5,
        rounded corners, thick, align=center,
        minimum width=3.8cm, minimum height=0.5cm,
        font=\footnotesize
      }
    ]
    \node[box] (repl) {\alert{REPL}};
    \node[box, below=of repl] (f1-1) {\mintinline{scheme}{(f1 3)}}
      edge [<-, thick, blue!70!black] (repl);
    \node[env-node=1, right=of f1-1] (stack1) { n $\mid$ 3 }
      edge [dashed, thick, blue!70!black] (f1-1);
    \node[box, below=of f1-1] (mul-1) {\mintinline{scheme}{(* (f1 (- n 1)) n)}}
      edge [<-, thick, blue!70!black] (f1-1)
      edge [dashed, thick, blue!70!black, bend right=10] (stack1);
    \node[box, below=of mul-1] (f1-2) {\mintinline{scheme}{(f1 2)}}
      edge [<-, thick, blue!70!black] (mul-1);
    \node[env-node=1, right=of f1-2] (stack2) { n $\mid$ 2 }
      edge [dashed, thick, blue!70!black] (f1-2);
    \node[box, below=of f1-2] (mul-2) {\mintinline{scheme}{(* (f1 (- n 1)) n)}}
      edge [<-, thick, blue!70!black] (f1-2)
      edge [dashed, thick, blue!70!black, bend right=10] (stack2);

    \node[box, below=of mul-2] (f1-3) {\mintinline{scheme}{(f1 1)}}
      edge [<-, thick, blue!70!black] (mul-2);
    \node[env-node=1, right=of f1-3] (stack3) { n $\mid$ 1 }
      edge [dashed, thick, blue!70!black] (f1-3);
    \node[box, below=of f1-3] (mul-3) {\mintinline{scheme}{1}}
      edge [<-, thick, blue!70!black] (f1-3)
      edge [->, thick, blue!70!black, bend left=30] (f1-3)
      edge [dashed, thick, blue!70!black, bend right=10] (stack3);

    \end{tikzpicture}

\end{columns}
\end{frame}


    \begin{frame}[fragile]{1.1 Обычная рекурсия: Шаг 16. Второй уровень рекурсии завершён}
\begin{columns}[c]

  % ======== левая колонка: код ========
  \column{0.4\textwidth}
  \centering
  \usebox{\codeFactorialSimple}
  \vspace{0.4cm}

  Вызов \mintinline{scheme}{  (f1 1) } полностью отработал и вернул значение \mintinline{scheme}{  1 }.

  \vfill

  % ======= правая колонка: схема ======
  \column{0.6\textwidth}
  \centering
  \begin{tikzpicture}[
      node distance=0.34cm,
      box/.style={
        draw=blue!70!black, fill=blue!5,
        rounded corners, thick, align=center,
        minimum width=3.8cm, minimum height=0.5cm,
        font=\footnotesize
      }
    ]
    \node[box] (repl) {\alert{REPL}};
    \node[box, below=of repl] (f1-1) {\mintinline{scheme}{(f1 3)}}
      edge [<-, thick, blue!70!black] (repl);
    \node[env-node=1, right=of f1-1] (stack1) { n $\mid$ 3 }
      edge [dashed, thick, blue!70!black] (f1-1);
    \node[box, below=of f1-1] (mul-1) {\mintinline{scheme}{(* (f1 (- n 1)) n)}}
      edge [<-, thick, blue!70!black] (f1-1)
      edge [dashed, thick, blue!70!black, bend right=10] (stack1);
    \node[box, below=of mul-1] (f1-2) {\mintinline{scheme}{(f1 2)}}
      edge [<-, thick, blue!70!black] (mul-1);
    \node[env-node=1, right=of f1-2] (stack2) { n $\mid$ 2 }
      edge [dashed, thick, blue!70!black] (f1-2);
    \node[box, below=of f1-2] (mul-2) {\mintinline{scheme}{(* (f1 (- n 1)) n)}}
      edge [<-, thick, blue!70!black] (f1-2)
      edge [dashed, thick, blue!70!black, bend right=10] (stack2);

    \node[box, below=of mul-2] (f1-3) {\mintinline{scheme}{1}}
      edge [<-, thick, blue!70!black] (mul-2)
      edge [->, thick, blue!70!black, bend left=30] (mul-2);
    \end{tikzpicture}

\end{columns}
\end{frame}


    \begin{frame}[fragile]{1.1 Обычная рекурсия: Шаг 17. Предпоследний шаг сворачивания}
\begin{columns}[c]

  % ======== левая колонка: код ========
  \column{0.4\textwidth}
  \centering
  \usebox{\codeFactorialSimple}
  \vspace{0.4cm}

  Значение \mintinline{scheme}{ 1 } от \mintinline{scheme}{(f1 1)} уже превратилось во втором уровне в \mintinline{scheme}{2}.

  \vfill

  % ======= правая колонка: схема ======
  \column{0.6\textwidth}
  \centering
  \begin{tikzpicture}[
      node distance=0.34cm,
      box/.style={
        draw=blue!70!black, fill=blue!5,
        rounded corners, thick, align=center,
        minimum width=3.8cm, minimum height=0.5cm,
        font=\footnotesize
      }
    ]
    \node[box] (repl) {\alert{REPL}};
    \node[box, below=of repl] (f1-1) {\mintinline{scheme}{(f1 3)}}
      edge [<-, thick, blue!70!black] (repl);
    \node[env-node=1, right=of f1-1] (stack1) { n $\mid$ 3 }
      edge [dashed, thick, blue!70!black] (f1-1);
    \node[box, below=of f1-1] (mul-1) {\mintinline{scheme}{(* (f1 (- n 1)) n)}}
      edge [<-, thick, blue!70!black] (f1-1)
      edge [dashed, thick, blue!70!black, bend right=10] (stack1);
    \node[box, below=of mul-1] (f1-2) {\mintinline{scheme}{(f1 2)}}
      edge [<-, thick, blue!70!black] (mul-1);
    \node[env-node=1, right=of f1-2] (stack2) { n $\mid$ 2 }
      edge [dashed, thick, blue!70!black] (f1-2);
    \node[box, below=of f1-2] (mul-2) {\mintinline{scheme}{(* 1 2)}}
      edge [<-, thick, blue!70!black] (f1-2)
      edge [dashed, thick, blue!70!black, bend right=10] (stack2);
    \end{tikzpicture}

\end{columns}
\end{frame}


    \begin{frame}[fragile]{1.1 Обычная рекурсия: Шаг 18. Последнее умножение перед завершением}
\begin{columns}[c]

  % ======== левая колонка: код ========
  \column{0.4\textwidth}
  \centering
  \usebox{\codeFactorialSimple}
  \vspace{0.4cm}

  Умножение второго уровня завершено и вернуло \mintinline{scheme}{2}  .
Этот результат подставлен в самое первое, «самое долго ждавшее» умножение: \mintinline{scheme}{  (* 2 n) } $\to$ \mintinline{scheme}{  (* 2 3) }.

  \vfill

  % ======= правая колонка: схема ======
  \column{0.6\textwidth}
  \centering
  \begin{tikzpicture}[
      node distance=0.36cm,
      box/.style={
        draw=blue!70!black, fill=blue!5,
        rounded corners, thick, align=center,
        minimum width=3.8cm, minimum height=0.5cm,
        font=\footnotesize
      }
    ]
    \node[box] (repl) {\alert{REPL}};
    \node[box, below=of repl] (f1-1) {\mintinline{scheme}{(f1 3)}}
      edge [<-, thick, blue!70!black] (repl);
    \node[env-node=1, right=of f1-1] (stack1) { n $\mid$ 3 }
      edge [dashed, thick, blue!70!black] (f1-1);
    \node[box, below=of f1-1] (mul-1) {\mintinline{scheme}{(* (f1 (- n 1)) n)}}
      edge [<-, thick, blue!70!black] (f1-1)
      edge [dashed, thick, blue!70!black, bend right=10] (stack1);
    \node[box, below=of mul-1] (f1-2) {\mintinline{scheme}{(f1 2)}}
      edge [<-, thick, blue!70!black] (mul-1);
    \node[env-node=1, right=of f1-2] (stack2) { n $\mid$ 2 }
      edge [dashed, thick, blue!70!black] (f1-2);
    \node[box, below=of f1-2] (mul-2) {\mintinline{scheme}{2}}
      edge [<-, thick, blue!70!black] (f1-2)
      edge [->, thick, blue!70!black, bend left=30] (f1-2)
      edge [dashed, thick, blue!70!black, bend right=10] (stack2);
    \end{tikzpicture}

\end{columns}
\end{frame}


    \begin{frame}[fragile]{1.1 Обычная рекурсия: Шаг 19. Самое последнее действие нехвостовой рекурсии}
\begin{columns}[c]

  % ======== левая колонка: код ========
  \column{0.4\textwidth}
  \centering
  \usebox{\codeFactorialSimple}
  \vspace{0.4cm}

  Выполняется финальное умножение \mintinline{scheme}{  (* 2 3) }.

  \vfill

  % ======= правая колонка: схема ======
  \column{0.6\textwidth}
  \centering
  \begin{tikzpicture}[
      node distance=0.36cm,
      box/.style={
        draw=blue!70!black, fill=blue!5,
        rounded corners, thick, align=center,
        minimum width=3.8cm, minimum height=0.5cm,
        font=\footnotesize
      }
    ]
    \node[box] (repl) {\alert{REPL}};
    \node[box, below=of repl] (f1-1) {\mintinline{scheme}{(f1 3)}}
      edge [<-, thick, blue!70!black] (repl);
    \node[env-node=1, right=of f1-1] (stack1) { n $\mid$ 3 }
      edge [dashed, thick, blue!70!black] (f1-1);
    \node[box, below=of f1-1] (mul-1) {\mintinline{scheme}{(* (f1 (- n 1)) n)}}
      edge [<-, thick, blue!70!black] (f1-1)
      edge [dashed, thick, blue!70!black, bend right=10] (stack1);
    \node[box, below=of mul-1] (f1-2) {\mintinline{scheme}{2}}
      edge [<-, thick, blue!70!black] (mul-1)
      edge [->, thick, blue!70!black, bend left=30] (mul-1);
    \end{tikzpicture}

\end{columns}
\end{frame}

    \begin{frame}[fragile]{1.1 Обычная рекурсия: Шаг 20. Финальный кадр обычной рекурсии}
\begin{columns}[c]

  % ======== левая колонка: код ========
  \column{0.4\textwidth}
  \centering
  \usebox{\codeFactorialSimple}
  \vspace{0.4cm}

  Выполняется последнее умножение \mintinline{scheme}{  (* 2 3) }.

  \vfill

  % ======= правая колонка: схема ======
  \column{0.6\textwidth}
  \centering
  \begin{tikzpicture}[
      node distance=0.38cm,
      box/.style={
        draw=blue!70!black, fill=blue!5,
        rounded corners, thick, align=center,
        minimum width=3.8cm, minimum height=0.5cm,
        font=\footnotesize
      }
    ]
    \node[box] (repl) {\alert{REPL}};
    \node[box, below=of repl] (f1-1) {\mintinline{scheme}{(f1 3)}}
      edge [<-, thick, blue!70!black] (repl);
    \node[env-node=1, right=of f1-1] (stack1) { n $\mid$ 3 }
      edge [dashed, thick, blue!70!black] (f1-1);
    \node[box, below=of f1-1] (mul-1) {\mintinline{scheme}{(* 2 3)}}
      edge [<-, thick, blue!70!black] (f1-1)
      edge [dashed, thick, blue!70!black, bend right=10] (stack1);
    \end{tikzpicture}

\end{columns}
\end{frame}


    \begin{frame}[fragile]{1.1 Обычная рекурсия: Шаг 21. Рекурсия завершена — результат готов}
\begin{columns}[c]

  % ======== левая колонка: код ========
  \column{0.4\textwidth}
  \centering
  \usebox{\codeFactorialSimple}
  \vspace{0.4cm}

  Умножение \mintinline{scheme}{  (* 2 3) } вернуло \mintinline{scheme}{  6}.
Это и есть окончательное значение всего выражения \mintinline{scheme}{  (f1 3)}.

  \vfill

  % ======= правая колонка: схема ======
  \column{0.6\textwidth}
  \centering
  \begin{tikzpicture}[
      node distance=0.38cm,
      box/.style={
        draw=blue!70!black, fill=blue!5,
        rounded corners, thick, align=center,
        minimum width=3.8cm, minimum height=0.5cm,
        font=\footnotesize
      }
    ]
    \node[box] (repl) {\alert{REPL}};
    \node[box, below=of repl] (f1-1) {\mintinline{scheme}{(f1 3)}}
      edge [<-, thick, blue!70!black] (repl);
    \node[env-node=1, right=of f1-1] (stack1) { n $\mid$ 3 }
      edge [dashed, thick, blue!70!black] (f1-1);
    \node[box, below=of f1-1] (mul-1) {\mintinline{scheme}{6}}
      edge [<-, thick, blue!70!black] (f1-1)
      edge [->, thick, blue!70!black, bend left=30] (f1-1)
      edge [dashed, thick, blue!70!black, bend right=10] (stack1);
    \end{tikzpicture}

\end{columns}
\end{frame}


    \begin{frame}[fragile]{1.1 Обычная рекурсия: Шаг 22. Обычная рекурсия завершена}
\begin{columns}[c]

  % ======== левая колонка: код ========
  \column{0.4\textwidth}
  \centering
  \usebox{\codeFactorialSimple}
  \vspace{0.4cm}

  Значение \mintinline{scheme}{  6 } возвращено в \alert{REPL}.

  \vfill

  % ======= правая колонка: схема ======
  \column{0.6\textwidth}
  \centering
  \begin{tikzpicture}[
      node distance=0.38cm,
      box/.style={
        draw=blue!70!black, fill=blue!5,
        rounded corners, thick, align=center,
        minimum width=3.8cm, minimum height=0.5cm,
        font=\footnotesize
      }
    ]
    \node[box] (repl) {\alert{REPL}};
    \node[box, below=of repl] (f1-1) {\mintinline{scheme}{6}}
      edge [<-, thick, blue!70!black] (repl);
    \end{tikzpicture}

\end{columns}
\end{frame}
\begin{frame}{1.2 Хвостовая рекурсия}
    \centering
    Продолжим анализ процедуры \mintinline{scheme}{f2}. Посмотрим как Scheme превращает красивую рекурсию в настоящую итерацию под капотом

    \usebox{\codeFactorialTailRec}
\end{frame}

\begin{frame}[fragile]{1.2 Хвостовая рекурсия: Шаг 1. Начальный вызов хвостовой рекурсии}
    \begin{columns}[c]

        % ======== левая колонка: код ========
        \column{0.4\textwidth}
        \centering
        \usebox{\codeFactorialTailRec}
        \vspace{0.4cm}

        Среда \alert{REPL} вызывает процедуру \mintinline{scheme}{f2} с аргументами \mintinline{scheme}{n = 3} и аккумулятором \mintinline{scheme}{acc = 1}

        \vfill

        % ======= правая колонка: схема ======
        \column{0.6\textwidth}
        \centering
        \begin{tikzpicture}[
            node distance=0.6cm,
            box/.style={
                draw=blue!70!black, fill=blue!5,
                rounded corners, thick, align=center,
                minimum width=3.8cm, minimum height=0.9cm,
                font=\footnotesize
            }
        ]
            \node[box] (repl) {\alert{REPL}};
            \node[box, below=of repl] (f2-1) {\mintinline{scheme}{(f2 3 1)}}
            edge [<-, thick, blue!70!black] (repl);
        \end{tikzpicture}

    \end{columns}
\end{frame}


\begin{frame}[fragile]{1.2 Хвостовая рекурсия: Шаг 2. Проверка условия завершения}
    \begin{columns}[c]

        % ======== левая колонка: код ========
        \column{0.4\textwidth}
        \centering
        \usebox{\codeFactorialTailRec}
        \vspace{0.4cm}

        Внутри процедуры \mintinline{scheme}{f2} интерпретатор начинает вычислять форму \mintinline{scheme}{(if (= n 0))}.

        \vfill

        % ======= правая колонка: схема ======
        \column{0.6\textwidth}
        \centering
        \begin{tikzpicture}[
            node distance=0.6cm,
            box/.style={
                draw=blue!70!black, fill=blue!5,
                rounded corners, thick, align=center,
                minimum width=3.8cm, minimum height=0.9cm,
                font=\footnotesize
            }
        ]
            \node[box] (repl) {\alert{REPL}};
            \node[box, below=of repl] (f1-1) {\mintinline{scheme}{(f2 3 1)}}
            edge [<-, thick, blue!70!black] (repl);
            \node[env-node=2, right=of f1-1] (stack1) { n $\mid$ 3 \nodepart{two} acc $\mid$ 1 }
            edge [dashed, thick, blue!70!black] (f1-1);
            \node[box, below=of f1-1] (mul-1) {\mintinline{scheme}{ (if (= n 0)) }}
            edge [dashed, thick, blue!70!black] (stack1)
            edge [<-, thick, blue!70!black] (f1-1);
        \end{tikzpicture}

    \end{columns}
\end{frame}


\begin{frame}[fragile]{1.2 Хвостовая рекурсия: Шаг 3. Вычисление предиката условия}
    \begin{columns}[c]

        % ======== левая колонка: код ========
        \column{0.4\textwidth}
        \centering
        \usebox{\codeFactorialTailRec}
        \vspace{0.4cm}

        Интерпретатор переходит к вычислению предиката ветки \mintinline{scheme}{if} — выражения \mintinline{scheme}{(= n 0)}.

        \vfill

        % ======= правая колонка: схема ======
        \column{0.6\textwidth}
        \centering
        \begin{tikzpicture}[
            node distance=0.6cm,
            box/.style={
                draw=blue!70!black, fill=blue!5,
                rounded corners, thick, align=center,
                minimum width=3.8cm, minimum height=0.9cm,
                font=\footnotesize
            }
        ]
            \node[box] (repl) {\alert{REPL}};
            \node[box, below=of repl] (f1-1) {\mintinline{scheme}{(f2 3 1)}}
            edge [<-, thick, blue!70!black] (repl);
            \node[env-node=2, right=of f1-1] (stack1) { n $\mid$ 3 \nodepart{two} acc $\mid$ 1 }
            edge [dashed, thick, blue!70!black] (f1-1);
            \node[box, below=of f1-1] (mul-1) {\mintinline{scheme}{ (if (= n 0)) }}
            edge [dashed, thick, blue!70!black] (stack1)
            edge [<-, thick, blue!70!black] (f1-1);
            \node[box, below=of mul-1] (eq1) {\mintinline{scheme}{ (= n 0) }}
            edge [dashed, thick, blue!70!black, bend right=30] (stack1)
            edge [<-, thick, blue!70!black] (mul-1);
        \end{tikzpicture}

    \end{columns}
\end{frame}


\begin{frame}[fragile]{1.2 Хвостовая рекурсия: Шаг 4. Условие ложно $\to$  выбирается ветка «иначе»}
    \begin{columns}[c]

        % ======== левая колонка: код ========
        \column{0.4\textwidth}
        \centering
        \usebox{\codeFactorialTailRec}
        \vspace{0.4cm}

        Вычисление предиката \mintinline{scheme}{(= n 0)} завершилось значением \mintinline{scheme}{#f}.

        \vfill

        % ======= правая колонка: схема ======
        \column{0.6\textwidth}
        \centering
        \begin{tikzpicture}[
            node distance=0.6cm,
            box/.style={
                draw=blue!70!black, fill=blue!5,
                rounded corners, thick, align=center,
                minimum width=3.8cm, minimum height=0.9cm,
                font=\footnotesize
            }
        ]
            \node[box] (repl) {\alert{REPL}};
            \node[box, below=of repl] (f1-1) {\mintinline{scheme}{(f2 3 1)}}
            edge [<-, thick, blue!70!black] (repl);
            \node[env-node=2, right=of f1-1] (stack1) { n $\mid$ 3 \nodepart{two} acc $\mid$ 1 }
            edge [dashed, thick, blue!70!black] (f1-1);
            \node[box, below=of f1-1] (mul-1) {\mintinline{scheme}{ (if (= n 0)) }}
            edge [dashed, thick, blue!70!black] (stack1)
            edge [<-, thick, blue!70!black] (f1-1);
            \node[box, below=of mul-1] (eq1) {\mintinline{scheme}{ #f }}
            edge [dashed, thick, blue!70!black, bend right=30] (stack1)
            edge [<-, thick, blue!70!black] (mul-1)
            edge [->, thick, blue!70!black, bend left=30] (mul-1);
        \end{tikzpicture}

    \end{columns}
\end{frame}


\begin{frame}[fragile]{1.2 Хвостовая рекурсия: Шаг 5. Подготовка к хвостовому рекурсивному вызову}
    \begin{columns}[c]

        % ======== левая колонка: код ========
        \column{0.4\textwidth}
        \centering
        \usebox{\codeFactorialTailRec}
        \vspace{0.4cm}

        Форма \mintinline{scheme}{if} уже получила значение предиката \mintinline{scheme}{#f} и «свернулась» до альтернативной ветки.

        \vfill

        % ======= правая колонка: схема ======
        \column{0.6\textwidth}
        \centering
        \begin{tikzpicture}[
            node distance=0.6cm,
            box/.style={
                draw=blue!70!black, fill=blue!5,
                rounded corners, thick, align=center,
                minimum width=3.8cm, minimum height=0.9cm,
                font=\footnotesize
            }
        ]
            \node[box] (repl) {\alert{REPL}};
            \node[box, below=of repl] (f1-1) {\mintinline{scheme}{(f2 3 1)}}
            edge [<-, thick, blue!70!black] (repl);
            \node[env-node=2, right=of f1-1] (stack1) { n $\mid$ 3 \nodepart{two} acc $\mid$ 1 }
            edge [dashed, thick, blue!70!black] (f1-1);
            \node[box, below=of f1-1] (mul-1) {\mintinline{scheme}{ (if #f) }}
            edge [dashed, thick, blue!70!black] (stack1)
            edge [<-, thick, blue!70!black] (f1-1);
        \end{tikzpicture}

    \end{columns}
\end{frame}


\begin{frame}[fragile]{1.2 Хвостовая рекурсия: Шаг 6. Хвостовой рекурсивный вызов готов к выполнению}
    \begin{columns}[c]

        % ======== левая колонка: код ========
        \column{0.4\textwidth}
        \centering
        \usebox{\codeFactorialTailRec}
        \vspace{0.4cm}

        Форма if полностью свелась к телу рекурсивного вызова — выражению \mintinline{scheme}{(f2 (- n 1) (* n acc))}.

        \vfill

        % ======= правая колонка: схема ======
        \column{0.6\textwidth}
        \centering
        \begin{tikzpicture}[
            node distance=0.6cm,
            box/.style={
                draw=blue!70!black, fill=blue!5,
                rounded corners, thick, align=center,
                minimum width=3.8cm, minimum height=0.9cm,
                font=\footnotesize
            }
        ]
            \node[box] (repl) {\alert{REPL}};
            \node[box, below=of repl] (f1-1) {\mintinline{scheme}{(f2 3 1)}}
            edge [<-, thick, blue!70!black] (repl);
            \node[env-node=2, right=of f1-1] (stack1) { n $\mid$ 3 \nodepart{two} acc $\mid$ 1 }
            edge [dashed, thick, blue!70!black] (f1-1);
            \node[box, below=of f1-1] (mul-1) {\mintinline{scheme}{ (f2 (- n 1) (* n acc)) }}
            edge [dashed, thick, blue!70!black] (stack1)
            edge [<-, thick, blue!70!black] (f1-1);
        \end{tikzpicture}

    \end{columns}
\end{frame}


\begin{frame}[fragile]{1.2 Хвостовая рекурсия: Шаг 7. Вычисление аргументов перед хвостовым вызовом}
    \begin{columns}[c]

        % ======== левая колонка: код ========
        \column{0.4\textwidth}
        \centering
        \usebox{\codeFactorialTailRec}
        \vspace{0.4cm}

        Интерпретатор начал вычислять аргументы рекурсивного вызова \mintinline{scheme}{(f2 (- n 1) (* n acc))}.

        \vfill

        % ======= правая колонка: схема ======
        \column{0.6\textwidth}
        \centering
        \begin{tikzpicture}[
            node distance=0.6cm,
            box/.style={
                draw=blue!70!black, fill=blue!5,
                rounded corners, thick, align=center,
                minimum width=3.8cm, minimum height=0.9cm,
                font=\footnotesize
            }
        ]
            \node[box] (repl) {\alert{REPL}};
            \node[box, below=of repl] (f1-1) {\mintinline{scheme}{(f2 3 1)}}
            edge [<-, thick, blue!70!black] (repl);
            \node[env-node=2, right=1.5cm of f1-1] (stack1) { n $\mid$ 3 \nodepart{two} acc $\mid$ 1 }
            edge [dashed, thick, blue!70!black] (f1-1);
            \node[box, below=of f1-1] (mul-1) {\mintinline{scheme}{ (f2 (- n 1) (* n acc)) }}
            edge [dashed, thick, blue!70!black] (stack1)
            edge [<-, thick, blue!70!black] (f1-1);
            \node[box, below=1.2cm of mul-1, xshift=-2cm] (sub-1) {\mintinline{scheme}{ (- n 1) }}
            edge [dashed, thick, blue!70!black, bend right=22] (stack1)
            edge [<-, thick, blue!70!black] (mul-1);
            \node[box, below=1.2cm of mul-1, right=of sub-1] (mul-2) {\mintinline{scheme}{ (* n acc) }}
            edge [dashed, thick, blue!70!black] (stack1)
            edge [<-, thick, blue!70!black] (mul-1);

        \end{tikzpicture}

    \end{columns}
\end{frame}

\begin{frame}[fragile]{1.2 Хвостовая рекурсия: Шаг 8. Аргументы хвостового вызова полностью вычислены}
    \begin{columns}[c]

        % ======== левая колонка: код ========
        \column{0.4\textwidth}
        \centering
        \usebox{\codeFactorialTailRec}
        \vspace{0.4cm}

        Оба аргумента рекурсивного вызова уже сведены к своим значениям

        \vfill

        % ======= правая колонка: схема ======
        \column{0.6\textwidth}
        \centering
        \begin{tikzpicture}[
            node distance=0.6cm,
            box/.style={
                draw=blue!70!black, fill=blue!5,
                rounded corners, thick, align=center,
                minimum width=3.8cm, minimum height=0.9cm,
                font=\footnotesize
            }
        ]
            \node[box] (repl) {\alert{REPL}};
            \node[box, below=of repl] (f1-1) {\mintinline{scheme}{(f2 3 1)}}
            edge [<-, thick, blue!70!black] (repl);
            \node[env-node=2, right=1.5cm of f1-1] (stack1) { n $\mid$ 3 \nodepart{two} acc $\mid$ 1 }
            edge [dashed, thick, blue!70!black] (f1-1);
            \node[box, below=of f1-1] (mul-1) {\mintinline{scheme}{ (f2 (- n 1) (* n acc)) }}
            edge [dashed, thick, blue!70!black] (stack1)
            edge [<-, thick, blue!70!black] (f1-1);
            \node[box, below=1.2cm of mul-1, xshift=-2cm] (sub-1) {\mintinline{scheme}{ (2) }}
            edge [dashed, thick, blue!70!black, bend right=22] (stack1)
            edge [<-, thick, blue!70!black] (mul-1);
            \node[box, below=1.2cm of mul-1, right=of sub-1] (mul-2) {\mintinline{scheme}{ (3) }}
            edge [dashed, thick, blue!70!black] (stack1)
            edge [<-, thick, blue!70!black] (mul-1);

        \end{tikzpicture}

    \end{columns}
\end{frame}


\begin{frame}[fragile]{1.2 Хвостовая рекурсия: Шаг 9. Момент хвостовой оптимизации (TCO)}
    \begin{columns}[c]

        % ======== левая колонка: код ========
        \column{0.4\textwidth}
        \centering
        \usebox{\codeFactorialTailRec}
        \vspace{0.4cm}

        Интерпретатор полностью подготовил хвостовой рекурсивный вызов \mintinline{scheme}{(f2 2 3)}.

        \vfill

        % ======= правая колонка: схема ======
        \column{0.6\textwidth}
        \centering
        \begin{tikzpicture}[
            node distance=0.6cm,
            box/.style={
                draw=blue!70!black, fill=blue!5,
                rounded corners, thick, align=center,
                minimum width=3.8cm, minimum height=0.9cm,
                font=\footnotesize
            }
        ]
            \node[box] (repl) {\alert{REPL}};
            \node[box, below=of repl] (f1-1) {\mintinline{scheme}{(f2 3 1)}}
            edge [<-, thick, blue!70!black] (repl);
            \node[env-node=2, right=1.5cm of f1-1] (stack1) { n $\mid$ 3 \nodepart{two} acc $\mid$ 1 }
            edge [dashed, thick, blue!70!black] (f1-1);
            \node[box, below=of f1-1] (mul-1) {\mintinline{scheme}{ (f2 2 3) }}
            edge [dashed, thick, blue!70!black] (stack1)
            edge [<-, thick, blue!70!black] (f1-1);

        \end{tikzpicture}

    \end{columns}
\end{frame}

\begin{frame}[fragile]{1.2 Хвостовая рекурсия: Шаг 10. Хвостовая оптимизация выполнена — кадр перезаписан}
    \begin{columns}[c]

        % ======== левая колонка: код ========
        \column{0.4\textwidth}
        \centering
        \usebox{\codeFactorialTailRec}
        \vspace{0.4cm}

        Сейчас произойдёт ключевое событие: вместо того чтобы положить на стек новый кадр процедуры \mintinline{scheme}{f2}, оптимизатор хвостовой рекурсии просто перезапишет текущий кадр значениями \mintinline{scheme}{n = 2} и \mintinline{scheme}{acc = 3} и передаст управление в начало тела процедуры \mintinline{scheme}{f2}.

        \vfill

        % ======= правая колонка: схема ======
        \column{0.6\textwidth}
        \centering
        \begin{tikzpicture}[
            node distance=0.6cm,
            box/.style={
                draw=blue!70!black, fill=blue!5,
                rounded corners, thick, align=center,
                minimum width=3.8cm, minimum height=0.9cm,
                font=\footnotesize
            }
        ]
            \node[box] (repl) {\alert{REPL}};
            \node[box, below=of repl] (f1-1) {\mintinline{scheme}{(f2 3 1)}}
            edge [<-, thick, blue!70!black] (repl);
            \node[env-node=2, right=1.5cm of f1-1] (stack1) { n $\mid$ 3 \nodepart{two} acc $\mid$ 1 };
            \node[box, below=of f1-1] (mul-1) {\mintinline{scheme}{ (f2 2 3) }}
            edge [dashed, thick, blue!70!black] (stack1)
            edge [<-, thick, blue!70!black] (f1-1);

        \end{tikzpicture}

    \end{columns}
\end{frame}

\begin{frame}[fragile]{1.2 Хвостовая рекурсия: Шаг 11. Переход к следующей итерации — стек остаётся плоским}
    \begin{columns}[c]

        % ======== левая колонка: код ========
        \column{0.4\textwidth}
        \centering
        \usebox{\codeFactorialTailRec}
        \vspace{0.4cm}

        Хвостовая оптимизация успешно выполнена: текущий кадр стека перезаписан новыми значениями \mintinline{scheme}{n = 2} и \mintinline{scheme}{acc = 3} (\mintinline{scheme}{acc} уже умножен: \mintinline{scheme}{1} $\times$ \mintinline{scheme}{3 = 3}).

        \vfill

        % ======= правая колонка: схема ======
        \column{0.6\textwidth}
        \centering
        \begin{tikzpicture}[
            node distance=0.6cm,
            box/.style={
                draw=blue!70!black, fill=blue!5,
                rounded corners, thick, align=center,
                minimum width=3.8cm, minimum height=0.9cm,
                font=\footnotesize
            }
        ]
            \node[box] (repl) {\alert{REPL}};
            \node[box, below=of repl] (f1-1) {\mintinline{scheme}{(f2 3 1)}}
            edge [<-, thick, blue!70!black] (repl);
            \node[env-node=2, right=1.5cm of f1-1] (stack1) { n $\mid$ 2 \nodepart{two} acc $\mid$ 2 };
            \node[box, below=of f1-1] (mul-1) {\mintinline{scheme}{ (f2 2 3) }}
            edge [dashed, thick, blue!70!black] (stack1)
            edge [<-, thick, blue!70!black] (f1-1);

        \end{tikzpicture}

    \end{columns}
\end{frame}


\begin{frame}[fragile]{1.2 Хвостовая рекурсия: Шаг 12. Ещё одна хвостовая оптимизация}
    \begin{columns}[c]

        % ======== левая колонка: код ========
        \column{0.4\textwidth}
        \centering
        \usebox{\codeFactorialTailRec}
        \vspace{0.4cm}

        После очередной итерации внутри того же единственного кадра стека значения снова обновились

        \vfill

        % ======= правая колонка: схема ======
        \column{0.6\textwidth}
        \centering
        \begin{tikzpicture}[
            node distance=0.6cm,
            box/.style={
                draw=blue!70!black, fill=blue!5,
                rounded corners, thick, align=center,
                minimum width=3.8cm, minimum height=0.9cm,
                font=\footnotesize
            }
        ]
            \node[box] (repl) {\alert{REPL}};
            \node[box, below=of repl] (f1-1) {\mintinline{scheme}{(f2 3 1)}}
            edge [<-, thick, blue!70!black] (repl);
            \node[env-node=2, right=1.5cm of f1-1] (stack1) { n $\mid$ 2 \nodepart{two} acc $\mid$ 2 };
            \node[box, below=of f1-1] (mul-1) {\mintinline{scheme}{ (f2 2 3) }}
            edge [dashed, thick, blue!70!black] (stack1)
            edge [<-, thick, blue!70!black] (f1-1);
            \node[box, below=of mul-1] (mul-2) {\mintinline{scheme}{ (f2 1 6) }}
            edge [dashed, thick, blue!70!black, bend right=30] (stack1)
            edge [<-, thick, blue!70!black] (mul-1);
        \end{tikzpicture}


    \end{columns}
\end{frame}


\begin{frame}[fragile]{1.2 Хвостовая рекурсия: Шаг 13. Последний рекурсивный шаг}
    \begin{columns}[c]

        % ======== левая колонка: код ========
        \column{0.4\textwidth}
        \centering
        \usebox{\codeFactorialTailRec}
        \vspace{0.4cm}

        Очередная хвостовая оптимизация выполнена

        \vfill

        % ======= правая колонка: схема ======
        \column{0.6\textwidth}
        \centering
        \begin{tikzpicture}[
            node distance=0.6cm,
            box/.style={
                draw=blue!70!black, fill=blue!5,
                rounded corners, thick, align=center,
                minimum width=3.8cm, minimum height=0.9cm,
                font=\footnotesize
            }
        ]
            \node[box] (repl) {\alert{REPL}};
            \node[box, below=of repl] (f1-1) {\mintinline{scheme}{(f2 3 1)}}
            edge [<-, thick, blue!70!black] (repl);
            \node[env-node=2, right=1.5cm of f1-1] (stack1) { n $\mid$ 1 \nodepart{two} acc $\mid$ 6 };
            \node[box, below=of f1-1] (mul-1) {\mintinline{scheme}{ (f2 2 3) }}
            edge [<-, thick, blue!70!black] (f1-1);
            \node[box, below=of mul-1] (mul-2) {\mintinline{scheme}{ (f2 1 6) }}
            edge [dashed, thick, blue!70!black, bend right=30] (stack1)
            edge [<-, thick, blue!70!black] (mul-1);
        \end{tikzpicture}

    \end{columns}
\end{frame}


\begin{frame}[fragile]{1.2 Хвостовая рекурсия: Шаг 14. Финальная хвостовая оптимизация}
    \begin{columns}[c]

        % ======== левая колонка: код ========
        \column{0.4\textwidth}
        \centering
        \usebox{\codeFactorialTailRec}
        \vspace{0.4cm}

        Ещё одна (уже четвёртая) хвостовая оптимизация выполнена

        \vfill

        % ======= правая колонка: схема ======
        \column{0.6\textwidth}
        \centering
        \begin{tikzpicture}[
            node distance=0.5cm,
            box/.style={
                draw=blue!70!black, fill=blue!5,
                rounded corners, thick, align=center,
                minimum width=3.8cm, minimum height=0.9cm,
                font=\footnotesize
            }
        ]
            \node[box] (repl) {\alert{REPL}};
            \node[box, below=of repl] (f1-1) {\mintinline{scheme}{(f2 3 1)}}
            edge [<-, thick, blue!70!black] (repl);
            \node[env-node=2, right=1.5cm of f1-1] (stack1) { n $\mid$ 1 \nodepart{two} acc $\mid$ 6 };
            \node[box, below=of f1-1] (mul-1) {\mintinline{scheme}{ (f2 2 3) }}
            edge [<-, thick, blue!70!black] (f1-1);
            \node[box, below=of mul-1] (mul-2) {\mintinline{scheme}{ (f2 1 6) }}
            edge [dashed, thick, blue!70!black, bend right=30] (stack1)
            edge [<-, thick, blue!70!black] (mul-1);
            \node[box, below=of mul-2] (mul-3) {\mintinline{scheme}{ (f2 0 6) }}
            edge [dashed, thick, blue!70!black, bend right=30] (stack1)
            edge [<-, thick, blue!70!black] (mul-2);
        \end{tikzpicture}

    \end{columns}
\end{frame}


\begin{frame}[fragile]{1.2 Хвостовая рекурсия: Шаг 15. Условие завершено — рекурсия заканчивается}
    \begin{columns}[c]

        % ======== левая колонка: код ========
        \column{0.4\textwidth}
        \centering
        \usebox{\codeFactorialTailRec}
        \vspace{0.4cm}

        После последней хвостовой оптимизации в единственном кадре стека теперь \mintinline{scheme}{n = 0}, \mintinline{scheme}{acc = 6}.

        \vfill

        % ======= правая колонка: схема ======
        \column{0.6\textwidth}
        \centering
        \begin{tikzpicture}[
            node distance=0.5cm,
            box/.style={
                draw=blue!70!black, fill=blue!5,
                rounded corners, thick, align=center,
                minimum width=3.8cm, minimum height=0.9cm,
                font=\footnotesize
            }
        ]
            \node[box] (repl) {\alert{REPL}};
            \node[box, below=of repl] (f1-1) {\mintinline{scheme}{(f2 3 1)}}
            edge [<-, thick, blue!70!black] (repl);
            \node[env-node=2, right=1.5cm of f1-1] (stack1) { n $\mid$ 0 \nodepart{two} acc $\mid$ 6 };
            \node[box, below=of f1-1] (mul-1) {\mintinline{scheme}{ (f2 2 3) }}
            edge [<-, thick, blue!70!black] (f1-1);
            \node[box, below=of mul-1] (mul-2) {\mintinline{scheme}{ (f2 1 6) }}
            edge [<-, thick, blue!70!black] (mul-1);
            \node[box, below=of mul-2] (mul-3) {\mintinline{scheme}{ (f2 0 6) }}
            edge [dashed, thick, blue!70!black, bend right=30] (stack1)
            edge [<-, thick, blue!70!black] (mul-2);
        \end{tikzpicture}

    \end{columns}
\end{frame}


\begin{frame}[fragile]{1.2 Хвостовая рекурсия: Шаг 16. Завершение вычисления — возврат результата}
    \begin{columns}[c]

        % ======== левая колонка: код ========
        \column{0.4\textwidth}
        \centering
        \usebox{\codeFactorialTailRec}
        \vspace{0.4cm}

        В единственном кадре стека теперь \mintinline{scheme}{n = 0}.

        \vfill

        % ======= правая колонка: схема ======
        \column{0.6\textwidth}
        \centering
        \begin{tikzpicture}[
            node distance=0.5cm,
            box/.style={
                draw=blue!70!black, fill=blue!5,
                rounded corners, thick, align=center,
                minimum width=3.8cm, minimum height=0.9cm,
                font=\footnotesize
            }
        ]
            \node[box] (repl) {\alert{REPL}};
            \node[box, below=of repl] (f1-1) {\mintinline{scheme}{(f2 3 1)}}
            edge [<-, thick, blue!70!black] (repl);
            \node[env-node=2, right=1.5cm of f1-1] (stack1) { n $\mid$ 0 \nodepart{two} acc $\mid$ 6 };
            \node[box, below=of f1-1] (mul-1) {\mintinline{scheme}{ (f2 2 3) }}
            edge [<-, thick, blue!70!black] (f1-1);
            \node[box, below=of mul-1] (mul-2) {\mintinline{scheme}{ (f2 1 6) }}
            edge [<-, thick, blue!70!black] (mul-1);
            \node[box, below=of mul-2] (mul-3) {\mintinline{scheme}{ (f2 0 6) }}
            edge [<-, thick, blue!70!black] (mul-2)
            edge [dashed, thick, blue!70!black, bend right=30] (stack1);
        \end{tikzpicture}

    \end{columns}
\end{frame}


\begin{frame}[fragile]{1.2 Хвостовая рекурсия: Шаг 17. Рекурсия завершена — возврат аккумулятора}
    \begin{columns}[c]

        % ======== левая колонка: код ========
        \column{0.4\textwidth}
        \centering
        \usebox{\codeFactorialTailRec}
        \vspace{0.4cm}

        Условие \mintinline{scheme}{(= n 0)} истинно, поэтому тело процедуры свелось к простому возврату значения переменной \mintinline{scheme}{acc}.

        \vfill

        % ======= правая колонка: схема ======
        \column{0.6\textwidth}
        \centering
        \begin{tikzpicture}[
            node distance=0.5cm,
            box/.style={
                draw=blue!70!black, fill=blue!5,
                rounded corners, thick, align=center,
                minimum width=3.8cm, minimum height=0.9cm,
                font=\footnotesize
            }
        ]
            \node[box] (repl) {\alert{REPL}};
            \node[box, below=of repl] (f1-1) {\mintinline{scheme}{(f2 3 1)}}
            edge [<-, thick, blue!70!black] (repl);
            \node[env-node=2, right=1.5cm of f1-1] (stack1) { n $\mid$ 0 \nodepart{two} acc $\mid$ 6 };
            \node[box, below=of f1-1] (mul-1) {\mintinline{scheme}{ (f2 2 3) }}
            edge [<-, thick, blue!70!black] (f1-1);
            \node[box, below=of mul-1] (mul-2) {\mintinline{scheme}{ (f2 1 6) }}
            edge [<-, thick, blue!70!black] (mul-1);
            \node[box, below=of mul-2] (mul-3) {\mintinline{scheme}{ (acc) }}
            edge [<-, thick, blue!70!black] (mul-2)
            edge [dashed, thick, blue!70!black, bend right=30] (stack1);
        \end{tikzpicture}

    \end{columns}
\end{frame}


\begin{frame}[fragile]{1.2 Хвостовая рекурсия: Шаг 18. Результат возвращён — рекурсия полностью завершена}
    \begin{columns}[c]

        % ======== левая колонка: код ========
        \column{0.4\textwidth}
        \centering
        \usebox{\codeFactorialTailRec}
        \vspace{0.4cm}

        Переменная acc в последнем кадре содержит значение \mintinline{scheme}{6}

        \vfill

        % ======= правая колонка: схема ======
        \column{0.6\textwidth}
        \centering
        \begin{tikzpicture}[
            node distance=0.5cm,
            box/.style={
                draw=blue!70!black, fill=blue!5,
                rounded corners, thick, align=center,
                minimum width=3.8cm, minimum height=0.9cm,
                font=\footnotesize
            }
        ]
            \node[box] (repl) {\alert{REPL}};
            \node[box, below=of repl] (f1-1) {\mintinline{scheme}{(f2 3 1)}}
            edge [<-, thick, blue!70!black] (repl);
            \node[env-node=2, right=1.5cm of f1-1] (stack1) { n $\mid$ 0 \nodepart{two} acc $\mid$ 6 };
            \node[box, below=of f1-1] (mul-1) {\mintinline{scheme}{ (f2 2 3) }}
            edge [<-, thick, blue!70!black] (f1-1);
            \node[box, below=of mul-1] (mul-2) {\mintinline{scheme}{ (f2 1 6) }}
            edge [<-, thick, blue!70!black] (mul-1);
            \node[box, below=of mul-2] (mul-3) {\mintinline{scheme}{ 6 }}
            edge [<-, thick, blue!70!black] (mul-2)
            edge [->, thick, blue!70!black, bend left=30] (mul-2)
            edge [dashed, thick, blue!70!black, bend right=30] (stack1);
        \end{tikzpicture}

    \end{columns}
\end{frame}


\begin{frame}[fragile]{1.2 Хвостовая рекурсия: Шаг 19. Возврат значения вверх по цепочке (хвостовой позиции)}
    \begin{columns}[c]

        % ======== левая колонка: код ========
        \column{0.4\textwidth}
        \centering
        \usebox{\codeFactorialTailRec}
        \vspace{0.4cm}

        Значение 6, полученное в последнем кадре, возвращается сразу в предыдущий «виртуальный» уровень.

        \vfill

        % ======= правая колонка: схема ======
        \column{0.6\textwidth}
        \centering
        \begin{tikzpicture}[
            node distance=0.5cm,
            box/.style={
                draw=blue!70!black, fill=blue!5,
                rounded corners, thick, align=center,
                minimum width=3.8cm, minimum height=0.9cm,
                font=\footnotesize
            }
        ]
            \node[box] (repl) {\alert{REPL}};
            \node[box, below=of repl] (f1-1) {\mintinline{scheme}{(f2 3 1)}}
            edge [<-, thick, blue!70!black] (repl);
            \node[env-node=2, right=1.5cm of f1-1] (stack1) { n $\mid$ 0 \nodepart{two} acc $\mid$ 6 };
            \node[box, below=of f1-1] (mul-1) {\mintinline{scheme}{ (f2 2 3) }}
            edge [<-, thick, blue!70!black] (f1-1);
            \node[box, below=of mul-1] (mul-2) {\mintinline{scheme}{ 6 }}
            edge [<-, thick, blue!70!black] (mul-1)
            edge [->, thick, blue!70!black, bend left=30] (mul-1);
        \end{tikzpicture}

    \end{columns}
\end{frame}



\begin{frame}[fragile]{1.2 Хвостовая рекурсия: Шаг 20. Финальный результат в REPL — рекурсия завершена без роста стека}
    \begin{columns}[c]

        % ======== левая колонка: код ========
        \column{0.4\textwidth}
        \centering
        \usebox{\codeFactorialTailRec}
        \vspace{0.4cm}

        Значение 6 поднялось по всем «виртуальным» уровням.

        \vfill

        % ======= правая колонка: схема ======
        \column{0.6\textwidth}
        \centering
        \begin{tikzpicture}[
            node distance=0.5cm,
            box/.style={
                draw=blue!70!black, fill=blue!5,
                rounded corners, thick, align=center,
                minimum width=3.8cm, minimum height=0.9cm,
                font=\footnotesize
            }
        ]
            \node[box] (repl) {\alert{REPL}};
            \node[box, below=of repl] (f1-1) {\mintinline{scheme}{(f2 3 1)}}
            edge [<-, thick, blue!70!black] (repl);
            \node[env-node=2, right=1.5cm of f1-1] (stack1) { n $\mid$ 0 \nodepart{two} acc $\mid$ 6 };
            \node[box, below=of f1-1] (mul-1) {\mintinline{scheme}{ 6 }}
            edge [<-, thick, blue!70!black] (f1-1)
            edge [->, thick, blue!70!black, bend left=30] (f1-1);
        \end{tikzpicture}

    \end{columns}
\end{frame}

\begin{frame}[fragile]{1.2 Хвостовая рекурсия: Шаг 21. Всё завершено — результат в REPL}
    \begin{columns}[c]

        % ======== левая колонка: код ========
        \column{0.4\textwidth}
        \centering
        \usebox{\codeFactorialTailRec}
        \vspace{0.4cm}

        Значение 6 окончательно возвращено в \alert{REPL}.

        \vfill

        % ======= правая колонка: схема ======
        \column{0.6\textwidth}
        \centering
        \begin{tikzpicture}[
            node distance=0.5cm,
            box/.style={
                draw=blue!70!black, fill=blue!5,
                rounded corners, thick, align=center,
                minimum width=3.8cm, minimum height=0.9cm,
                font=\footnotesize
            }
        ]
            \node[box] (repl) {\alert{REPL}};
            \node[box, below=of repl] (f1-1) {\mintinline{scheme}{ 6 }}
            edge [<-, thick, blue!70!black] (repl);
        \end{tikzpicture}

    \end{columns}
\end{frame}

% ============================
% ==        Глава 2.        ==
% ==                        ==
% ==    Замыкание лямбды    ==
% ============================
\begin{frame}{Глава 2: Замыкание лямбда-выражения}
    \centering

    На примере функций \mintinline{scheme}{f} и \mintinline{scheme}{g}
    разберем, как лямбда-выражение захватывает внешнюю переменную
    \mintinline{scheme}{x} и сохраняет доступ к ней в замыкании — механизм,
    позволяющий функциям «помнить» окружение.

    \usebox{\codeLambdaExample}
\end{frame}


\begin{frame}[fragile]{2.1 Замыкание лямбды: Шаг 1. Начальный вызов процедуры f}
    \begin{columns}[c]

        % ======== левая колонка: код ========
        \column{0.4\textwidth}
        \centering
        \usebox{\codeLambdaExample}
        \vspace{0.4cm}

        Среда \alert{REPL} вызывает пользовательскую процедуру \mintinline{scheme}{f} с аргументом \mintinline{scheme}{4}.

        \vfill

        % ======= правая колонка: схема ======
        \column{0.6\textwidth}
        \centering
        \begin{tikzpicture}[
            node distance=0.5cm,
            box/.style={
                draw=blue!70!black, fill=blue!5,
                rounded corners, thick, align=center,
                minimum width=3.8cm, minimum height=0.9cm,
                font=\footnotesize
            }
        ]
            \node[box] (repl) {\alert{REPL}};
            \node[box, below=of repl] (f1-1) {\mintinline{scheme}{ (f 4) }}
            edge [<-, thick, blue!70!black] (repl);
        \end{tikzpicture}

    \end{columns}
\end{frame}


\begin{frame}[fragile]{2.1 Замыкание лямбды: Шаг 2. Начало вычисления тела процедуры f}
    \begin{columns}[c]

        % ======== левая колонка: код ========
        \column{0.4\textwidth}
        \centering
        \usebox{\codeLambdaExample}
        \vspace{0.4cm}

        Интерпретатор приступил к вычислению тела процедуры \mintinline{scheme}{f}: формы \mintinline{scheme}{(+ (g ...) 3)}.

        \vfill

        % ======= правая колонка: схема ======
        \column{0.6\textwidth}
        \centering
        \begin{tikzpicture}[
            node distance=0.8cm,
            box/.style={
                draw=blue!70!black, fill=blue!5,
                rounded corners, thick, align=center,
                minimum width=3.8cm, minimum height=0.9cm,
                font=\footnotesize
            }
        ]
            \node[box] (repl) {\alert{REPL}};
            \node[box, below=of repl] (f1-1) {\mintinline{scheme}{ (f 4) }}
            edge [<-, thick, blue!70!black] (repl);
            \node[env-node=1, below=of repl, right=of f1-1] (stack1) { x $\mid$ 4 }
            edge [dashed, thick, blue!70!black] (f1-1);
            \node[box, below=of f1-1] (p1) {\mintinline{scheme}{ (+ (g (lambda (y) (* x y)) 3) }}
            edge [<-, thick, blue!70!black] (f1-1)
            edge [dashed, thick, blue!70!black] (stack1);
        \end{tikzpicture}

    \end{columns}
\end{frame}

\begin{frame}[fragile]{2.1 Замыкание лямбды: Шаг 3. Вычисление фактического аргумента для вызова g}
    \begin{columns}[c]

        % ======== левая колонка: код ========
        \column{0.4\textwidth}
        \centering
        \usebox{\codeLambdaExample}
        \vspace{0.4cm}

        Интерпретатор завершил создание замыкания \mintinline{scheme}{(lambda (y) (* x y))} в окружающей среде, где \mintinline{scheme}{x} = 4.

        \vfill

        % ======= правая колонка: схема ======
        \column{0.6\textwidth}
        \centering
        \begin{tikzpicture}[
            node distance=0.8cm,
            box/.style={
                draw=blue!70!black, fill=blue!5,
                rounded corners, thick, align=center,
                minimum width=3.8cm, minimum height=0.9cm,
                font=\footnotesize
            }
        ]
            \node[box] (repl) {\alert{REPL}};
            \node[box, below=of repl] (f1-1) {\mintinline{scheme}{ (f 4) }}
            edge [<-, thick, blue!70!black] (repl);
            \node[env-node=1, below=of repl, right=of f1-1] (stack1) { x $\mid$ 4 }
            edge [dashed, thick, blue!70!black] (f1-1);
            \node[box, below=of f1-1] (p1) {\mintinline{scheme}{ (+ (g (lambda (y) (* x y)) 3) }}
            edge [<-, thick, blue!70!black] (f1-1)
            edge [dashed, thick, blue!70!black] (stack1);
            \node[box, below=of p1] (g1) {\mintinline{scheme}{  (g (lambda (y) (* x y)) }}
            edge [<-, thick, blue!70!black] (p1);
        \end{tikzpicture}

    \end{columns}
\end{frame}


\begin{frame}[fragile]{2.1 Замыкание лямбды: Шаг 4. Вызов процедуры g — рост стека}
    \begin{columns}[c]

        % ======== левая колонка: код ========
        \column{0.4\textwidth}
        \centering
        \usebox{\codeLambdaExample}
        \vspace{0.4cm}

        Интерпретатор выполнил применение процедуры \mintinline{scheme}{g} к замыканию \mintinline{scheme}{(lambda (y) (* x y))}.

        \vfill

        % ======= правая колонка: схема ======
        \column{0.6\textwidth}
        \centering
        \begin{tikzpicture}[
            node distance=0.7cm,
            box/.style={
                draw=blue!70!black, fill=blue!5,
                rounded corners, thick, align=center,
                minimum width=3.8cm, minimum height=0.9cm,
                font=\footnotesize
            }
        ]
            \node[box] (repl) {\alert{REPL}};
            \node[box, below=of repl] (f1-1) {\mintinline{scheme}{ (f 4) }}
            edge [<-, thick, blue!70!black] (repl);
            \node[env-node=1, below=of repl, right=1cm of f1-1] (stack1) { x $\mid$ 4 }
            edge [dashed, thick, blue!70!black] (f1-1);
            \node[box, below=of f1-1] (p1) {\mintinline{scheme}{ (+ (g (lambda (y) (* x y)) 3) }}
            edge [<-, thick, blue!70!black] (f1-1)
            edge [dashed, thick, blue!70!black] (stack1);
            \node[box, below=1.3cm of p1, xshift=-0.5cm] (g1) {\mintinline{scheme}{ (g (lambda (y) (* x y)) }}
            edge [<-, thick, blue!70!black, bend left=25] (p1);
            \node[env-node=1, right=of g1, above=0.7cm, xshift=0.6cm] (stack2) { h $\mid$ \mintinline{scheme}{ (lambda (y) (* x y)) } }
            edge [dashed, thick, blue!70!black, bend left=20] (g1)
            edge [dashed, thick, blue!70!black, bend right=30] (stack1);
        \end{tikzpicture}

    \end{columns}
\end{frame}


\begin{frame}[fragile]{2.1 Замыкание лямбды: Шаг 5. Начало вычисления тела процедуры g}
    \begin{columns}[c]

        % ======== левая колонка: код ========
        \column{0.4\textwidth}
        \centering
        \usebox{\codeLambdaExample}
        \vspace{0.4cm}

        Интерпретатор приступил к вычислению тела процедуры \mintinline{scheme}{g}

        \vfill

        % ======= правая колонка: схема ======
        \column{0.6\textwidth}
        \centering
        \begin{tikzpicture}[
            node distance=0.5cm,
            box/.style={
                draw=blue!70!black, fill=blue!5,
                rounded corners, thick, align=center,
                minimum width=3.8cm, minimum height=0.6cm,
                font=\footnotesize
            }
        ]
            \node[box] (repl) {\alert{REPL}};
            \node[box, below=of repl] (f1-1) {\mintinline{scheme}{ (f 4) }}
            edge [<-, thick, blue!70!black] (repl);
            \node[env-node=1, below=of repl, right=1cm of f1-1] (stack1) { x $\mid$ 4 }
            edge [dashed, thick, blue!70!black] (f1-1);
            \node[box, below=of f1-1] (p1) {\mintinline{scheme}{ (+ (g (lambda (y) (* x y)) 3) }}
            edge [<-, thick, blue!70!black] (f1-1)
            edge [dashed, thick, blue!70!black] (stack1);
            \node[box, below=1.4cm of p1, xshift=-0.7cm] (g1) {\mintinline{scheme}{ (g (lambda (y) (* x y)) }}
            edge [<-, thick, blue!70!black, bend left=35] (p1);
            \node[env-node=1, right=of g1, above=0.65cm, xshift=0.6cm] (stack2) { h $\mid$ \mintinline{scheme}{ (lambda (y) (* x y)) } }
            edge [dashed, thick, blue!70!black, bend left=20] (g1)
            edge [dashed, thick, blue!70!black, bend right=45] (stack1);
            \node[box, below=of g1] (h1) {\mintinline{scheme}{ (h (h 7)) }}
            edge [dashed, thick, blue!70!black, bend right=30] (stack2)
            edge [<-, thick, blue!70!black] (g1);
        \end{tikzpicture}

    \end{columns}
\end{frame}

\begin{frame}[fragile]{2.1 Замыкание лямбды: Шаг 6. Первый (внешний) вызов замыкания h — рост стека продолжается}
    \begin{columns}[c]

        % ======== левая колонка: код ========
        \column{0.4\textwidth}
        \centering
        \usebox{\codeLambdaExample}
        \vspace{0.4cm}

        Интерпретатор вычислил аргумент внутреннего вызова — число \mintinline{scheme}{7} — и теперь готов применить процедуру \mintinline{scheme}{h} к этому аргументу.

        \vfill

        % ======= правая колонка: схема ======
        \column{0.6\textwidth}
        \centering
        \begin{tikzpicture}[
            node distance=0.4cm,
            box/.style={
                draw=blue!70!black, fill=blue!5,
                rounded corners, thick, align=center,
                minimum width=3.8cm, minimum height=0.55cm,
                font=\footnotesize
            }
        ]
            \node[box] (repl) {\alert{REPL}};
            \node[box, below=of repl] (f1-1) {\mintinline{scheme}{ (f 4) }}
            edge [<-, thick, blue!70!black] (repl);
            \node[env-node=1, below=of repl, right=1cm of f1-1] (stack1) { x $\mid$ 4 }
            edge [dashed, thick, blue!70!black] (f1-1);
            \node[box, below=of f1-1] (p1) {\mintinline{scheme}{ (+ (g (lambda (y) (* x y)) 3) }}
            edge [<-, thick, blue!70!black] (f1-1)
            edge [dashed, thick, blue!70!black] (stack1);
            \node[box, below=1.4cm of p1, xshift=-0.7cm] (g1) {\mintinline{scheme}{ (g (lambda (y) (* x y)) }}
            edge [<-, thick, blue!70!black, bend left=35] (p1);
            \node[env-node=1, right=of g1, above=0.65cm, xshift=0.6cm] (stack2) { h $\mid$ \mintinline{scheme}{ (lambda (y) (* x y)) } }
            edge [dashed, thick, blue!70!black, bend left=20] (g1)
            edge [dashed, thick, blue!70!black, bend right=45] (stack1);
            \node[box, below=of g1] (h1) {\mintinline{scheme}{ (h (h 7)) }}
            edge [dashed, thick, blue!70!black, bend right=30] (stack2)
            edge [<-, thick, blue!70!black] (g1);
            \node[box, below=of h1] (h2) {\mintinline{scheme}{ (h 7) }}
            edge [dashed, thick, blue!70!black, bend right=35] (stack2)
            edge [<-, thick, blue!70!black] (h1);
        \end{tikzpicture}


    \end{columns}
\end{frame}
\begin{frame}[fragile]{2.1 Замыкание лямбды: Шаг 7. Первое применение замыкания}
    \begin{columns}[c]

        % ======== левая колонка: код ========
        \column{0.4\textwidth}
        \centering
        \usebox{\codeLambdaExample}
        \vspace{0.4cm}

        Интерпретатор выполнил первый вызов замыкания \mintinline{scheme}{h} к аргументом \mintinline{scheme}{7}.

        \vfill

        % ======= правая колонка: схема ======
        \column{0.6\textwidth}
        \centering
        \begin{tikzpicture}[
            node distance=0.4cm,
            box/.style={
                draw=blue!70!black, fill=blue!5,
                rounded corners, thick, align=center,
                minimum width=3.8cm, minimum height=0.55cm,
                font=\footnotesize
            }
        ]
            \node[box] (repl) {\alert{REPL}};
            \node[box, below=of repl] (f1-1) {\mintinline{scheme}{ (f 4) }}
            edge [<-, thick, blue!70!black] (repl);
            \node[env-node=1, below=of repl, right=1cm of f1-1] (stack1) { x $\mid$ 4 }
            edge [dashed, thick, blue!70!black] (f1-1);
            \node[box, below=of f1-1] (p1) {\mintinline{scheme}{ (+ (g (lambda (y) (* x y)) 3) }}
            edge [<-, thick, blue!70!black] (f1-1)
            edge [dashed, thick, blue!70!black] (stack1);
            \node[box, below=1.4cm of p1, xshift=-0.7cm] (g1) {\mintinline{scheme}{ (g (lambda (y) (* x y)) }}
            edge [<-, thick, blue!70!black, bend left=35] (p1);
            \node[env-node=1, right=of g1, above=0.65cm, xshift=0.6cm] (stack2) { h $\mid$ \mintinline{scheme}{ (lambda (y) (* x y)) } }
            edge [dashed, thick, blue!70!black, bend left=20] (g1)
            edge [dashed, thick, blue!70!black, bend right=45] (stack1);
            \node[box, below=of g1] (h1) {\mintinline{scheme}{ (h (h 7)) }}
            edge [dashed, thick, blue!70!black, bend right=30] (stack2)
            edge [<-, thick, blue!70!black] (g1);
            \node[box, below=of h1] (h2) {\mintinline{scheme}{ ((lambda (y) (* x y)) 7) }}
            edge [dashed, thick, blue!70!black, bend right=35] (stack2)
            edge [dashed, thick, red!70!black, bend right=40] (stack1)
            edge [<-, thick, blue!70!black] (h1);
            \node[env-node=1, below=of repl, right=1cm of h2] (stack3) { y $\mid$ 7 }
            edge [dashed, thick, blue!70!black] (h2);
        \end{tikzpicture}


    \end{columns}
\end{frame}

\begin{frame}[fragile]{2.1 Замыкание лямбды: Шаг 8. Завершение первого применения замыкания}
    \begin{columns}[c]

        % ======== левая колонка: код ========
        \column{0.4\textwidth}
        \centering
        \usebox{\codeLambdaExample}
        \vspace{0.4cm}

        Тело лямбда-выражения \mintinline{scheme}{(* x y)} полностью вычислено

        \vfill

        % ======= правая колонка: схема ======
        \column{0.6\textwidth}
        \centering
        \begin{tikzpicture}[
            node distance=0.4cm,
            box/.style={
                draw=blue!70!black, fill=blue!5,
                rounded corners, thick, align=center,
                minimum width=3.8cm, minimum height=0.55cm,
                font=\footnotesize
            }
        ]
            \node[box] (repl) {\alert{REPL}};
            \node[box, below=of repl] (f1-1) {\mintinline{scheme}{ (f 4) }}
            edge [<-, thick, blue!70!black] (repl);
            \node[env-node=1, below=of repl, right=1cm of f1-1] (stack1) { x $\mid$ 4 }
            edge [dashed, thick, blue!70!black] (f1-1);
            \node[box, below=of f1-1] (p1) {\mintinline{scheme}{ (+ (g (lambda (y) (* x y)) 3) }}
            edge [<-, thick, blue!70!black] (f1-1)
            edge [dashed, thick, blue!70!black] (stack1);
            \node[box, below=1.4cm of p1, xshift=-0.7cm] (g1) {\mintinline{scheme}{ (g (lambda (y) (* x y)) }}
            edge [<-, thick, blue!70!black, bend left=35] (p1);
            \node[env-node=1, right=of g1, above=0.65cm, xshift=0.6cm] (stack2) { h $\mid$ \mintinline{scheme}{ (lambda (y) (* x y)) } }
            edge [dashed, thick, blue!70!black, bend left=20] (g1)
            edge [dashed, thick, blue!70!black, bend right=45] (stack1);
            \node[box, below=of g1] (h1) {\mintinline{scheme}{ (h (h 7)) }}
            edge [dashed, thick, blue!70!black, bend right=30] (stack2)
            edge [<-, thick, blue!70!black] (g1);
            \node[box, below=of h1] (h2) {\mintinline{scheme}{ 28 }}
            edge [dashed, thick, blue!70!black, bend right=35] (stack2)
            edge [dashed, thick, red!70!black, bend right=40] (stack1)
            edge [->, thick, blue!70!black, bend left=30] (h1)
            edge [<-, thick, blue!70!black] (h1);
            \node[env-node=1, below=of repl, right=1cm of h2] (stack3) { y $\mid$ 7 }
            edge [dashed, thick, blue!70!black] (h2);
        \end{tikzpicture}


    \end{columns}
\end{frame}

\begin{frame}[fragile]{2.1 Замыкание лямбды: Шаг 9. Возврат из первого замыкания}
    \begin{columns}[c]

        % ======== левая колонка: код ========
        \column{0.4\textwidth}
        \centering
        \usebox{\codeLambdaExample}
        \vspace{0.4cm}

        Первый вызов замыкания \mintinline{scheme}{(lambda (y) (* x y))} завершён и полностью вытеснен из стека (фрейм с \mintinline{scheme}{y | 7} удалён).

        \vfill

        % ======= правая колонка: схема ======
        \column{0.6\textwidth}
        \centering
        \begin{tikzpicture}[
            node distance=0.5cm,
            box/.style={
                draw=blue!70!black, fill=blue!5,
                rounded corners, thick, align=center,
                minimum width=3.8cm, minimum height=0.6cm,
                font=\footnotesize
            }
        ]
            \node[box] (repl) {\alert{REPL}};
            \node[box, below=of repl] (f1-1) {\mintinline{scheme}{ (f 4) }}
            edge [<-, thick, blue!70!black] (repl);
            \node[env-node=1, below=of repl, right=1cm of f1-1] (stack1) { x $\mid$ 4 }
            edge [dashed, thick, blue!70!black] (f1-1);
            \node[box, below=of f1-1] (p1) {\mintinline{scheme}{ (+ (g (lambda (y) (* x y)) 3) }}
            edge [<-, thick, blue!70!black] (f1-1)
            edge [dashed, thick, blue!70!black] (stack1);
            \node[box, below=1.4cm of p1, xshift=-0.7cm] (g1) {\mintinline{scheme}{ (g (lambda (y) (* x y)) }}
            edge [<-, thick, blue!70!black, bend left=35] (p1);
            \node[env-node=1, right=of g1, above=0.65cm, xshift=0.6cm] (stack2) { h $\mid$ \mintinline{scheme}{ (lambda (y) (* x y)) } }
            edge [dashed, thick, blue!70!black, bend left=20] (g1)
            edge [dashed, thick, blue!70!black, bend right=45] (stack1);
            \node[box, below=of g1] (h1) {\mintinline{scheme}{ (h 28) }}
            edge [dashed, thick, blue!70!black, bend right=30] (stack2)
            edge [<-, thick, blue!70!black] (g1);
        \end{tikzpicture}


    \end{columns}
\end{frame}

\begin{frame}[fragile]{2.1 Замыкание лямбды: Шаг 10. Второй вызов того же замыкания}
    \begin{columns}[c]

        % ======== левая колонка: код ========
        \column{0.4\textwidth}
        \centering
        \usebox{\codeLambdaExample}
        \vspace{0.4cm}

        Интерпретатор выполнил второй вызов замыкания \mintinline{scheme}{h} уже с аргументом \mintinline{scheme}{28}.

        \vfill

        % ======= правая колонка: схема ======
        \column{0.6\textwidth}
        \centering
        \begin{tikzpicture}[
            node distance=0.5cm,
            box/.style={
                draw=blue!70!black, fill=blue!5,
                rounded corners, thick, align=center,
                minimum width=3.8cm, minimum height=0.6cm,
                font=\footnotesize
            }
        ]
            \node[box] (repl) {\alert{REPL}};
            \node[box, below=of repl] (f1-1) {\mintinline{scheme}{ (f 4) }}
            edge [<-, thick, blue!70!black] (repl);
            \node[env-node=1, below=of repl, right=1cm of f1-1] (stack1) { x $\mid$ 4 }
            edge [dashed, thick, blue!70!black] (f1-1);
            \node[box, below=of f1-1] (p1) {\mintinline{scheme}{ (+ (g (lambda (y) (* x y)) 3) }}
            edge [<-, thick, blue!70!black] (f1-1)
            edge [dashed, thick, blue!70!black] (stack1);
            \node[box, below=1.4cm of p1, xshift=-0.7cm] (g1) {\mintinline{scheme}{ (g (lambda (y) (* x y)) }}
            edge [<-, thick, blue!70!black, bend left=35] (p1);
            \node[env-node=1, right=of g1, above=0.65cm, xshift=0.6cm] (stack2) { h $\mid$ \mintinline{scheme}{ (lambda (y) (* x y)) } }
            edge [dashed, thick, blue!70!black, bend left=20] (g1)
            edge [dashed, thick, blue!70!black, bend right=45] (stack1);
            \node[box, below=of g1] (h1) {\mintinline{scheme}{ ((lambda (y) (* x y)) 28) }}
            edge [dashed, thick, blue!70!black, bend right=30] (stack2)
            edge [dashed, thick, red!70!black, bend right=40] (stack1)
            edge [<-, thick, blue!70!black] (g1);
            \node[env-node=1, below=of g1, right=1cm of h1] (stack3) { y $\mid$ 28 }
            edge [dashed, thick, blue!70!black] (h1);
        \end{tikzpicture}


    \end{columns}
\end{frame}

\begin{frame}[fragile]{2.1 Замыкание лямбды: Шаг 11. Завершение второго применения замыкания}
    \begin{columns}[c]

        % ======== левая колонка: код ========
        \column{0.4\textwidth}
        \centering
        \usebox{\codeLambdaExample}
        \vspace{0.4cm}

        Тело лямбда-выражения \mintinline{scheme}{(* x y)} вычислено во второй раз:
        при \mintinline{scheme}{x} = 4 и \mintinline{scheme}{y} = 28 получено значение \mintinline{scheme}{112}.

        \vfill

        % ======= правая колонка: схема ======
        \column{0.6\textwidth}
        \centering
        \begin{tikzpicture}[
            node distance=0.5cm,
            box/.style={
                draw=blue!70!black, fill=blue!5,
                rounded corners, thick, align=center,
                minimum width=3.8cm, minimum height=0.6cm,
                font=\footnotesize
            }
        ]
            \node[box] (repl) {\alert{REPL}};
            \node[box, below=of repl] (f1-1) {\mintinline{scheme}{ (f 4) }}
            edge [<-, thick, blue!70!black] (repl);
            \node[env-node=1, below=of repl, right=1cm of f1-1] (stack1) { x $\mid$ 4 }
            edge [dashed, thick, blue!70!black] (f1-1);
            \node[box, below=of f1-1] (p1) {\mintinline{scheme}{ (+ (g (lambda (y) (* x y)) 3) }}
            edge [<-, thick, blue!70!black] (f1-1)
            edge [dashed, thick, blue!70!black] (stack1);
            \node[box, below=1.4cm of p1, xshift=-0.7cm] (g1) {\mintinline{scheme}{ (g (lambda (y) (* x y)) }}
            edge [<-, thick, blue!70!black, bend left=35] (p1);
            \node[env-node=1, right=of g1, above=0.65cm, xshift=0.6cm] (stack2) { h $\mid$ \mintinline{scheme}{ (lambda (y) (* x y)) } }
            edge [dashed, thick, blue!70!black, bend left=20] (g1)
            edge [dashed, thick, blue!70!black, bend right=45] (stack1);
            \node[box, below=of g1] (h1) {\mintinline{scheme}{ 112 }}
            edge [dashed, thick, blue!70!black, bend right=30] (stack2)
            edge [dashed, thick, red!70!black, bend right=40] (stack1)
            edge [->, thick, blue!70!black, bend left=30] (g1)
            edge [<-, thick, blue!70!black] (g1);
            \node[env-node=1, below=of g1, right=1cm of h1] (stack3) { y $\mid$ 28 }
            edge [dashed, thick, blue!70!black] (h1);
        \end{tikzpicture}


    \end{columns}
\end{frame}

\begin{frame}[fragile]{2.1 Замыкание лямбды: Шаг 12. Завершение вызова g — возврат значения 112 в тело процедуры f}
    \begin{columns}[c]

        % ======== левая колонка: код ========
        \column{0.4\textwidth}
        \centering
        \usebox{\codeLambdaExample}
        \vspace{0.4cm}

        Процедура \mintinline{scheme}{g} полностью отработала и вернула значение \mintinline{scheme}{112}.

        \vfill

        % ======= правая колонка: схема ======
        \column{0.6\textwidth}
        \centering
        \begin{tikzpicture}[
            node distance=0.7cm,
            box/.style={
                draw=blue!70!black, fill=blue!5,
                rounded corners, thick, align=center,
                minimum width=3.8cm, minimum height=0.9cm,
                font=\footnotesize
            }
        ]
            \node[box] (repl) {\alert{REPL}};
            \node[box, below=of repl] (f1-1) {\mintinline{scheme}{ (f 4) }}
            edge [<-, thick, blue!70!black] (repl);
            \node[env-node=1, below=of repl, right=1cm of f1-1] (stack1) { x $\mid$ 4 }
            edge [dashed, thick, blue!70!black] (f1-1);
            \node[box, below=of f1-1] (p1) {\mintinline{scheme}{ (+ (g (lambda (y) (* x y)) 3) }}
            edge [<-, thick, blue!70!black] (f1-1)
            edge [dashed, thick, blue!70!black] (stack1);
            \node[box, below=1.3cm of p1, xshift=-1cm] (g1) {112}
            edge [->, thick, blue!70!black, bend left=30] (p1)
            edge [<-, thick, blue!70!black] (p1);
            \node[env-node=1, right=of g1, above=0.7cm, xshift=1.8cm] (stack2) { h $\mid$ \mintinline{scheme}{ (lambda (y) (* x y)) } }
            edge [dashed, thick, blue!70!black, bend left=20] (g1)
            edge [dashed, thick, blue!70!black, bend right=30] (stack1);
        \end{tikzpicture}


    \end{columns}
\end{frame}


\begin{frame}[fragile]{2.1 Замыкание лямбды: Шаг 13. инальное вычисление в теле процедуры}
    \begin{columns}[c]

        % ======== левая колонка: код ========
        \column{0.4\textwidth}
        \centering
        \usebox{\codeLambdaExample}
        \vspace{0.4cm}

        Вызов \mintinline{scheme}{g} полностью завершён, его результат \mintinline{scheme}{112} подставлен в тело процедуры \mintinline{scheme}{f}.

        \vfill

        % ======= правая колонка: схема ======
        \column{0.6\textwidth}
        \centering
        \begin{tikzpicture}[
            node distance=0.7cm,
            box/.style={
                draw=blue!70!black, fill=blue!5,
                rounded corners, thick, align=center,
                minimum width=3.8cm, minimum height=0.9cm,
                font=\footnotesize
            }
        ]
            \node[box] (repl) {\alert{REPL}};
            \node[box, below=of repl] (f1-1) {\mintinline{scheme}{ (f 4) }}
            edge [<-, thick, blue!70!black] (repl);
            \node[env-node=1, below=of repl, right=of f1-1] (stack1) { x $\mid$ 4 }
            edge [dashed, thick, blue!70!black] (f1-1);
            \node[box, below=of f1-1] (p1) {\mintinline{scheme}{ (+ 112 3) }}
            edge [<-, thick, blue!70!black] (f1-1)
            edge [dashed, thick, blue!70!black] (stack1);
        \end{tikzpicture}


    \end{columns}
\end{frame}

\begin{frame}[fragile]{2.1 Замыкание лямбды: Шаг 14. Окончередной результат — возврат в REPL}
    \begin{columns}[c]

        % ======== левая колонка: код ========
        \column{0.4\textwidth}
        \centering
        \usebox{\codeLambdaExample}
        \vspace{0.4cm}

        Примитивная операция \mintinline{scheme}{(+ 112 3)} вычислена, получено окончательное значение \mintinline{scheme}{115}.

        \vfill

        % ======= правая колонка: схема ======
        \column{0.6\textwidth}
        \centering
        \begin{tikzpicture}[
            node distance=0.7cm,
            box/.style={
                draw=blue!70!black, fill=blue!5,
                rounded corners, thick, align=center,
                minimum width=3.8cm, minimum height=0.9cm,
                font=\footnotesize
            }
        ]
            \node[box] (repl) {\alert{REPL}};
            \node[box, below=of repl] (f1-1) {\mintinline{scheme}{ (f 4) }}
            edge [<-, thick, blue!70!black] (repl);
            \node[env-node=1, below=of repl, right=of f1-1] (stack1) { x $\mid$ 4 }
            edge [dashed, thick, blue!70!black] (f1-1);
            \node[box, below=of f1-1] (p1) {\mintinline{scheme}{ 115 }}
            edge [<-, thick, blue!70!black] (f1-1)
            edge [->, thick, blue!70!black, bend left=30] (f1-1)
            edge [dashed, thick, blue!70!black] (stack1);
        \end{tikzpicture}


    \end{columns}
\end{frame}

\begin{frame}[fragile]{2.1 Замыкание лямбды: Шаг 15. Конец вычисления}
    \begin{columns}[c]

        % ======== левая колонка: код ========
        \column{0.4\textwidth}
        \centering
        \usebox{\codeLambdaExample}
        \vspace{0.4cm}

        Процедура \mintinline{scheme}{f} полностью отработала и вернула значение \mintinline{scheme}{115} прямо в \alert{REPL}.

        \vfill

        % ======= правая колонка: схема ======
        \column{0.6\textwidth}
        \centering
        \begin{tikzpicture}[
            node distance=0.7cm,
            box/.style={
                draw=blue!70!black, fill=blue!5,
                rounded corners, thick, align=center,
                minimum width=3.8cm, minimum height=0.9cm,
                font=\footnotesize
            }
        ]
            \node[box] (repl) {\alert{REPL}};
            \node[box, below=of repl] (f1-1) {\mintinline{scheme}{ 115 }}
            edge [<-, thick, blue!70!black] (repl);
        \end{tikzpicture}


    \end{columns}
\end{frame}
\begin{frame}{2.2 Замыкание для хранения состояния}
    \centering

    Разберём по шагам, как обычная лямбда-функция «захватывает»
    переменные из места своего рождения, сохраняет их в куче и
    продолжает к ним обращаться даже после завершения всех вызовов

    \usebox{\codeClosureCounter}
\end{frame}


\begin{frame}[fragile]{2.2 Замыкание лямбды: Шаг 1. Определение \alert{counter}}
    \begin{columns}[c]

        % ======== левая колонка: код ========
        \column{0.4\textwidth}
        \centering
        \usebox{\codeClosureCounter}
        \vspace{0.4cm}

        В \alert{REPL} вводится определение переменной \mintinline{scheme}{counter}.
        Процедура создаётся с помощью немедленного вызова
        \mintinline{scheme}{(lambda (n) ...)} с начальным значением \mintinline{scheme}{n = 0}.

        \vfill

        % ======= правая колонка: схема ======
        \column{0.6\textwidth}
        \centering
        \begin{tikzpicture}[
            node distance=0.5cm,
            box/.style={
                draw=blue!70!black, fill=blue!5,
                rounded corners, thick, align=center,
                minimum width=2cm, minimum height=0.9cm,
                font=\footnotesize
            }
        ]
            \node[box] (repl) {\alert{REPL}};
            \node[box, above=of repl, align=left] (f1-1) {
                \mintinline{scheme}{(define counter } \\
                \phantom{\ \ (}\mintinline{scheme}{((lambda (n) } \\
                \phantom{\ \ \ \ (}\mintinline{scheme}{(lambda () } \\
                \phantom{\ \ \ \ \ \ (}\phantom{(}\mintinline{scheme}{(set! n (+ n 1)) } \\
                \phantom{\ \ \ \ \ \ (}\phantom{(}\mintinline{scheme}{n } \\
                \phantom{\ \ \ \ (}\mintinline{scheme}{0)) }
            }
                edge [<-, thick, blue!70!black] (repl);
        \end{tikzpicture}
    \end{columns}
\end{frame}


\begin{frame}[fragile]{2.2 Замыкание лямбды: Шаг 2. Завершение внешней лямбды}
    \begin{columns}[c]

        % ======== левая колонка: код ========
        \column{0.4\textwidth}
        \centering
        \usebox{\codeClosureCounter}
        \vspace{0.4cm}

        Вычисление \mintinline{scheme}{(define counter ...)} завершено.

        \vfill

        % ======= правая колонка: схема ======
        \column{0.6\textwidth}
        \centering
        \begin{tikzpicture}[
            node distance=0.5cm,
            box/.style={
                draw=blue!70!black, fill=blue!5,
                rounded corners, thick, align=center,
                minimum width=2cm, minimum height=0.9cm,
                font=\footnotesize
            }
        ]
            \node[box] (repl) {\alert{REPL}};
            \node[box, above=of repl, align=left] (f1-1) {
                \mintinline{scheme}{(define counter } \\
                \phantom{\ \ (}\mintinline{scheme}{((lambda (n) } \\
                \phantom{\ \ \ \ (}\mintinline{scheme}{(lambda () } \\
                \phantom{\ \ \ \ \ \ (}\phantom{(}\mintinline{scheme}{(set! n (+ n 1)) } \\
                \phantom{\ \ \ \ \ \ (}\phantom{(}\mintinline{scheme}{n } \\
                \phantom{\ \ \ \ (}\mintinline{scheme}{0)) }
            }
                edge [<-, thick, blue!70!black] (repl);
            \node[env-node=2, rectangle split, right=of repl] (stack1) { \mintinline{scheme}{counter} \nodepart{two} \mintinline{scheme}{((lambda (n) ...) 0)} }
                edge [dashed, thick, blue!70!black] (repl);
        \end{tikzpicture}
    \end{columns}
\end{frame}


\begin{frame}[fragile]{2.2 Замыкание лямбды: Шаг 3. Глобальная среда после определения}
    \begin{columns}[c]

        % ======== левая колонка: код ========
        \column{0.4\textwidth}
        \centering
        \usebox{\codeClosureCounter}
        \vspace{0.4cm}

        Определение \mintinline{scheme}{counter} полностью вычислено и
        записано в глобальную среду.

        \vfill

        % ======= правая колонка: схема ======
        \column{0.6\textwidth}
        \centering
        \begin{tikzpicture}[
            node distance=0.5cm,
            box/.style={
                draw=blue!70!black, fill=blue!5,
                rounded corners, thick, align=center,
                minimum width=2cm, minimum height=0.9cm,
                font=\footnotesize
            }
        ]
            \node[box] (repl) {\alert{REPL}};
            \node[box, above=of repl, align=left] (f1-1) {
                \mintinline{scheme}{(define counter } \\
                \phantom{\ \ (}\mintinline{scheme}{((lambda (n) } \\
                \phantom{\ \ \ \ (}\mintinline{scheme}{(lambda () } \\
                \phantom{\ \ \ \ \ \ (}\phantom{(}\mintinline{scheme}{(set! n (+ n 1)) } \\
                \phantom{\ \ \ \ \ \ (}\phantom{(}\mintinline{scheme}{n } \\
                \phantom{\ \ \ \ (}\mintinline{scheme}{0)) }
            }
                edge [<-, thick, blue!70!black] (repl);
            \node[env-node=2, rectangle split, right=of repl] (stack1) { \mintinline{scheme}{counter} \nodepart{two} \mintinline{scheme}{((lambda (n) ...) 0)} }
                edge [dashed, thick, blue!70!black] (repl);
            \node[env-node=1, right=of f1-1] (stack2) {n $\mid$ 0}
                edge [dashed, thick, blue!70!black] (f1-1);
        \end{tikzpicture}
    \end{columns}
\end{frame}


\begin{frame}[fragile]{2.2 Замыкание лямбды: Шаг 4. Замыкание полностью сформировано и «привязано» к своей среде}
    \begin{columns}[c]

        % ======== левая колонка: код ========
        \column{0.4\textwidth}
        \centering
        \usebox{\codeClosureCounter}
        \vspace{0.2cm}

        Замыкание полностью сформировано: \mintinline{scheme}{counter} ссылается на
        внутреннюю лямбду, которая «помнит» \mintinline{scheme}{n=0} — глобальная
        среда обновлена без изменений.

        \vfill

        % ======= правая колонка: схема ======
        \column{0.6\textwidth}
        \centering
        \begin{tikzpicture}[
            node distance=0.5cm,
            box/.style={
                draw=blue!70!black, fill=blue!5,
                rounded corners, thick, align=center,
                minimum width=2cm, minimum height=0.9cm,
                font=\footnotesize
            }
        ]
            \node[box] (repl) {\alert{REPL}};
            \node[box, above=of repl, align=left] (f1-1) {
                \mintinline{scheme}{(define counter } \\
                \phantom{\ \ (}\mintinline{scheme}{((lambda (n) } \\
                \phantom{\ \ \ \ (}\mintinline{scheme}{(lambda () } \\
                \phantom{\ \ \ \ \ \ (}\phantom{(}\mintinline{scheme}{(set! n (+ n 1)) } \\
                \phantom{\ \ \ \ \ \ (}\phantom{(}\mintinline{scheme}{n } \\
                \phantom{\ \ \ \ (}\mintinline{scheme}{0)) }
            }
                edge [<-, thick, blue!70!black] (repl);
            \node[env-node=2, rectangle split, right=of repl] (stack1) { \mintinline{scheme}{counter} \nodepart{two} \mintinline{scheme}{((lambda (n) ...) 0)} }
                edge [dashed, thick, blue!70!black] (repl);
            \node[env-node=1, right=of f1-1] (stack2) {n $\mid$ 0}
                edge [dashed, thick, blue!70!black] (f1-1)
                edge [dashed, thick, red!70!black] (stack1);
        \end{tikzpicture}
    \end{columns}
\end{frame}


\begin{frame}[fragile]{2.2 Замыкание лямбды: Шаг 5. Готовность к первому вызову}
    \begin{columns}[c]

        % ======== левая колонка: код ========
        \column{0.4\textwidth}
        \centering
        \usebox{\codeClosureCounter}
        \vspace{0.4cm}

        Выражение определения \mintinline{scheme}{(define counter ...)}
        полностью отработало и исчезло из вида.
        В стеке вызовов ничего нет — мы снова на верхнем уровне REPL.

        \vfill

        % ======= правая колонка: схема ======
        \column{0.6\textwidth}
        \centering
        \begin{tikzpicture}[
            node distance=0.5cm,
            box/.style={
                draw=blue!70!black, fill=blue!5,
                rounded corners, thick, align=center,
                minimum width=2cm, minimum height=0.9cm,
                font=\footnotesize
            }
        ]
            \node[box] (repl) {\alert{REPL}};
            \node[env-node=2, rectangle split, right=of repl] (stack1) { \mintinline{scheme}{counter} \nodepart{two} \mintinline{scheme}{((lambda (n) ...) 0)} }
                edge [dashed, thick, blue!70!black] (repl);
            \node[env-node=1, above=of stack1] (stack2) {n $\mid$ 0}
                edge [dashed, thick, red!70!black] (stack1);
        \end{tikzpicture}
    \end{columns}
\end{frame}


\begin{frame}[fragile]{2.2 Замыкание лямбды: Шаг 6. Вызов счётчика в первый раз}
    \begin{columns}[c]

        % ======== левая колонка: код ========
        \column{0.4\textwidth}
        \centering
        \usebox{\codeClosureCounter}
        \vspace{0.4cm}

        Среда \alert{REPL} выполняет выражение \mintinline{scheme}{(counter)}.

        \vfill

        % ======= правая колонка: схема ======
        \column{0.6\textwidth}
        \centering
        \begin{tikzpicture}[
            node distance=0.5cm,
            box/.style={
                draw=blue!70!black, fill=blue!5,
                rounded corners, thick, align=center,
                minimum width=2cm, minimum height=0.9cm,
                font=\footnotesize
            }
        ]
            \node[box] (repl) {\alert{REPL}};
            \node[env-node=2, rectangle split, right=of repl] (stack1) { \mintinline{scheme}{counter} \nodepart{two} \mintinline{scheme}{((lambda (n) ...) 0)} }
                edge [dashed, thick, blue!70!black] (repl);
            \node[env-node=1, above=of stack1] (stack2) {n $\mid$ 0}
                edge [dashed, thick, red!70!black] (stack1);
            \node[box, above=of repl] (c1) { \mintinline{scheme}{ (counter) } }
                edge [dashed, thick, blue!70!black] (stack1)
                edge [<-, thick, blue!70!black] (repl);
        \end{tikzpicture}
    \end{columns}
\end{frame}


\begin{frame}[fragile]{2.2 Замыкание лямбды: Шаг 7. Вход в тело замыкания}
    \begin{columns}[c]

        % ======== левая колонка: код ========
        \column{0.4\textwidth}
        \centering
        \usebox{\codeClosureCounter}
        \vspace{0.4cm}

        Вызов перенаправлен во внешнюю лямбду с \mintinline{scheme}{n=0}; тело
        начинает вычисляться — добавлен фрейм, ссылающийся на захваченную \mintinline{scheme}{n}.

        \vfill

        % ======= правая колонка: схема ======
        \column{0.6\textwidth}
        \centering
        \begin{tikzpicture}[
            node distance=0.5cm,
            box/.style={
                draw=blue!70!black, fill=blue!5,
                rounded corners, thick, align=center,
                minimum width=2cm, minimum height=0.9cm,
                font=\footnotesize
            }
        ]
            \node[box] (repl) {\alert{REPL}};
            \node[env-node=2, rectangle split, right=of repl] (stack1) { \mintinline{scheme}{counter} \nodepart{two} \mintinline{scheme}{((lambda (n) ...) 0)} }
                edge [dashed, thick, blue!70!black] (repl);
            \node[env-node=1, above=of stack1] (stack2) {n $\mid$ 0}
                edge [dashed, thick, red!70!black] (stack1);
            \node[box, above=of repl, align=left] (c1) {
                \mintinline{scheme}{ ((lambda (n) } \\
                \phantom{\ \ (}\mintinline{scheme}{(lambda () } \\
                \phantom{\ \ \ \ (}\phantom{(}\mintinline{scheme}{(set! n (+ n 1)) } \\
                \phantom{\ \ \ \ (}\phantom{(}\mintinline{scheme}{n)) } \\
                \phantom{\ \ (}\mintinline{scheme}{0) }
            }
                edge [dashed, thick, blue!70!black] (stack1)
                edge [dashed, thick, blue!70!black] (stack2)
                edge [<-, thick, blue!70!black] (repl);
        \end{tikzpicture}
    \end{columns}
\end{frame}


\begin{frame}[fragile]{2.2 Замыкание лямбды: Шаг 8. Создание внутренней лямбды}
    \begin{columns}[c]

        % ======== левая колонка: код ========
        \column{0.4\textwidth}
        \centering
        \usebox{\codeClosureCounter}
        \vspace{0.4cm}

        Интерпретатор дошёл до тела внешней лямбды и теперь вычисляет её
        единственное выражение — внутреннюю лямбду

        \vfill

        % ======= правая колонка: схема ======
        \column{0.6\textwidth}
        \centering
        \begin{tikzpicture}[
            node distance=0.6cm,
            box/.style={
                draw=blue!70!black, fill=blue!5,
                rounded corners, thick, align=center,
                minimum width=2cm, minimum height=0.9cm,
                font=\footnotesize
            }
        ]
            \node[box] (repl) {\alert{REPL}};
            \node[env-node=2, rectangle split, right=of repl] (stack1) { \mintinline{scheme}{counter} \nodepart{two} \mintinline{scheme}{((lambda (n) ...) 0)} }
                edge [dashed, thick, blue!70!black] (repl);
            \node[env-node=1, above=1cm of stack1] (stack2) {n $\mid$ 0}
                edge [dashed, thick, red!70!black] (stack1);
            \node[box, above=of repl, align=left] (c1) {
                \mintinline{scheme}{ ((lambda (n) } \\
                \phantom{\ \ (}\mintinline{scheme}{(lambda () } \\
                \phantom{\ \ \ \ (}\phantom{(}\mintinline{scheme}{(set! n (+ n 1)) } \\
                \phantom{\ \ \ \ (}\phantom{(}\mintinline{scheme}{n)) } \\
                \phantom{\ \ (}\mintinline{scheme}{0) }
            }
                edge [dashed, thick, blue!70!black] (stack1)
                edge [dashed, thick, blue!70!black] (stack2)
                edge [<-, thick, blue!70!black] (repl);
            \node[box, above=of c1, align=left] (c2) {
                \mintinline{scheme}{ (lambda () } \\
                \phantom{\ \ (}\phantom{(}\mintinline{scheme}{(set! n (+ n 1)) } \\
                \phantom{\ \ (}\phantom{(}\mintinline{scheme}{n)) } \\
            }
                edge [dashed, thick, blue!70!black, bend left=40] (stack2)
                edge [<-, thick, blue!70!black] (c1);
        \end{tikzpicture}
    \end{columns}
\end{frame}


\begin{frame}[fragile]{2.2 Замыкание лямбды: Шаг 9. Вызов внутренней лямбды}
    \begin{columns}[c]

        % ======== левая колонка: код ========
        \column{0.4\textwidth}
        \centering
        \usebox{\codeClosureCounter}
        \vspace{0.4cm}

        Внутренняя лямбда вызвана: интерпретатор входит в её тело и начинает
        \mintinline{scheme}{set!} — фрейм ссылается на
        \mintinline{scheme}{n=0}, подготавливая инкремент

        \vfill

        % ======= правая колонка: схема ======
        \column{0.6\textwidth}
        \centering
        \begin{tikzpicture}[
            node distance=0.6cm,
            box/.style={
                draw=blue!70!black, fill=blue!5,
                rounded corners, thick, align=center,
                minimum width=2cm, minimum height=0.9cm,
                font=\footnotesize
            }
        ]
            \node[box] (repl) {\alert{REPL}};
            \node[env-node=2, rectangle split, right=of repl] (stack1) { \mintinline{scheme}{counter} \nodepart{two} \mintinline{scheme}{((lambda (n) ...) 0)} }
                edge [dashed, thick, blue!70!black] (repl);
            \node[env-node=1, above=1cm of stack1] (stack2) {n $\mid$ 0}
                edge [dashed, thick, red!70!black] (stack1);
            \node[box, above=of repl, align=left] (c1) {
                \mintinline{scheme}{ ((lambda (n) } \\
                \phantom{\ \ (}\mintinline{scheme}{(lambda () } \\
                \phantom{\ \ \ \ (}\phantom{(}\mintinline{scheme}{(set! n (+ n 1)) } \\
                \phantom{\ \ \ \ (}\phantom{(}\mintinline{scheme}{n)) } \\
                \phantom{\ \ (}\mintinline{scheme}{0) }
            }
                edge [dashed, thick, blue!70!black] (stack1)
                edge [dashed, thick, blue!70!black] (stack2)
                edge [<-, thick, blue!70!black] (repl);
            \node[box, above=of c1, align=left] (c2) {
                \mintinline{scheme}{ (lambda () } \\
                \phantom{\ \ (}\phantom{(}\mintinline{scheme}{(set! n (+ n 1)) } \\
                \phantom{\ \ (}\phantom{(}\mintinline{scheme}{n)) } \\
            }
                edge [dashed, thick, blue!70!black, bend left=40] (stack2)
                edge [<-, thick, blue!70!black] (c1);
            \node[box, right=of c2, yshift=0.5cm] (c3) {
                \mintinline{scheme}{ (set! n (+ n 1)) } }
                edge [dashed, thick, blue!70!black] (stack2)
                edge [<-, thick, blue!70!black] (c2);
        \end{tikzpicture}
    \end{columns}
\end{frame}


\begin{frame}[fragile]{2.2 Замыкание лямбды: Шаг 10. Изменение захваченной переменной}
    \begin{columns}[c]

        % ======== левая колонка: код ========
        \column{0.4\textwidth}
        \centering
        \usebox{\codeClosureCounter}
        \vspace{0.4cm}

        Выражение \mintinline{scheme}{(set! n (+ n 1))} уже вычислено и
        превратилось в \mintinline{scheme}{(set! n 1)}.

        \vfill

        % ======= правая колонка: схема ======
        \column{0.6\textwidth}
        \centering
        \begin{tikzpicture}[
            node distance=0.6cm,
            box/.style={
                draw=blue!70!black, fill=blue!5,
                rounded corners, thick, align=center,
                minimum width=2cm, minimum height=0.9cm,
                font=\footnotesize
            }
        ]
            \node[box] (repl) {\alert{REPL}};
            \node[env-node=2, rectangle split, right=of repl] (stack1) { \mintinline{scheme}{counter} \nodepart{two} \mintinline{scheme}{((lambda (n) ...) 0)} }
                edge [dashed, thick, blue!70!black] (repl);
            \node[env-node=1, above=1cm of stack1] (stack2) {n $\mid$ 0}
                edge [dashed, thick, red!70!black] (stack1);
            \node[box, above=of repl, align=left] (c1) {
                \mintinline{scheme}{ ((lambda (n) } \\
                \phantom{\ \ (}\mintinline{scheme}{(lambda () } \\
                \phantom{\ \ \ \ (}\phantom{(}\mintinline{scheme}{(set! n (+ n 1)) } \\
                \phantom{\ \ \ \ (}\phantom{(}\mintinline{scheme}{n)) } \\
                \phantom{\ \ (}\mintinline{scheme}{0) }
            }
                edge [dashed, thick, blue!70!black] (stack1)
                edge [dashed, thick, blue!70!black] (stack2)
                edge [<-, thick, blue!70!black] (repl);
            \node[box, above=of c1, align=left] (c2) {
                \mintinline{scheme}{ (lambda () } \\
                \phantom{\ \ (}\phantom{(}\mintinline{scheme}{(set! n (+ n 1)) } \\
                \phantom{\ \ (}\phantom{(}\mintinline{scheme}{n)) } \\
            }
                edge [dashed, thick, blue!70!black, bend left=40] (stack2)
                edge [<-, thick, blue!70!black] (c1);
            \node[box, right=of c2, yshift=0.5cm] (c3) {
                \mintinline{scheme}{ (set! n 1) } }
                edge [dashed, thick, blue!70!black] (stack2)
                edge [<-, thick, blue!70!black] (c2);
        \end{tikzpicture}
    \end{columns}
\end{frame}


\begin{frame}[fragile]{2.2 Замыкание лямбды: Шаг 11. Возврат значения из замыкания}
    \begin{columns}[c]

        % ======== левая колонка: код ========
        \column{0.4\textwidth}
        \centering
        \usebox{\codeClosureCounter}
        \vspace{0.4cm}

        Присваивание \mintinline{scheme}{(set! n (+ n 1))} завершено — в
        захваченной ячейке теперь лежит \mintinline{scheme}{n=1}.

        \vfill

        % ======= правая колонка: схема ======
        \column{0.6\textwidth}
        \centering
        \begin{tikzpicture}[
            node distance=0.6cm,
            box/.style={
                draw=blue!70!black, fill=blue!5,
                rounded corners, thick, align=center,
                minimum width=2cm, minimum height=0.9cm,
                font=\footnotesize
            }
        ]
            \node[box] (repl) {\alert{REPL}};
            \node[env-node=2, rectangle split, right=of repl] (stack1) { \mintinline{scheme}{counter} \nodepart{two} \mintinline{scheme}{((lambda (n) ...) 0)} }
                edge [dashed, thick, blue!70!black] (repl);
            \node[env-node=1, above=1cm of stack1] (stack2) {n $\mid$ 1}
                edge [dashed, thick, red!70!black] (stack1);
            \node[box, above=of repl, align=left] (c1) {
                \mintinline{scheme}{ ((lambda (n) } \\
                \phantom{\ \ (}\mintinline{scheme}{(lambda () } \\
                \phantom{\ \ \ \ (}\phantom{(}\mintinline{scheme}{(set! n (+ n 1)) } \\
                \phantom{\ \ \ \ (}\phantom{(}\mintinline{scheme}{n)) } \\
                \phantom{\ \ (}\mintinline{scheme}{0) }
            }
                edge [dashed, thick, blue!70!black] (stack1)
                edge [dashed, thick, blue!70!black] (stack2)
                edge [<-, thick, blue!70!black] (repl);
            \node[box, above=of c1] (c2) {
                \mintinline{scheme}{ n }
            }
                edge [dashed, thick, blue!70!black, bend left=40] (stack2)
                edge [<-, thick, blue!70!black] (c1);
        \end{tikzpicture}
    \end{columns}
\end{frame}


\begin{frame}[fragile]{2.2 Замыкание лямбды: Шаг 12. Завершение первого вызова}
    \begin{columns}[c]

        % ======== левая колонка: код ========
        \column{0.4\textwidth}
        \centering
        \usebox{\codeClosureCounter}
        \vspace{0.4cm}

        Вызов \mintinline{scheme}{(counter)} полностью
        завершён и вернул значение \mintinline{scheme}{1}.

        \vfill

        % ======= правая колонка: схема ======
        \column{0.6\textwidth}
        \centering
        \begin{tikzpicture}[
            node distance=0.6cm,
            box/.style={
                draw=blue!70!black, fill=blue!5,
                rounded corners, thick, align=center,
                minimum width=2cm, minimum height=0.9cm,
                font=\footnotesize
            }
        ]
            \node[box] (repl) {\alert{REPL}};
            \node[env-node=2, rectangle split, right=of repl] (stack1) { \mintinline{scheme}{counter} \nodepart{two} \mintinline{scheme}{((lambda (n) ...) 0)} }
                edge [dashed, thick, blue!70!black] (repl);
            \node[env-node=1, above=1cm of stack1] (stack2) {n $\mid$ 1}
                edge [dashed, thick, red!70!black] (stack1);
            \node[box, above=of repl] (c1) { 1 }
                edge [<-, thick, blue!70!black] (repl);
        \end{tikzpicture}
    \end{columns}
\end{frame}


\end{document}
