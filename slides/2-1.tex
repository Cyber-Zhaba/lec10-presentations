\begin{frame}{Глава 2: Замыкание лямбда-выражения}
    \centering

    На примере функций \mintinline{scheme}{f} и \mintinline{scheme}{g}
    разберем, как лямбда-выражение захватывает внешнюю переменную
    \mintinline{scheme}{x} и сохраняет доступ к ней в замыкании — механизм,
    позволяющий функциям «помнить» окружение.

    \usebox{\codeLambdaExample}
\end{frame}


\begin{frame}[fragile]{2.1 Замыкание лямбды: Шаг 1. Начальный вызов процедуры f}
    \begin{columns}[c]

        % ======== левая колонка: код ========
        \column{0.4\textwidth}
        \centering
        \usebox{\codeLambdaExample}
        \vspace{0.4cm}

        Среда \alert{REPL} вызывает пользовательскую процедуру \mintinline{scheme}{f} с аргументом \mintinline{scheme}{4}.

        \vfill

        % ======= правая колонка: схема ======
        \column{0.6\textwidth}
        \centering
        \begin{tikzpicture}[
            node distance=0.5cm,
            box/.style={
                draw=blue!70!black, fill=blue!5,
                rounded corners, thick, align=center,
                minimum width=3.8cm, minimum height=0.9cm,
                font=\footnotesize
            }
        ]
            \node[box] (repl) {\alert{REPL}};
            \node[box, below=of repl] (f1-1) {\mintinline{scheme}{ (f 4) }}
            edge [<-, thick, blue!70!black] (repl);
        \end{tikzpicture}

    \end{columns}
\end{frame}


\begin{frame}[fragile]{2.1 Замыкание лямбды: Шаг 2. Начало вычисления тела процедуры f}
    \begin{columns}[c]

        % ======== левая колонка: код ========
        \column{0.4\textwidth}
        \centering
        \usebox{\codeLambdaExample}
        \vspace{0.4cm}

        Интерпретатор приступил к вычислению тела процедуры \mintinline{scheme}{f}: формы \mintinline{scheme}{(+ (g ...) 3)}.

        \vfill

        % ======= правая колонка: схема ======
        \column{0.6\textwidth}
        \centering
        \begin{tikzpicture}[
            node distance=0.8cm,
            box/.style={
                draw=blue!70!black, fill=blue!5,
                rounded corners, thick, align=center,
                minimum width=3.8cm, minimum height=0.9cm,
                font=\footnotesize
            }
        ]
            \node[box] (repl) {\alert{REPL}};
            \node[box, below=of repl] (f1-1) {\mintinline{scheme}{ (f 4) }}
            edge [<-, thick, blue!70!black] (repl);
            \node[env-node=1, below=of repl, right=of f1-1] (stack1) { x $\mid$ 4 }
            edge [dashed, thick, blue!70!black] (f1-1);
            \node[box, below=of f1-1] (p1) {\mintinline{scheme}{ (+ (g (lambda (y) (* x y)) 3) }}
            edge [<-, thick, blue!70!black] (f1-1)
            edge [dashed, thick, blue!70!black] (stack1);
        \end{tikzpicture}

    \end{columns}
\end{frame}

\begin{frame}[fragile]{2.1 Замыкание лямбды: Шаг 3. Вычисление фактического аргумента для вызова g}
    \begin{columns}[c]

        % ======== левая колонка: код ========
        \column{0.4\textwidth}
        \centering
        \usebox{\codeLambdaExample}
        \vspace{0.4cm}

        Интерпретатор завершил создание замыкания \mintinline{scheme}{(lambda (y) (* x y))} в окружающей среде, где \mintinline{scheme}{x} = 4.

        \vfill

        % ======= правая колонка: схема ======
        \column{0.6\textwidth}
        \centering
        \begin{tikzpicture}[
            node distance=0.8cm,
            box/.style={
                draw=blue!70!black, fill=blue!5,
                rounded corners, thick, align=center,
                minimum width=3.8cm, minimum height=0.9cm,
                font=\footnotesize
            }
        ]
            \node[box] (repl) {\alert{REPL}};
            \node[box, below=of repl] (f1-1) {\mintinline{scheme}{ (f 4) }}
            edge [<-, thick, blue!70!black] (repl);
            \node[env-node=1, below=of repl, right=of f1-1] (stack1) { x $\mid$ 4 }
            edge [dashed, thick, blue!70!black] (f1-1);
            \node[box, below=of f1-1] (p1) {\mintinline{scheme}{ (+ (g (lambda (y) (* x y)) 3) }}
            edge [<-, thick, blue!70!black] (f1-1)
            edge [dashed, thick, blue!70!black] (stack1);
            \node[box, below=of p1] (g1) {\mintinline{scheme}{  (g (lambda (y) (* x y)) }}
            edge [<-, thick, blue!70!black] (p1);
        \end{tikzpicture}

    \end{columns}
\end{frame}


\begin{frame}[fragile]{2.1 Замыкание лямбды: Шаг 4. Вызов процедуры g — рост стека}
    \begin{columns}[c]

        % ======== левая колонка: код ========
        \column{0.4\textwidth}
        \centering
        \usebox{\codeLambdaExample}
        \vspace{0.4cm}

        Интерпретатор выполнил применение процедуры \mintinline{scheme}{g} к замыканию \mintinline{scheme}{(lambda (y) (* x y))}.

        \vfill

        % ======= правая колонка: схема ======
        \column{0.6\textwidth}
        \centering
        \begin{tikzpicture}[
            node distance=0.7cm,
            box/.style={
                draw=blue!70!black, fill=blue!5,
                rounded corners, thick, align=center,
                minimum width=3.8cm, minimum height=0.9cm,
                font=\footnotesize
            }
        ]
            \node[box] (repl) {\alert{REPL}};
            \node[box, below=of repl] (f1-1) {\mintinline{scheme}{ (f 4) }}
            edge [<-, thick, blue!70!black] (repl);
            \node[env-node=1, below=of repl, right=1cm of f1-1] (stack1) { x $\mid$ 4 }
            edge [dashed, thick, blue!70!black] (f1-1);
            \node[box, below=of f1-1] (p1) {\mintinline{scheme}{ (+ (g (lambda (y) (* x y)) 3) }}
            edge [<-, thick, blue!70!black] (f1-1)
            edge [dashed, thick, blue!70!black] (stack1);
            \node[box, below=1.3cm of p1, xshift=-0.5cm] (g1) {\mintinline{scheme}{ (g (lambda (y) (* x y)) }}
            edge [<-, thick, blue!70!black, bend left=25] (p1);
            \node[env-node=1, right=of g1, above=0.7cm, xshift=0.6cm] (stack2) { h $\mid$ \mintinline{scheme}{ (lambda (y) (* x y)) } }
            edge [dashed, thick, blue!70!black, bend left=20] (g1)
            edge [dashed, thick, blue!70!black, bend right=30] (stack1);
        \end{tikzpicture}

    \end{columns}
\end{frame}


\begin{frame}[fragile]{2.1 Замыкание лямбды: Шаг 5. Начало вычисления тела процедуры g}
    \begin{columns}[c]

        % ======== левая колонка: код ========
        \column{0.4\textwidth}
        \centering
        \usebox{\codeLambdaExample}
        \vspace{0.4cm}

        Интерпретатор приступил к вычислению тела процедуры \mintinline{scheme}{g}

        \vfill

        % ======= правая колонка: схема ======
        \column{0.6\textwidth}
        \centering
        \begin{tikzpicture}[
            node distance=0.5cm,
            box/.style={
                draw=blue!70!black, fill=blue!5,
                rounded corners, thick, align=center,
                minimum width=3.8cm, minimum height=0.6cm,
                font=\footnotesize
            }
        ]
            \node[box] (repl) {\alert{REPL}};
            \node[box, below=of repl] (f1-1) {\mintinline{scheme}{ (f 4) }}
            edge [<-, thick, blue!70!black] (repl);
            \node[env-node=1, below=of repl, right=1cm of f1-1] (stack1) { x $\mid$ 4 }
            edge [dashed, thick, blue!70!black] (f1-1);
            \node[box, below=of f1-1] (p1) {\mintinline{scheme}{ (+ (g (lambda (y) (* x y)) 3) }}
            edge [<-, thick, blue!70!black] (f1-1)
            edge [dashed, thick, blue!70!black] (stack1);
            \node[box, below=1.4cm of p1, xshift=-0.7cm] (g1) {\mintinline{scheme}{ (g (lambda (y) (* x y)) }}
            edge [<-, thick, blue!70!black, bend left=35] (p1);
            \node[env-node=1, right=of g1, above=0.65cm, xshift=0.6cm] (stack2) { h $\mid$ \mintinline{scheme}{ (lambda (y) (* x y)) } }
            edge [dashed, thick, blue!70!black, bend left=20] (g1)
            edge [dashed, thick, blue!70!black, bend right=45] (stack1);
            \node[box, below=of g1] (h1) {\mintinline{scheme}{ (h (h 7)) }}
            edge [dashed, thick, blue!70!black, bend right=30] (stack2)
            edge [<-, thick, blue!70!black] (g1);
        \end{tikzpicture}

    \end{columns}
\end{frame}

\begin{frame}[fragile]{2.1 Замыкание лямбды: Шаг 6. Первый (внешний) вызов замыкания h — рост стека продолжается}
    \begin{columns}[c]

        % ======== левая колонка: код ========
        \column{0.4\textwidth}
        \centering
        \usebox{\codeLambdaExample}
        \vspace{0.4cm}

        Интерпретатор вычислил аргумент внутреннего вызова — число \mintinline{scheme}{7} — и теперь готов применить процедуру \mintinline{scheme}{h} к этому аргументу.

        \vfill

        % ======= правая колонка: схема ======
        \column{0.6\textwidth}
        \centering
        \begin{tikzpicture}[
            node distance=0.4cm,
            box/.style={
                draw=blue!70!black, fill=blue!5,
                rounded corners, thick, align=center,
                minimum width=3.8cm, minimum height=0.55cm,
                font=\footnotesize
            }
        ]
            \node[box] (repl) {\alert{REPL}};
            \node[box, below=of repl] (f1-1) {\mintinline{scheme}{ (f 4) }}
            edge [<-, thick, blue!70!black] (repl);
            \node[env-node=1, below=of repl, right=1cm of f1-1] (stack1) { x $\mid$ 4 }
            edge [dashed, thick, blue!70!black] (f1-1);
            \node[box, below=of f1-1] (p1) {\mintinline{scheme}{ (+ (g (lambda (y) (* x y)) 3) }}
            edge [<-, thick, blue!70!black] (f1-1)
            edge [dashed, thick, blue!70!black] (stack1);
            \node[box, below=1.4cm of p1, xshift=-0.7cm] (g1) {\mintinline{scheme}{ (g (lambda (y) (* x y)) }}
            edge [<-, thick, blue!70!black, bend left=35] (p1);
            \node[env-node=1, right=of g1, above=0.65cm, xshift=0.6cm] (stack2) { h $\mid$ \mintinline{scheme}{ (lambda (y) (* x y)) } }
            edge [dashed, thick, blue!70!black, bend left=20] (g1)
            edge [dashed, thick, blue!70!black, bend right=45] (stack1);
            \node[box, below=of g1] (h1) {\mintinline{scheme}{ (h (h 7)) }}
            edge [dashed, thick, blue!70!black, bend right=30] (stack2)
            edge [<-, thick, blue!70!black] (g1);
            \node[box, below=of h1] (h2) {\mintinline{scheme}{ (h 7) }}
            edge [dashed, thick, blue!70!black, bend right=35] (stack2)
            edge [<-, thick, blue!70!black] (h1);
        \end{tikzpicture}


    \end{columns}
\end{frame}
\begin{frame}[fragile]{2.1 Замыкание лямбды: Шаг 7. Первое применение замыкания}
    \begin{columns}[c]

        % ======== левая колонка: код ========
        \column{0.4\textwidth}
        \centering
        \usebox{\codeLambdaExample}
        \vspace{0.4cm}

        Интерпретатор выполнил первый вызов замыкания \mintinline{scheme}{h} к аргументом \mintinline{scheme}{7}.

        \vfill

        % ======= правая колонка: схема ======
        \column{0.6\textwidth}
        \centering
        \begin{tikzpicture}[
            node distance=0.4cm,
            box/.style={
                draw=blue!70!black, fill=blue!5,
                rounded corners, thick, align=center,
                minimum width=3.8cm, minimum height=0.55cm,
                font=\footnotesize
            }
        ]
            \node[box] (repl) {\alert{REPL}};
            \node[box, below=of repl] (f1-1) {\mintinline{scheme}{ (f 4) }}
            edge [<-, thick, blue!70!black] (repl);
            \node[env-node=1, below=of repl, right=1cm of f1-1] (stack1) { x $\mid$ 4 }
            edge [dashed, thick, blue!70!black] (f1-1);
            \node[box, below=of f1-1] (p1) {\mintinline{scheme}{ (+ (g (lambda (y) (* x y)) 3) }}
            edge [<-, thick, blue!70!black] (f1-1)
            edge [dashed, thick, blue!70!black] (stack1);
            \node[box, below=1.4cm of p1, xshift=-0.7cm] (g1) {\mintinline{scheme}{ (g (lambda (y) (* x y)) }}
            edge [<-, thick, blue!70!black, bend left=35] (p1);
            \node[env-node=1, right=of g1, above=0.65cm, xshift=0.6cm] (stack2) { h $\mid$ \mintinline{scheme}{ (lambda (y) (* x y)) } }
            edge [dashed, thick, blue!70!black, bend left=20] (g1)
            edge [dashed, thick, blue!70!black, bend right=45] (stack1);
            \node[box, below=of g1] (h1) {\mintinline{scheme}{ (h (h 7)) }}
            edge [dashed, thick, blue!70!black, bend right=30] (stack2)
            edge [<-, thick, blue!70!black] (g1);
            \node[box, below=of h1] (h2) {\mintinline{scheme}{ ((lambda (y) (* x y)) 7) }}
            edge [dashed, thick, blue!70!black, bend right=35] (stack2)
            edge [dashed, thick, red!70!black, bend right=40] (stack1)
            edge [<-, thick, blue!70!black] (h1);
            \node[env-node=1, below=of repl, right=1cm of h2] (stack3) { y $\mid$ 7 }
            edge [dashed, thick, blue!70!black] (h2);
        \end{tikzpicture}


    \end{columns}
\end{frame}

\begin{frame}[fragile]{2.1 Замыкание лямбды: Шаг 8. Завершение первого применения замыкания}
    \begin{columns}[c]

        % ======== левая колонка: код ========
        \column{0.4\textwidth}
        \centering
        \usebox{\codeLambdaExample}
        \vspace{0.4cm}

        Тело лямбда-выражения \mintinline{scheme}{(* x y)} полностью вычислено

        \vfill

        % ======= правая колонка: схема ======
        \column{0.6\textwidth}
        \centering
        \begin{tikzpicture}[
            node distance=0.4cm,
            box/.style={
                draw=blue!70!black, fill=blue!5,
                rounded corners, thick, align=center,
                minimum width=3.8cm, minimum height=0.55cm,
                font=\footnotesize
            }
        ]
            \node[box] (repl) {\alert{REPL}};
            \node[box, below=of repl] (f1-1) {\mintinline{scheme}{ (f 4) }}
            edge [<-, thick, blue!70!black] (repl);
            \node[env-node=1, below=of repl, right=1cm of f1-1] (stack1) { x $\mid$ 4 }
            edge [dashed, thick, blue!70!black] (f1-1);
            \node[box, below=of f1-1] (p1) {\mintinline{scheme}{ (+ (g (lambda (y) (* x y)) 3) }}
            edge [<-, thick, blue!70!black] (f1-1)
            edge [dashed, thick, blue!70!black] (stack1);
            \node[box, below=1.4cm of p1, xshift=-0.7cm] (g1) {\mintinline{scheme}{ (g (lambda (y) (* x y)) }}
            edge [<-, thick, blue!70!black, bend left=35] (p1);
            \node[env-node=1, right=of g1, above=0.65cm, xshift=0.6cm] (stack2) { h $\mid$ \mintinline{scheme}{ (lambda (y) (* x y)) } }
            edge [dashed, thick, blue!70!black, bend left=20] (g1)
            edge [dashed, thick, blue!70!black, bend right=45] (stack1);
            \node[box, below=of g1] (h1) {\mintinline{scheme}{ (h (h 7)) }}
            edge [dashed, thick, blue!70!black, bend right=30] (stack2)
            edge [<-, thick, blue!70!black] (g1);
            \node[box, below=of h1] (h2) {\mintinline{scheme}{ 28 }}
            edge [dashed, thick, blue!70!black, bend right=35] (stack2)
            edge [dashed, thick, red!70!black, bend right=40] (stack1)
            edge [->, thick, blue!70!black, bend left=30] (h1)
            edge [<-, thick, blue!70!black] (h1);
            \node[env-node=1, below=of repl, right=1cm of h2] (stack3) { y $\mid$ 7 }
            edge [dashed, thick, blue!70!black] (h2);
        \end{tikzpicture}


    \end{columns}
\end{frame}

\begin{frame}[fragile]{2.1 Замыкание лямбды: Шаг 9. Возврат из первого замыкания}
    \begin{columns}[c]

        % ======== левая колонка: код ========
        \column{0.4\textwidth}
        \centering
        \usebox{\codeLambdaExample}
        \vspace{0.4cm}

        Первый вызов замыкания \mintinline{scheme}{(lambda (y) (* x y))} завершён и полностью вытеснен из стека (фрейм с \mintinline{scheme}{y | 7} удалён).

        \vfill

        % ======= правая колонка: схема ======
        \column{0.6\textwidth}
        \centering
        \begin{tikzpicture}[
            node distance=0.5cm,
            box/.style={
                draw=blue!70!black, fill=blue!5,
                rounded corners, thick, align=center,
                minimum width=3.8cm, minimum height=0.6cm,
                font=\footnotesize
            }
        ]
            \node[box] (repl) {\alert{REPL}};
            \node[box, below=of repl] (f1-1) {\mintinline{scheme}{ (f 4) }}
            edge [<-, thick, blue!70!black] (repl);
            \node[env-node=1, below=of repl, right=1cm of f1-1] (stack1) { x $\mid$ 4 }
            edge [dashed, thick, blue!70!black] (f1-1);
            \node[box, below=of f1-1] (p1) {\mintinline{scheme}{ (+ (g (lambda (y) (* x y)) 3) }}
            edge [<-, thick, blue!70!black] (f1-1)
            edge [dashed, thick, blue!70!black] (stack1);
            \node[box, below=1.4cm of p1, xshift=-0.7cm] (g1) {\mintinline{scheme}{ (g (lambda (y) (* x y)) }}
            edge [<-, thick, blue!70!black, bend left=35] (p1);
            \node[env-node=1, right=of g1, above=0.65cm, xshift=0.6cm] (stack2) { h $\mid$ \mintinline{scheme}{ (lambda (y) (* x y)) } }
            edge [dashed, thick, blue!70!black, bend left=20] (g1)
            edge [dashed, thick, blue!70!black, bend right=45] (stack1);
            \node[box, below=of g1] (h1) {\mintinline{scheme}{ (h 28) }}
            edge [dashed, thick, blue!70!black, bend right=30] (stack2)
            edge [<-, thick, blue!70!black] (g1);
        \end{tikzpicture}


    \end{columns}
\end{frame}

\begin{frame}[fragile]{2.1 Замыкание лямбды: Шаг 10. Второй вызов того же замыкания}
    \begin{columns}[c]

        % ======== левая колонка: код ========
        \column{0.4\textwidth}
        \centering
        \usebox{\codeLambdaExample}
        \vspace{0.4cm}

        Интерпретатор выполнил второй вызов замыкания \mintinline{scheme}{h} уже с аргументом \mintinline{scheme}{28}.

        \vfill

        % ======= правая колонка: схема ======
        \column{0.6\textwidth}
        \centering
        \begin{tikzpicture}[
            node distance=0.5cm,
            box/.style={
                draw=blue!70!black, fill=blue!5,
                rounded corners, thick, align=center,
                minimum width=3.8cm, minimum height=0.6cm,
                font=\footnotesize
            }
        ]
            \node[box] (repl) {\alert{REPL}};
            \node[box, below=of repl] (f1-1) {\mintinline{scheme}{ (f 4) }}
            edge [<-, thick, blue!70!black] (repl);
            \node[env-node=1, below=of repl, right=1cm of f1-1] (stack1) { x $\mid$ 4 }
            edge [dashed, thick, blue!70!black] (f1-1);
            \node[box, below=of f1-1] (p1) {\mintinline{scheme}{ (+ (g (lambda (y) (* x y)) 3) }}
            edge [<-, thick, blue!70!black] (f1-1)
            edge [dashed, thick, blue!70!black] (stack1);
            \node[box, below=1.4cm of p1, xshift=-0.7cm] (g1) {\mintinline{scheme}{ (g (lambda (y) (* x y)) }}
            edge [<-, thick, blue!70!black, bend left=35] (p1);
            \node[env-node=1, right=of g1, above=0.65cm, xshift=0.6cm] (stack2) { h $\mid$ \mintinline{scheme}{ (lambda (y) (* x y)) } }
            edge [dashed, thick, blue!70!black, bend left=20] (g1)
            edge [dashed, thick, blue!70!black, bend right=45] (stack1);
            \node[box, below=of g1] (h1) {\mintinline{scheme}{ ((lambda (y) (* x y)) 28) }}
            edge [dashed, thick, blue!70!black, bend right=30] (stack2)
            edge [dashed, thick, red!70!black, bend right=40] (stack1)
            edge [<-, thick, blue!70!black] (g1);
            \node[env-node=1, below=of g1, right=1cm of h1] (stack3) { y $\mid$ 28 }
            edge [dashed, thick, blue!70!black] (h1);
        \end{tikzpicture}


    \end{columns}
\end{frame}

\begin{frame}[fragile]{2.1 Замыкание лямбды: Шаг 11. Завершение второго применения замыкания}
    \begin{columns}[c]

        % ======== левая колонка: код ========
        \column{0.4\textwidth}
        \centering
        \usebox{\codeLambdaExample}
        \vspace{0.4cm}

        Тело лямбда-выражения \mintinline{scheme}{(* x y)} вычислено во второй раз:
        при \mintinline{scheme}{x} = 4 и \mintinline{scheme}{y} = 28 получено значение \mintinline{scheme}{112}.

        \vfill

        % ======= правая колонка: схема ======
        \column{0.6\textwidth}
        \centering
        \begin{tikzpicture}[
            node distance=0.5cm,
            box/.style={
                draw=blue!70!black, fill=blue!5,
                rounded corners, thick, align=center,
                minimum width=3.8cm, minimum height=0.6cm,
                font=\footnotesize
            }
        ]
            \node[box] (repl) {\alert{REPL}};
            \node[box, below=of repl] (f1-1) {\mintinline{scheme}{ (f 4) }}
            edge [<-, thick, blue!70!black] (repl);
            \node[env-node=1, below=of repl, right=1cm of f1-1] (stack1) { x $\mid$ 4 }
            edge [dashed, thick, blue!70!black] (f1-1);
            \node[box, below=of f1-1] (p1) {\mintinline{scheme}{ (+ (g (lambda (y) (* x y)) 3) }}
            edge [<-, thick, blue!70!black] (f1-1)
            edge [dashed, thick, blue!70!black] (stack1);
            \node[box, below=1.4cm of p1, xshift=-0.7cm] (g1) {\mintinline{scheme}{ (g (lambda (y) (* x y)) }}
            edge [<-, thick, blue!70!black, bend left=35] (p1);
            \node[env-node=1, right=of g1, above=0.65cm, xshift=0.6cm] (stack2) { h $\mid$ \mintinline{scheme}{ (lambda (y) (* x y)) } }
            edge [dashed, thick, blue!70!black, bend left=20] (g1)
            edge [dashed, thick, blue!70!black, bend right=45] (stack1);
            \node[box, below=of g1] (h1) {\mintinline{scheme}{ 112 }}
            edge [dashed, thick, blue!70!black, bend right=30] (stack2)
            edge [dashed, thick, red!70!black, bend right=40] (stack1)
            edge [->, thick, blue!70!black, bend left=30] (g1)
            edge [<-, thick, blue!70!black] (g1);
            \node[env-node=1, below=of g1, right=1cm of h1] (stack3) { y $\mid$ 28 }
            edge [dashed, thick, blue!70!black] (h1);
        \end{tikzpicture}


    \end{columns}
\end{frame}

\begin{frame}[fragile]{2.1 Замыкание лямбды: Шаг 12. Завершение вызова g — возврат значения 112 в тело процедуры f}
    \begin{columns}[c]

        % ======== левая колонка: код ========
        \column{0.4\textwidth}
        \centering
        \usebox{\codeLambdaExample}
        \vspace{0.4cm}

        Процедура \mintinline{scheme}{g} полностью отработала и вернула значение \mintinline{scheme}{112}.

        \vfill

        % ======= правая колонка: схема ======
        \column{0.6\textwidth}
        \centering
        \begin{tikzpicture}[
            node distance=0.7cm,
            box/.style={
                draw=blue!70!black, fill=blue!5,
                rounded corners, thick, align=center,
                minimum width=3.8cm, minimum height=0.9cm,
                font=\footnotesize
            }
        ]
            \node[box] (repl) {\alert{REPL}};
            \node[box, below=of repl] (f1-1) {\mintinline{scheme}{ (f 4) }}
            edge [<-, thick, blue!70!black] (repl);
            \node[env-node=1, below=of repl, right=1cm of f1-1] (stack1) { x $\mid$ 4 }
            edge [dashed, thick, blue!70!black] (f1-1);
            \node[box, below=of f1-1] (p1) {\mintinline{scheme}{ (+ (g (lambda (y) (* x y)) 3) }}
            edge [<-, thick, blue!70!black] (f1-1)
            edge [dashed, thick, blue!70!black] (stack1);
            \node[box, below=1.3cm of p1, xshift=-1cm] (g1) {112}
            edge [->, thick, blue!70!black, bend left=30] (p1)
            edge [<-, thick, blue!70!black] (p1);
            \node[env-node=1, right=of g1, above=0.7cm, xshift=1.8cm] (stack2) { h $\mid$ \mintinline{scheme}{ (lambda (y) (* x y)) } }
            edge [dashed, thick, blue!70!black, bend left=20] (g1)
            edge [dashed, thick, blue!70!black, bend right=30] (stack1);
        \end{tikzpicture}


    \end{columns}
\end{frame}


\begin{frame}[fragile]{2.1 Замыкание лямбды: Шаг 13. инальное вычисление в теле процедуры}
    \begin{columns}[c]

        % ======== левая колонка: код ========
        \column{0.4\textwidth}
        \centering
        \usebox{\codeLambdaExample}
        \vspace{0.4cm}

        Вызов \mintinline{scheme}{g} полностью завершён, его результат \mintinline{scheme}{112} подставлен в тело процедуры \mintinline{scheme}{f}.

        \vfill

        % ======= правая колонка: схема ======
        \column{0.6\textwidth}
        \centering
        \begin{tikzpicture}[
            node distance=0.7cm,
            box/.style={
                draw=blue!70!black, fill=blue!5,
                rounded corners, thick, align=center,
                minimum width=3.8cm, minimum height=0.9cm,
                font=\footnotesize
            }
        ]
            \node[box] (repl) {\alert{REPL}};
            \node[box, below=of repl] (f1-1) {\mintinline{scheme}{ (f 4) }}
            edge [<-, thick, blue!70!black] (repl);
            \node[env-node=1, below=of repl, right=of f1-1] (stack1) { x $\mid$ 4 }
            edge [dashed, thick, blue!70!black] (f1-1);
            \node[box, below=of f1-1] (p1) {\mintinline{scheme}{ (+ 112 3) }}
            edge [<-, thick, blue!70!black] (f1-1)
            edge [dashed, thick, blue!70!black] (stack1);
        \end{tikzpicture}


    \end{columns}
\end{frame}

\begin{frame}[fragile]{2.1 Замыкание лямбды: Шаг 14. Окончередной результат — возврат в REPL}
    \begin{columns}[c]

        % ======== левая колонка: код ========
        \column{0.4\textwidth}
        \centering
        \usebox{\codeLambdaExample}
        \vspace{0.4cm}

        Примитивная операция \mintinline{scheme}{(+ 112 3)} вычислена, получено окончательное значение \mintinline{scheme}{115}.

        \vfill

        % ======= правая колонка: схема ======
        \column{0.6\textwidth}
        \centering
        \begin{tikzpicture}[
            node distance=0.7cm,
            box/.style={
                draw=blue!70!black, fill=blue!5,
                rounded corners, thick, align=center,
                minimum width=3.8cm, minimum height=0.9cm,
                font=\footnotesize
            }
        ]
            \node[box] (repl) {\alert{REPL}};
            \node[box, below=of repl] (f1-1) {\mintinline{scheme}{ (f 4) }}
            edge [<-, thick, blue!70!black] (repl);
            \node[env-node=1, below=of repl, right=of f1-1] (stack1) { x $\mid$ 4 }
            edge [dashed, thick, blue!70!black] (f1-1);
            \node[box, below=of f1-1] (p1) {\mintinline{scheme}{ 115 }}
            edge [<-, thick, blue!70!black] (f1-1)
            edge [->, thick, blue!70!black, bend left=30] (f1-1)
            edge [dashed, thick, blue!70!black] (stack1);
        \end{tikzpicture}


    \end{columns}
\end{frame}

\begin{frame}[fragile]{2.1 Замыкание лямбды: Шаг 15. Конец вычисления}
    \begin{columns}[c]

        % ======== левая колонка: код ========
        \column{0.4\textwidth}
        \centering
        \usebox{\codeLambdaExample}
        \vspace{0.4cm}

        Процедура \mintinline{scheme}{f} полностью отработала и вернула значение \mintinline{scheme}{115} прямо в \alert{REPL}.

        \vfill

        % ======= правая колонка: схема ======
        \column{0.6\textwidth}
        \centering
        \begin{tikzpicture}[
            node distance=0.7cm,
            box/.style={
                draw=blue!70!black, fill=blue!5,
                rounded corners, thick, align=center,
                minimum width=3.8cm, minimum height=0.9cm,
                font=\footnotesize
            }
        ]
            \node[box] (repl) {\alert{REPL}};
            \node[box, below=of repl] (f1-1) {\mintinline{scheme}{ 115 }}
            edge [<-, thick, blue!70!black] (repl);
        \end{tikzpicture}


    \end{columns}
\end{frame}