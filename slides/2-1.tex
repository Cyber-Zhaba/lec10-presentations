\begin{frame}{Глава 2: Замыкание лямбда-выражения}
    \centering

    На примере функций \mintinline{scheme}{f} и \mintinline{scheme}{g}
    разберем, как лямбда-выражение захватывает внешнюю переменную
    \mintinline{scheme}{x} и сохраняет доступ к ней в замыкании — механизм,
    позволяющий процедурам «помнить» окружение.

    \usebox{\codeLambdaExample}
\end{frame}


\begin{frame}[fragile]{2.1 Замыкание лямбды: Шаг 1. Вызов \alert{f}}
    \begin{columns}[c]

        % ======== левая колонка: код ========
        \column{0.4\textwidth}
        \centering
        \usebox{\codeLambdaExample}
        \vspace{0.4cm}

        Среда \alert{REPL} вызывает \mintinline{scheme}{f} с аргументом
        \mintinline{scheme}{x=4}, создавая первый фрейм стека — это запустит цепочку,
        где лямбда внутри \mintinline{scheme}{f} захватит \mintinline{scheme}{x} для
        последующих вызовов.

        \vfill

        % ======= правая колонка: схема ======
        \column{0.6\textwidth}
        \centering
        \begin{tikzpicture}[
            node distance=0.5cm,
            box/.style={
                draw=blue!70!black, fill=blue!5,
                rounded corners, thick, align=center,
                minimum width=3.8cm, minimum height=0.9cm,
                font=\footnotesize
            }
        ]
            \node[box] (repl) {\alert{REPL}};
            \node[box, below=of repl] (f1-1) {\mintinline{scheme}{ (f 4) }}
                edge [<-, thick, blue!70!black] (repl);
        \end{tikzpicture}
    \end{columns}
\end{frame}


\begin{frame}[fragile]{2.1 Замыкание лямбды: Шаг 2. Тело \alert{f}: подготовка вызова \alert{g}}
    \begin{columns}[c]

        % ======== левая колонка: код ========
        \column{0.4\textwidth}
        \centering
        \usebox{\codeLambdaExample}
        \vspace{0.4cm}

        После привязки \mintinline{scheme}{x=4} интерпретатор переходит к телу
        \mintinline{scheme}{f}

        \vfill

        % ======= правая колонка: схема ======
        \column{0.6\textwidth}
        \centering
        \begin{tikzpicture}[
            node distance=0.8cm,
            box/.style={
                draw=blue!70!black, fill=blue!5,
                rounded corners, thick, align=center,
                minimum width=3.8cm, minimum height=0.9cm,
                font=\footnotesize
            }
        ]
            \node[box] (repl) {\alert{REPL}};
            \node[box, below=of repl] (f1-1) {\mintinline{scheme}{ (f 4) }}
                edge [<-, thick, blue!70!black] (repl);
            \node[env-node=1, below=of repl, right=of f1-1] (stack1) { x $\mid$ 4 }
                edge [dashed, thick, blue!70!black] (f1-1);
            \node[box, below=of f1-1] (p1) {\mintinline{scheme}{ (+ (g (lambda (y) (* x y)) 3) }}
                edge [<-, thick, blue!70!black] (f1-1)
                edge [dashed, thick, blue!70!black] (stack1);
        \end{tikzpicture}
    \end{columns}
\end{frame}


\begin{frame}[fragile]{2.1 Замыкание лямбды: Шаг 3. Создание замыкания}
    \begin{columns}[c]

        % ======== левая колонка: код ========
        \column{0.4\textwidth}
        \centering
        \usebox{\codeLambdaExample}
        \vspace{0.4cm}

        Интерпретатор перерходит к вызову процедуры \mintinline{scheme}{g}

        \vfill

        % ======= правая колонка: схема ======
        \column{0.6\textwidth}
        \centering
        \begin{tikzpicture}[
            node distance=0.8cm,
            box/.style={
                draw=blue!70!black, fill=blue!5,
                rounded corners, thick, align=center,
                minimum width=3.8cm, minimum height=0.9cm,
                font=\footnotesize
            }
        ]
            \node[box] (repl) {\alert{REPL}};
            \node[box, below=of repl] (f1-1) {\mintinline{scheme}{ (f 4) }}
                edge [<-, thick, blue!70!black] (repl);
            \node[env-node=1, below=of repl, right=of f1-1] (stack1) { x $\mid$ 4 }
                edge [dashed, thick, blue!70!black] (f1-1);
            \node[box, below=of f1-1] (p1) {\mintinline{scheme}{ (+ (g (lambda (y) (* x y)) 3) }}
                edge [<-, thick, blue!70!black] (f1-1)
                edge [dashed, thick, blue!70!black] (stack1);
            \node[box, below=of p1] (g1) {\mintinline{scheme}{  (g (lambda (y) (* x y)) }}
                edge [<-, thick, blue!70!black] (p1);
        \end{tikzpicture}
    \end{columns}
\end{frame}


\begin{frame}[fragile]{2.1 Замыкание лямбды: Шаг 4. Вызов \alert{g}: новый фрейм}
    \begin{columns}[c]

        % ======== левая колонка: код ========
        \column{0.4\textwidth}
        \centering
        \usebox{\codeLambdaExample}
        \vspace{0.4cm}

        Замыкание передано в \mintinline{scheme}{g} как \mintinline{scheme}{h};
        создан второй фрейм стека с \mintinline{scheme}{h} — стек растет, но замыкание
        сохраняет ссылку на \mintinline{scheme}{x=4} из первого фрейма

        \vfill

        % ======= правая колонка: схема ======
        \column{0.6\textwidth}
        \centering
        \begin{tikzpicture}[
            node distance=0.7cm,
            box/.style={
                draw=blue!70!black, fill=blue!5,
                rounded corners, thick, align=center,
                minimum width=3.8cm, minimum height=0.9cm,
                font=\footnotesize
            }
        ]
            \node[box] (repl) {\alert{REPL}};
            \node[box, below=of repl] (f1-1) {\mintinline{scheme}{ (f 4) }}
                edge [<-, thick, blue!70!black] (repl);
            \node[env-node=1, below=of repl, right=1cm of f1-1] (stack1) { x $\mid$ 4 }
                edge [dashed, thick, blue!70!black] (f1-1);
            \node[box, below=of f1-1] (p1) {\mintinline{scheme}{ (+ (g (lambda (y) (* x y)) 3) }}
                edge [<-, thick, blue!70!black] (f1-1)
                edge [dashed, thick, blue!70!black] (stack1);
            \node[box, below=1.3cm of p1, xshift=-0.5cm] (g1) {\mintinline{scheme}{ (g (lambda (y) (* x y)) }}
                edge [<-, thick, blue!70!black, bend left=25] (p1);
            \node[env-node=1, right=of g1, above=0.7cm, xshift=0.6cm] (stack2) { h $\mid$ \mintinline{scheme}{ (lambda (y) (* x y)) } }
                edge [dashed, thick, blue!70!black, bend left=20] (g1)
                edge [dashed, thick, blue!70!black, bend right=30] (stack1);
        \end{tikzpicture}
    \end{columns}
\end{frame}


\begin{frame}[fragile]{2.1 Замыкание лямбды: Шаг 5. Тело \alert{g}: двойной вызов \alert{h}}
    \begin{columns}[c]

        % ======== левая колонка: код ========
        \column{0.4\textwidth}
        \centering
        \usebox{\codeLambdaExample}
        \vspace{0.4cm}

        Интерпретатор входит в тело \mintinline{scheme}{g}:
        \mintinline{scheme}{(h (h 7))} — фрейм \mintinline{scheme}{g} обновлен,
        готовясь к первому вызову \mintinline{scheme}{h} (замыкания),
        которое использует захваченное \mintinline{scheme}{x}.

        \vfill

        % ======= правая колонка: схема ======
        \column{0.6\textwidth}
        \centering
        \begin{tikzpicture}[
            node distance=0.5cm,
            box/.style={
                draw=blue!70!black, fill=blue!5,
                rounded corners, thick, align=center,
                minimum width=3.8cm, minimum height=0.6cm,
                font=\footnotesize
            }
        ]
            \node[box] (repl) {\alert{REPL}};
            \node[box, below=of repl] (f1-1) {\mintinline{scheme}{ (f 4) }}
                edge [<-, thick, blue!70!black] (repl);
            \node[env-node=1, below=of repl, right=1cm of f1-1] (stack1) { x $\mid$ 4 }
                edge [dashed, thick, blue!70!black] (f1-1);
            \node[box, below=of f1-1] (p1) {\mintinline{scheme}{ (+ (g (lambda (y) (* x y)) 3) }}
                edge [<-, thick, blue!70!black] (f1-1)
                edge [dashed, thick, blue!70!black] (stack1);
            \node[box, below=1.4cm of p1, xshift=-0.7cm] (g1) {\mintinline{scheme}{ (g (lambda (y) (* x y)) }}
                edge [<-, thick, blue!70!black, bend left=35] (p1);
            \node[env-node=1, right=of g1, above=0.65cm, xshift=0.6cm] (stack2) { h $\mid$ \mintinline{scheme}{ (lambda (y) (* x y)) } }
                edge [dashed, thick, blue!70!black, bend left=20] (g1)
                edge [dashed, thick, blue!70!black, bend right=45] (stack1);
            \node[box, below=of g1] (h1) {\mintinline{scheme}{ (h (h 7)) }}
                edge [dashed, thick, blue!70!black, bend right=30] (stack2)
                edge [<-, thick, blue!70!black] (g1);
        \end{tikzpicture}
    \end{columns}
\end{frame}


\begin{frame}[fragile]{2.1 Замыкание лямбды: Шаг 6. Первый вызов \mintinline{scheme}{h}: аргумент 7}
    \begin{columns}[c]

        % ======== левая колонка: код ========
        \column{0.4\textwidth}
        \centering
        \usebox{\codeLambdaExample}
        \vspace{0.4cm}

        Аргумент \mintinline{scheme}{7} вычислен; интерпретатор готовит применение
        \mintinline{scheme}{h} к \mintinline{scheme}{7} — это создаст третий фрейм
        для лямбды, с доступом к \mintinline{scheme}{x=4} через цепочку окружений.

        \vfill

        % ======= правая колонка: схема ======
        \column{0.6\textwidth}
        \centering
        \begin{tikzpicture}[
            node distance=0.4cm,
            box/.style={
                draw=blue!70!black, fill=blue!5,
                rounded corners, thick, align=center,
                minimum width=3.8cm, minimum height=0.55cm,
                font=\footnotesize
            }
        ]
            \node[box] (repl) {\alert{REPL}};
            \node[box, below=of repl] (f1-1) {\mintinline{scheme}{ (f 4) }}
                edge [<-, thick, blue!70!black] (repl);
            \node[env-node=1, below=of repl, right=1cm of f1-1] (stack1) { x $\mid$ 4 }
                edge [dashed, thick, blue!70!black] (f1-1);
            \node[box, below=of f1-1] (p1) {\mintinline{scheme}{ (+ (g (lambda (y) (* x y)) 3) }}
                edge [<-, thick, blue!70!black] (f1-1)
                edge [dashed, thick, blue!70!black] (stack1);
            \node[box, below=1.4cm of p1, xshift=-0.7cm] (g1) {\mintinline{scheme}{ (g (lambda (y) (* x y)) }}
                edge [<-, thick, blue!70!black, bend left=35] (p1);
            \node[env-node=1, right=of g1, above=0.65cm, xshift=0.6cm] (stack2) { h $\mid$ \mintinline{scheme}{ (lambda (y) (* x y)) } }
                edge [dashed, thick, blue!70!black, bend left=20] (g1)
                edge [dashed, thick, blue!70!black, bend right=45] (stack1);
            \node[box, below=of g1] (h1) {\mintinline{scheme}{ (h (h 7)) }}
                edge [dashed, thick, blue!70!black, bend right=30] (stack2)
                edge [<-, thick, blue!70!black] (g1);
            \node[box, below=of h1] (h2) {\mintinline{scheme}{ (h 7) }}
                edge [dashed, thick, blue!70!black, bend right=35] (stack2)
                edge [<-, thick, blue!70!black] (h1);
        \end{tikzpicture}
    \end{columns}
\end{frame}


\begin{frame}[fragile]{2.1 Замыкание лямбды: Шаг 7. Применение лямбды: фрейм \alert{y=7}}
    \begin{columns}[c]

        % ======== левая колонка: код ========
        \column{0.4\textwidth}
        \centering
        \usebox{\codeLambdaExample}
        \vspace{0.4cm}

        \mintinline{scheme}{h} (замыкание) применено к \mintinline{scheme}{7}: создан
        фрейм с \mintinline{scheme}{y=7}, который ссылается на \mintinline{scheme}{x=4}
        из внешнего окружения \mintinline{scheme}{f} — стек вырос до трех уровней.

        \vfill

        % ======= правая колонка: схема ======
        \column{0.6\textwidth}
        \centering
        \begin{tikzpicture}[
            node distance=0.4cm,
            box/.style={
                draw=blue!70!black, fill=blue!5,
                rounded corners, thick, align=center,
                minimum width=3.8cm, minimum height=0.55cm,
                font=\footnotesize
            }
        ]
            \node[box] (repl) {\alert{REPL}};
            \node[box, below=of repl] (f1-1) {\mintinline{scheme}{ (f 4) }}
                edge [<-, thick, blue!70!black] (repl);
            \node[env-node=1, below=of repl, right=1cm of f1-1] (stack1) { x $\mid$ 4 }
                edge [dashed, thick, blue!70!black] (f1-1);
            \node[box, below=of f1-1] (p1) {\mintinline{scheme}{ (+ (g (lambda (y) (* x y)) 3) }}
                edge [<-, thick, blue!70!black] (f1-1)
                edge [dashed, thick, blue!70!black] (stack1);
            \node[box, below=1.4cm of p1, xshift=-0.7cm] (g1) {\mintinline{scheme}{ (g (lambda (y) (* x y)) }}
                edge [<-, thick, blue!70!black, bend left=35] (p1);
            \node[env-node=1, right=of g1, above=0.65cm, xshift=0.6cm] (stack2) { h $\mid$ \mintinline{scheme}{ (lambda (y) (* x y)) } }
                edge [dashed, thick, blue!70!black, bend left=20] (g1)
                edge [dashed, thick, blue!70!black, bend right=45] (stack1);
            \node[box, below=of g1] (h1) {\mintinline{scheme}{ (h (h 7)) }}
                edge [dashed, thick, blue!70!black, bend right=30] (stack2)
                edge [<-, thick, blue!70!black] (g1);
            \node[box, below=of h1] (h2) {\mintinline{scheme}{ ((lambda (y) (* x y)) 7) }}
                edge [dashed, thick, blue!70!black, bend right=35] (stack2)
                edge [<-, thick, blue!70!black] (h1);
            \node[env-node=1, below=of repl, right=1cm of h2] (stack3) { y $\mid$ 7 }
                edge [dashed, thick, blue!70!black] (h2);
        \end{tikzpicture}
    \end{columns}
\end{frame}


\begin{frame}[fragile]{2.1 Замыкание лямбды: Шаг 8. Вычисление в лямбде: \alert{(* x y)}}
    \begin{columns}[c]

        % ======== левая колонка: код ========
        \column{0.4\textwidth}
        \centering
        \usebox{\codeLambdaExample}
        \vspace{0.4cm}

        В лямбде вычисляется \mintinline{scheme}{(* x y) = (* 4 7) = 28} — доступ к
        \mintinline{scheme}{x=4}; результат готов к возврату, без изменений в других фреймах.

        \vfill

        % ======= правая колонка: схема ======
        \column{0.6\textwidth}
        \centering
        \begin{tikzpicture}[
            node distance=0.4cm,
            box/.style={
                draw=blue!70!black, fill=blue!5,
                rounded corners, thick, align=center,
                minimum width=3.8cm, minimum height=0.55cm,
                font=\footnotesize
            }
        ]
            \node[box] (repl) {\alert{REPL}};
            \node[box, below=of repl] (f1-1) {\mintinline{scheme}{ (f 4) }}
                edge [<-, thick, blue!70!black] (repl);
            \node[env-node=1, below=of repl, right=1cm of f1-1] (stack1) { x $\mid$ 4 }
                edge [dashed, thick, blue!70!black] (f1-1);
            \node[box, below=of f1-1] (p1) {\mintinline{scheme}{ (+ (g (lambda (y) (* x y)) 3) }}
                edge [<-, thick, blue!70!black] (f1-1)
                edge [dashed, thick, blue!70!black] (stack1);
            \node[box, below=1.4cm of p1, xshift=-0.7cm] (g1) {\mintinline{scheme}{ (g (lambda (y) (* x y)) }}
                edge [<-, thick, blue!70!black, bend left=35] (p1);
            \node[env-node=1, right=of g1, above=0.65cm, xshift=0.6cm] (stack2) { h $\mid$ \mintinline{scheme}{ (lambda (y) (* x y)) } }
                edge [dashed, thick, blue!70!black, bend left=20] (g1)
                edge [dashed, thick, blue!70!black, bend right=45] (stack1);
            \node[box, below=of g1] (h1) {\mintinline{scheme}{ (h (h 7)) }}
                edge [dashed, thick, blue!70!black, bend right=30] (stack2)
                edge [<-, thick, blue!70!black] (g1);
            \node[box, below=of h1] (h2) {\mintinline{scheme}{ 28 }}
                edge [dashed, thick, blue!70!black, bend right=35] (stack2)
                edge [->, thick, blue!70!black, bend left=30] (h1)
                edge [<-, thick, blue!70!black] (h1);
            \node[env-node=1, below=of repl, right=1cm of h2] (stack3) { y $\mid$ 7 }
                edge [dashed, thick, blue!70!black] (h2);
        \end{tikzpicture}
    \end{columns}
\end{frame}


\begin{frame}[fragile]{2.1 Замыкание лямбды: Шаг 9. Возврат из первого \alert{h}: фрейм удален}
    \begin{columns}[c]

        % ======== левая колонка: код ========
        \column{0.4\textwidth}
        \centering
        \usebox{\codeLambdaExample}
        \vspace{0.4cm}

        Лямбда вернула \mintinline{scheme}{28}; фрейм \mintinline{scheme}{y=7}
        удален из стека — результат подставлен во второй вызов
        \mintinline{scheme}{(h 28)}, стек сократился до двух уровней.

        \vfill

        % ======= правая колонка: схема ======
        \column{0.6\textwidth}
        \centering
        \begin{tikzpicture}[
            node distance=0.5cm,
            box/.style={
                draw=blue!70!black, fill=blue!5,
                rounded corners, thick, align=center,
                minimum width=3.8cm, minimum height=0.6cm,
                font=\footnotesize
            }
        ]
            \node[box] (repl) {\alert{REPL}};
            \node[box, below=of repl] (f1-1) {\mintinline{scheme}{ (f 4) }}
                edge [<-, thick, blue!70!black] (repl);
            \node[env-node=1, below=of repl, right=1cm of f1-1] (stack1) { x $\mid$ 4 }
                edge [dashed, thick, blue!70!black] (f1-1);
            \node[box, below=of f1-1] (p1) {\mintinline{scheme}{ (+ (g (lambda (y) (* x y)) 3) }}
                edge [<-, thick, blue!70!black] (f1-1)
                edge [dashed, thick, blue!70!black] (stack1);
            \node[box, below=1.4cm of p1, xshift=-0.7cm] (g1) {\mintinline{scheme}{ (g (lambda (y) (* x y)) }}
                edge [<-, thick, blue!70!black, bend left=35] (p1);
            \node[env-node=1, right=of g1, above=0.65cm, xshift=0.6cm] (stack2) { h $\mid$ \mintinline{scheme}{ (lambda (y) (* x y)) } }
                edge [dashed, thick, blue!70!black, bend left=20] (g1)
                edge [dashed, thick, blue!70!black, bend right=45] (stack1);
            \node[box, below=of g1] (h1) {\mintinline{scheme}{ (h 28) }}
                edge [dashed, thick, blue!70!black, bend right=30] (stack2)
                edge [<-, thick, blue!70!black] (g1);
        \end{tikzpicture}
    \end{columns}
\end{frame}


\begin{frame}[fragile]{2.1 Замыкание лямбды: Шаг 10. Второй вызов \alert{h}: аргумент 28}
    \begin{columns}[c]

        % ======== левая колонка: код ========
        \column{0.4\textwidth}
        \centering
        \usebox{\codeLambdaExample}
        \vspace{0.4cm}

        Теперь \mintinline{scheme}{h} применяется к \mintinline{scheme}{28}:
        создан новый фрейм с \mintinline{scheme}{y=28}, снова ссылающийся на
        \mintinline{scheme}{x=4} — замыкание демонстрирует сохранение
        окружения во втором вызове.

        \vfill

        % ======= правая колонка: схема ======
        \column{0.6\textwidth}
        \centering
        \begin{tikzpicture}[
            node distance=0.5cm,
            box/.style={
                draw=blue!70!black, fill=blue!5,
                rounded corners, thick, align=center,
                minimum width=3.8cm, minimum height=0.6cm,
                font=\footnotesize
            }
        ]
            \node[box] (repl) {\alert{REPL}};
            \node[box, below=of repl] (f1-1) {\mintinline{scheme}{ (f 4) }}
                edge [<-, thick, blue!70!black] (repl);
            \node[env-node=1, below=of repl, right=1cm of f1-1] (stack1) { x $\mid$ 4 }
                edge [dashed, thick, blue!70!black] (f1-1);
            \node[box, below=of f1-1] (p1) {\mintinline{scheme}{ (+ (g (lambda (y) (* x y)) 3) }}
                edge [<-, thick, blue!70!black] (f1-1)
                edge [dashed, thick, blue!70!black] (stack1);
            \node[box, below=1.4cm of p1, xshift=-0.7cm] (g1) {\mintinline{scheme}{ (g (lambda (y) (* x y)) }}
                edge [<-, thick, blue!70!black, bend left=35] (p1);
            \node[env-node=1, right=of g1, above=0.65cm, xshift=0.6cm] (stack2) { h $\mid$ \mintinline{scheme}{ (lambda (y) (* x y)) } }
                edge [dashed, thick, blue!70!black, bend left=20] (g1)
                edge [dashed, thick, blue!70!black, bend right=45] (stack1);
            \node[box, below=of g1] (h1) {\mintinline{scheme}{ ((lambda (y) (* x y)) 28) }}
                edge [dashed, thick, blue!70!black, bend right=30] (stack2)
                edge [<-, thick, blue!70!black] (g1);
            \node[env-node=1, below=of g1, right=1cm of h1] (stack3) { y $\mid$ 28 }
                edge [dashed, thick, blue!70!black] (h1);
        \end{tikzpicture}
    \end{columns}
\end{frame}


\begin{frame}[fragile]{2.1 Замыкание лямбды: Шаг 11. Вычисление второго \alert{(* x y)}}
    \begin{columns}[c]

        % ======== левая колонка: код ========
        \column{0.4\textwidth}
        \centering
        \usebox{\codeLambdaExample}
        \vspace{0.4cm}

        В лямбде \mintinline{scheme}{(* x y) = (* 4 28) = 112} — \mintinline{scheme}{x=4}
        по-прежнему доступно; результат возвращен, фрейм \mintinline{scheme}{y=28} готов к удалению.

        \vfill

        % ======= правая колонка: схема ======
        \column{0.6\textwidth}
        \centering
        \begin{tikzpicture}[
            node distance=0.5cm,
            box/.style={
                draw=blue!70!black, fill=blue!5,
                rounded corners, thick, align=center,
                minimum width=3.8cm, minimum height=0.6cm,
                font=\footnotesize
            }
        ]
            \node[box] (repl) {\alert{REPL}};
            \node[box, below=of repl] (f1-1) {\mintinline{scheme}{ (f 4) }}
                edge [<-, thick, blue!70!black] (repl);
            \node[env-node=1, below=of repl, right=1cm of f1-1] (stack1) { x $\mid$ 4 }
                edge [dashed, thick, blue!70!black] (f1-1);
            \node[box, below=of f1-1] (p1) {\mintinline{scheme}{ (+ (g (lambda (y) (* x y)) 3) }}
                edge [<-, thick, blue!70!black] (f1-1)
                edge [dashed, thick, blue!70!black] (stack1);
            \node[box, below=1.4cm of p1, xshift=-0.7cm] (g1) {\mintinline{scheme}{ (g (lambda (y) (* x y)) }}
                edge [<-, thick, blue!70!black, bend left=35] (p1);
            \node[env-node=1, right=of g1, above=0.65cm, xshift=0.6cm] (stack2) { h $\mid$ \mintinline{scheme}{ (lambda (y) (* x y)) } }
                edge [dashed, thick, blue!70!black, bend left=20] (g1)
                edge [dashed, thick, blue!70!black, bend right=45] (stack1);
            \node[box, below=of g1] (h1) {\mintinline{scheme}{ 112 }}
                edge [dashed, thick, blue!70!black, bend right=30] (stack2)
                edge [dashed, thick, red!70!black, bend right=40] (stack1)
                edge [->, thick, blue!70!black, bend left=30] (g1)
                edge [<-, thick, blue!70!black] (g1);
            \node[env-node=1, below=of g1, right=1cm of h1] (stack3) { y $\mid$ 28 }
                edge [dashed, thick, blue!70!black] (h1);
        \end{tikzpicture}
    \end{columns}
\end{frame}


\begin{frame}[fragile]{2.1 Замыкание лямбды: Шаг 12. Завершение \alert{g}: возврат 112}
    \begin{columns}[c]

        % ======== левая колонка: код ========
        \column{0.4\textwidth}
        \centering
        \usebox{\codeLambdaExample}
        \vspace{0.4cm}

        \mintinline{scheme}{g} завершена, вернув \mintinline{scheme}{112}; фрейм
        \mintinline{scheme}{g} удален, результат передан в тело \mintinline{scheme}{f} —
        стек сократился до одного уровня, замыкание \mintinline{scheme}{h} больше не нужно.

        \vfill

        % ======= правая колонка: схема ======
        \column{0.6\textwidth}
        \centering
        \begin{tikzpicture}[
            node distance=0.7cm,
            box/.style={
                draw=blue!70!black, fill=blue!5,
                rounded corners, thick, align=center,
                minimum width=3.8cm, minimum height=0.9cm,
                font=\footnotesize
            }
        ]
            \node[box] (repl) {\alert{REPL}};
            \node[box, below=of repl] (f1-1) {\mintinline{scheme}{ (f 4) }}
                edge [<-, thick, blue!70!black] (repl);
            \node[env-node=1, below=of repl, right=1cm of f1-1] (stack1) { x $\mid$ 4 }
                edge [dashed, thick, blue!70!black] (f1-1);
            \node[box, below=of f1-1] (p1) {\mintinline{scheme}{ (+ (g (lambda (y) (* x y)) 3) }}
                edge [<-, thick, blue!70!black] (f1-1)
                edge [dashed, thick, blue!70!black] (stack1);
            \node[box, below=1.3cm of p1, xshift=-1cm] (g1) {112}
                edge [->, thick, blue!70!black, bend left=30] (p1)
                edge [<-, thick, blue!70!black] (p1);
            \node[env-node=1, right=of g1, above=0.7cm, xshift=1.8cm] (stack2) { h $\mid$ \mintinline{scheme}{ (lambda (y) (* x y)) } }
                edge [dashed, thick, blue!70!black, bend left=20] (g1)
                edge [dashed, thick, blue!70!black, bend right=30] (stack1);
        \end{tikzpicture}
    \end{columns}
\end{frame}


\begin{frame}[fragile]{2.1 Замыкание лямбды: Шаг 13. Финальное +3 в \alert{f}}
    \begin{columns}[c]

        % ======== левая колонка: код ========
        \column{0.4\textwidth}
        \centering
        \usebox{\codeLambdaExample}
        \vspace{0.4cm}

        112 подставлено в \mintinline{scheme}{(+ ... 3) = (+ 112 3)} —
        интерпретатор вычисляет финальное выражение в единственном фрейме
        \mintinline{scheme}{f}.

        \vfill

        % ======= правая колонка: схема ======
        \column{0.6\textwidth}
        \centering
        \begin{tikzpicture}[
            node distance=0.7cm,
            box/.style={
                draw=blue!70!black, fill=blue!5,
                rounded corners, thick, align=center,
                minimum width=3.8cm, minimum height=0.9cm,
                font=\footnotesize
            }
        ]
            \node[box] (repl) {\alert{REPL}};
            \node[box, below=of repl] (f1-1) {\mintinline{scheme}{ (f 4) }}
                edge [<-, thick, blue!70!black] (repl);
            \node[env-node=1, below=of repl, right=of f1-1] (stack1) { x $\mid$ 4 }
                edge [dashed, thick, blue!70!black] (f1-1);
            \node[box, below=of f1-1] (p1) {\mintinline{scheme}{ (+ 112 3) }}
                edge [<-, thick, blue!70!black] (f1-1)
                edge [dashed, thick, blue!70!black] (stack1);
        \end{tikzpicture}
    \end{columns}
\end{frame}


\begin{frame}[fragile]{2.1 Замыкание лямбды: Шаг 14. Результат \alert{f}: 115}
    \begin{columns}[c]

        % ======== левая колонка: код ========
        \column{0.4\textwidth}
        \centering
        \usebox{\codeLambdaExample}
        \vspace{0.4cm}

        \mintinline{scheme}{(+ 112 3) = 115}

        \vfill

        % ======= правая колонка: схема ======
        \column{0.6\textwidth}
        \centering
        \begin{tikzpicture}[
            node distance=0.7cm,
            box/.style={
                draw=blue!70!black, fill=blue!5,
                rounded corners, thick, align=center,
                minimum width=3.8cm, minimum height=0.9cm,
                font=\footnotesize
            }
        ]
            \node[box] (repl) {\alert{REPL}};
            \node[box, below=of repl] (f1-1) {\mintinline{scheme}{ (f 4) }}
                edge [<-, thick, blue!70!black] (repl);
            \node[env-node=1, below=of repl, right=of f1-1] (stack1) { x $\mid$ 4 }
                edge [dashed, thick, blue!70!black] (f1-1);
            \node[box, below=of f1-1] (p1) {\mintinline{scheme}{ 115 }}
                edge [<-, thick, blue!70!black] (f1-1)
                edge [->, thick, blue!70!black, bend left=30] (f1-1)
                edge [dashed, thick, blue!70!black] (stack1);
        \end{tikzpicture}
    \end{columns}
\end{frame}


\begin{frame}[fragile]{2.1 Замыкание лямбды: Шаг 15. Завершение вычисления}
    \begin{columns}[c]

        % ======== левая колонка: код ========
        \column{0.4\textwidth}
        \centering
        \usebox{\codeLambdaExample}
        \vspace{0.4cm}

        \mintinline{scheme}{115} доставлено в \alert{REPL}; стек полностью
        очищен — замыкание обеспечило доступ к \mintinline{scheme}{x=4} на
        всех уровнях, без потери контекста.

        \vfill

        % ======= правая колонка: схема ======
        \column{0.6\textwidth}
        \centering
        \begin{tikzpicture}[
            node distance=0.7cm,
            box/.style={
                draw=blue!70!black, fill=blue!5,
                rounded corners, thick, align=center,
                minimum width=3.8cm, minimum height=0.9cm,
                font=\footnotesize
            }
        ]
            \node[box] (repl) {\alert{REPL}};
            \node[box, below=of repl] (f1-1) {\mintinline{scheme}{ 115 }}
                edge [<-, thick, blue!70!black] (repl);
        \end{tikzpicture}
    \end{columns}
\end{frame}
