\begin{frame}{1.2 Введение в хвостовую рекурсию}
    \centering

    В отличие от обычной рекурсии (1.1), хвостовая рекурсия
    в \mintinline{scheme}{f2} использует аккумулятор
    \mintinline{scheme}{acc} для накопления результата,
    позволяя оптимизатору перезаписывать стек.
    На практике это превращает рекурсию в итерацию.

    \usebox{\codeFactorialTailRec}
\end{frame}


\begin{frame}[fragile]{1.2 Хвостовая рекурсия: Шаг 1. Начальный вызов}
    \begin{columns}[c]

        % ======== левая колонка: код ========
        \column{0.4\textwidth}
        \centering
        \usebox{\codeFactorialTailRec}
        \vspace{0.4cm}

        Среда \alert{REPL} запускает \mintinline{scheme}{f2} с
        аргументами \mintinline{scheme}{n = 3} и
        аккумулятором \mintinline{scheme}{acc = 1}

        \vfill

        % ======= правая колонка: схема ======
        \column{0.6\textwidth}
        \centering
        \begin{tikzpicture}[
            node distance=0.6cm,
            box/.style={
                draw=blue!70!black, fill=blue!5,
                rounded corners, thick, align=center,
                minimum width=3.8cm, minimum height=0.9cm,
                font=\footnotesize
            }
        ]
            \node[box] (repl) {\alert{REPL}};
            \node[box, below=of repl] (f2-1) {\mintinline{scheme}{(f2 3 1)}}
                edge [<-, thick, blue!70!black] (repl);
        \end{tikzpicture}
    \end{columns}
\end{frame}


\begin{frame}[fragile]{1.2 Хвостовая рекурсия: Шаг 2. Проверка базового случая}
    \begin{columns}[c]

        % ======== левая колонка: код ========
        \column{0.4\textwidth}
        \centering
        \usebox{\codeFactorialTailRec}
        \vspace{0.4cm}

        После начального вызова интерпретатор переходит к
        \mintinline{scheme}{if (= n 0)}, используя текущий фрейм
        с \mintinline{scheme}{n=3} и \mintinline{scheme}{acc=1}
        для проверки условия завершения.

        \vfill

        % ======= правая колонка: схема ======
        \column{0.6\textwidth}
        \centering
        \begin{tikzpicture}[
            node distance=0.6cm,
            box/.style={
                draw=blue!70!black, fill=blue!5,
                rounded corners, thick, align=center,
                minimum width=3.8cm, minimum height=0.9cm,
                font=\footnotesize
            }
        ]
            \node[box] (repl) {\alert{REPL}};
            \node[box, below=of repl] (f1-1) {\mintinline{scheme}{(f2 3 1)}}
                edge [<-, thick, blue!70!black] (repl);
            \node[env-node=2, right=of f1-1] (stack1) { n $\mid$ 3 \nodepart{two} acc $\mid$ 1 }
                edge [dashed, thick, blue!70!black] (f1-1);
            \node[box, below=of f1-1] (mul-1) {\mintinline{scheme}{ (if (= n 0)) }}
                edge [dashed, thick, blue!70!black] (stack1)
                edge [<-, thick, blue!70!black] (f1-1);
        \end{tikzpicture}
    \end{columns}
\end{frame}


\begin{frame}[fragile]{1.2 Хвостовая рекурсия: Шаг 3. Вычисление предиката}
    \begin{columns}[c]

        % ======== левая колонка: код ========
        \column{0.4\textwidth}
        \centering
        \usebox{\codeFactorialTailRec}
        \vspace{0.4cm}

        Внутри \mintinline{scheme}{if} вычисляется предикат
        \mintinline{scheme}{(= n 0)}.

        \vfill

        % ======= правая колонка: схема ======
        \column{0.6\textwidth}
        \centering
        \begin{tikzpicture}[
            node distance=0.6cm,
            box/.style={
                draw=blue!70!black, fill=blue!5,
                rounded corners, thick, align=center,
                minimum width=3.8cm, minimum height=0.9cm,
                font=\footnotesize
            }
        ]
            \node[box] (repl) {\alert{REPL}};
            \node[box, below=of repl] (f1-1) {\mintinline{scheme}{(f2 3 1)}}
                edge [<-, thick, blue!70!black] (repl);
            \node[env-node=2, right=of f1-1] (stack1) { n $\mid$ 3 \nodepart{two} acc $\mid$ 1 }
                edge [dashed, thick, blue!70!black] (f1-1);
            \node[box, below=of f1-1] (mul-1) {\mintinline{scheme}{ (if (= n 0)) }}
                edge [dashed, thick, blue!70!black] (stack1)
                edge [<-, thick, blue!70!black] (f1-1);
            \node[box, below=of mul-1] (eq1) {\mintinline{scheme}{ (= n 0) }}
                edge [dashed, thick, blue!70!black, bend right=30] (stack1)
                edge [<-, thick, blue!70!black] (mul-1);
        \end{tikzpicture}
    \end{columns}
\end{frame}


\begin{frame}[fragile]{1.2 Хвостовая рекурсия: Шаг 4. Переход к рекурсивной ветке}
    \begin{columns}[c]

        % ======== левая колонка: код ========
        \column{0.4\textwidth}
        \centering
        \usebox{\codeFactorialTailRec}
        \vspace{0.4cm}

        Предикат вернул \mintinline{scheme}{(= n 0)}
        вернул \mintinline{scheme}{#f}, поэтому
        \mintinline{scheme}{if} выбирает

        \vfill

        % ======= правая колонка: схема ======
        \column{0.6\textwidth}
        \centering
        \begin{tikzpicture}[
            node distance=0.6cm,
            box/.style={
                draw=blue!70!black, fill=blue!5,
                rounded corners, thick, align=center,
                minimum width=3.8cm, minimum height=0.9cm,
                font=\footnotesize
            }
        ]
            \node[box] (repl) {\alert{REPL}};
            \node[box, below=of repl] (f1-1) {\mintinline{scheme}{(f2 3 1)}}
                edge [<-, thick, blue!70!black] (repl);
            \node[env-node=2, right=of f1-1] (stack1) { n $\mid$ 3 \nodepart{two} acc $\mid$ 1 }
                edge [dashed, thick, blue!70!black] (f1-1);
            \node[box, below=of f1-1] (mul-1) {\mintinline{scheme}{ (if (= n 0)) }}
                edge [dashed, thick, blue!70!black] (stack1)
                edge [<-, thick, blue!70!black] (f1-1);
            \node[box, below=of mul-1] (eq1) {\mintinline{scheme}{ #f }}
                edge [dashed, thick, blue!70!black, bend right=30] (stack1)
                edge [<-, thick, blue!70!black] (mul-1)
                edge [->, thick, blue!70!black, bend left=30] (mul-1);
        \end{tikzpicture}
    \end{columns}
\end{frame}


\begin{frame}[fragile]{1.2 Хвостовая рекурсия: Шаг 5. Сведение к хвостовому вызову}
    \begin{columns}[c]

        % ======== левая колонка: код ========
        \column{0.4\textwidth}
        \centering
        \usebox{\codeFactorialTailRec}
        \vspace{0.4cm}

        \mintinline{scheme}{if} свертывается до
        \mintinline{scheme}{(f2 (- n 1) (* n acc))} — это
        чистый хвостовой вызов, готовый к оптимизации
        без сохранения состояния.

        \vfill

        % ======= правая колонка: схема ======
        \column{0.6\textwidth}
        \centering
        \begin{tikzpicture}[
            node distance=0.6cm,
            box/.style={
                draw=blue!70!black, fill=blue!5,
                rounded corners, thick, align=center,
                minimum width=3.8cm, minimum height=0.9cm,
                font=\footnotesize
            }
        ]
            \node[box] (repl) {\alert{REPL}};
            \node[box, below=of repl] (f1-1) {\mintinline{scheme}{(f2 3 1)}}
                edge [<-, thick, blue!70!black] (repl);
            \node[env-node=2, right=of f1-1] (stack1) { n $\mid$ 3 \nodepart{two} acc $\mid$ 1 }
                edge [dashed, thick, blue!70!black] (f1-1);
            \node[box, below=of f1-1] (mul-1) {\mintinline{scheme}{ (if #f) }}
                edge [dashed, thick, blue!70!black] (stack1)
                edge [<-, thick, blue!70!black] (f1-1);
        \end{tikzpicture}
    \end{columns}
\end{frame}


\begin{frame}[fragile]{1.2 Хвостовая рекурсия: Шаг 6. Подготовка аргументов}
    \begin{columns}[c]

        % ======== левая колонка: код ========
        \column{0.4\textwidth}
        \centering
        \usebox{\codeFactorialTailRec}
        \vspace{0.4cm}

        Форма if полностью свелась к телу рекурсивного
        вызова — выражению \mintinline{scheme}{(f2 (- n 1) (* n acc))}.

        \vfill

        % ======= правая колонка: схема ======
        \column{0.6\textwidth}
        \centering
        \begin{tikzpicture}[
            node distance=0.6cm,
            box/.style={
                draw=blue!70!black, fill=blue!5,
                rounded corners, thick, align=center,
                minimum width=3.8cm, minimum height=0.9cm,
                font=\footnotesize
            }
        ]
            \node[box] (repl) {\alert{REPL}};
            \node[box, below=of repl] (f1-1) {\mintinline{scheme}{(f2 3 1)}}
                edge [<-, thick, blue!70!black] (repl);
            \node[env-node=2, right=of f1-1] (stack1) { n $\mid$ 3 \nodepart{two} acc $\mid$ 1 }
                edge [dashed, thick, blue!70!black] (f1-1);
            \node[box, below=of f1-1] (mul-1) {\mintinline{scheme}{ (f2 (- n 1) (* n acc)) }}
                edge [dashed, thick, blue!70!black] (stack1)
                edge [<-, thick, blue!70!black] (f1-1);
        \end{tikzpicture}
    \end{columns}
\end{frame}


\begin{frame}[fragile]{1.2 Хвостовая рекурсия: Шаг 7. Вычисление аргументов перед хвостовым вызовом}
    \begin{columns}[c]

        % ======== левая колонка: код ========
        \column{0.4\textwidth}
        \centering
        \usebox{\codeFactorialTailRec}
        \vspace{0.4cm}

        Интерпретатор начал вычислять аргументы рекурсивного вызова
        \mintinline{scheme}{(f2 (- n 1) (* n acc))}.

        \vfill

        % ======= правая колонка: схема ======
        \column{0.6\textwidth}
        \centering
        \begin{tikzpicture}[
            node distance=0.6cm,
            box/.style={
                draw=blue!70!black, fill=blue!5,
                rounded corners, thick, align=center,
                minimum width=3.8cm, minimum height=0.9cm,
                font=\footnotesize
            }
        ]
            \node[box] (repl) {\alert{REPL}};
            \node[box, below=of repl] (f1-1) {\mintinline{scheme}{(f2 3 1)}}
                edge [<-, thick, blue!70!black] (repl);
            \node[env-node=2, right=1.5cm of f1-1] (stack1) { n $\mid$ 3 \nodepart{two} acc $\mid$ 1 }
                edge [dashed, thick, blue!70!black] (f1-1);
            \node[box, below=of f1-1] (mul-1) {\mintinline{scheme}{ (f2 (- n 1) (* n acc)) }}
                edge [dashed, thick, blue!70!black] (stack1)
                edge [<-, thick, blue!70!black] (f1-1);
            \node[box, below=1.2cm of mul-1, xshift=-2cm] (sub-1) {\mintinline{scheme}{ (- n 1) }}
                edge [dashed, thick, blue!70!black, bend right=22] (stack1)
                edge [<-, thick, blue!70!black] (mul-1);
            \node[box, below=1.2cm of mul-1, right=of sub-1] (mul-2) {\mintinline{scheme}{ (* n acc) }}
                edge [dashed, thick, blue!70!black] (stack1)
                edge [<-, thick, blue!70!black] (mul-1);
        \end{tikzpicture}
    \end{columns}
\end{frame}


\begin{frame}[fragile]{1.2 Хвостовая рекурсия: Шаг 8. Аргументы хвостового вызова полностью вычислены}
    \begin{columns}[c]

        % ======== левая колонка: код ========
        \column{0.4\textwidth}
        \centering
        \usebox{\codeFactorialTailRec}
        \vspace{0.4cm}

        Оба аргумента рекурсивного вызова уже сведены к своим значениям

        \vfill

        % ======= правая колонка: схема ======
        \column{0.6\textwidth}
        \centering
        \begin{tikzpicture}[
            node distance=0.6cm,
            box/.style={
                draw=blue!70!black, fill=blue!5,
                rounded corners, thick, align=center,
                minimum width=3.8cm, minimum height=0.9cm,
                font=\footnotesize
            }
        ]
            \node[box] (repl) {\alert{REPL}};
            \node[box, below=of repl] (f1-1) {\mintinline{scheme}{(f2 3 1)}}
                edge [<-, thick, blue!70!black] (repl);
            \node[env-node=2, right=1.5cm of f1-1] (stack1) { n $\mid$ 3 \nodepart{two} acc $\mid$ 1 }
                edge [dashed, thick, blue!70!black] (f1-1);
            \node[box, below=of f1-1] (mul-1) {\mintinline{scheme}{ (f2 (- n 1) (* n acc)) }}
                edge [dashed, thick, blue!70!black] (stack1)
                edge [<-, thick, blue!70!black] (f1-1);
            \node[box, below=1.2cm of mul-1, xshift=-2cm] (sub-1) {\mintinline{scheme}{ (2) }}
                edge [dashed, thick, blue!70!black, bend right=22] (stack1)
                edge [<-, thick, blue!70!black] (mul-1);
            \node[box, below=1.2cm of mul-1, right=of sub-1] (mul-2) {\mintinline{scheme}{ (3) }}
                edge [dashed, thick, blue!70!black] (stack1)
                edge [<-, thick, blue!70!black] (mul-1);
        \end{tikzpicture}
    \end{columns}
\end{frame}


\begin{frame}[fragile]{1.2 Хвостовая рекурсия: Шаг 9. Готовность к хвостовой оптимизации}
    \begin{columns}[c]

        % ======== левая колонка: код ========
        \column{0.4\textwidth}
        \centering
        \usebox{\codeFactorialTailRec}
        \vspace{0.4cm}

        Интерпретатор полностью подготовил хвостовой
        рекурсивный вызов \mintinline{scheme}{(f2 2 3)}.

        \vfill

        % ======= правая колонка: схема ======
        \column{0.6\textwidth}
        \centering
        \begin{tikzpicture}[
            node distance=0.6cm,
            box/.style={
                draw=blue!70!black, fill=blue!5,
                rounded corners, thick, align=center,
                minimum width=3.8cm, minimum height=0.9cm,
                font=\footnotesize
            }
        ]
            \node[box] (repl) {\alert{REPL}};
            \node[box, below=of repl] (f1-1) {\mintinline{scheme}{(f2 3 1)}}
                edge [<-, thick, blue!70!black] (repl);
            \node[env-node=2, right=1.5cm of f1-1] (stack1) { n $\mid$ 3 \nodepart{two} acc $\mid$ 1 }
                edge [dashed, thick, blue!70!black] (f1-1);
            \node[box, below=of f1-1] (mul-1) {\mintinline{scheme}{ (f2 2 3) }}
                edge [dashed, thick, blue!70!black] (stack1)
                edge [<-, thick, blue!70!black] (f1-1);
        \end{tikzpicture}
    \end{columns}
\end{frame}

\begin{frame}[fragile]{1.2 Хвостовая рекурсия: Шаг 10. Новый фрейм не создаётся}
    \begin{columns}[c]

        % ======== левая колонка: код ========
        \column{0.4\textwidth}
        \centering
        \usebox{\codeFactorialTailRec}
        \vspace{0.4cm}

        После оптимизации фрейм обновлен (
        \mintinline{scheme}{n=2}, \mintinline{scheme}{acc=3}),
        и цикл повторяется в том же фрейме — это итерация, а не рекурсия

        \vfill

        % ======= правая колонка: схема ======
        \column{0.6\textwidth}
        \centering
        \begin{tikzpicture}[
            node distance=0.6cm,
            box/.style={
                draw=blue!70!black, fill=blue!5,
                rounded corners, thick, align=center,
                minimum width=3.8cm, minimum height=0.9cm,
                font=\footnotesize
            }
        ]
            \node[box] (repl) {\alert{REPL}};
            \node[box, below=of repl] (f1-1) {\mintinline{scheme}{(f2 3 1)}}
                edge [<-, thick, blue!70!black] (repl);
            \node[env-node=2, right=1.5cm of f1-1] (stack1) { n $\mid$ 3 \nodepart{two} acc $\mid$ 1 };
            \node[box, below=of f1-1] (mul-1) {\mintinline{scheme}{ (f2 2 3) }}
                edge [dashed, thick, blue!70!black] (stack1)
                edge [<-, thick, blue!70!black] (f1-1);
        \end{tikzpicture}
    \end{columns}
\end{frame}


\begin{frame}[fragile]{1.2 Хвостовая рекурсия: Шаг 11. Итерация: обновление фрейма}
    \begin{columns}[c]

        % ======== левая колонка: код ========
        \column{0.4\textwidth}
        \centering
        \usebox{\codeFactorialTailRec}
        \vspace{0.4cm}

        Перменные во фрейме перезаписываются новыми значениями
        \mintinline{scheme}{n = 2} и \mintinline{scheme}{acc = 2}

        \vfill

        % ======= правая колонка: схема ======
        \column{0.6\textwidth}
        \centering
        \begin{tikzpicture}[
            node distance=0.6cm,
            box/.style={
                draw=blue!70!black, fill=blue!5,
                rounded corners, thick, align=center,
                minimum width=3.8cm, minimum height=0.9cm,
                font=\footnotesize
            }
        ]
            \node[box] (repl) {\alert{REPL}};
            \node[box, below=of repl] (f1-1) {\mintinline{scheme}{(f2 3 1)}}
                edge [<-, thick, blue!70!black] (repl);
            \node[env-node=2, right=1.5cm of f1-1] (stack1) { n $\mid$ 2 \nodepart{two} acc $\mid$ 2 };
            \node[box, below=of f1-1] (mul-1) {\mintinline{scheme}{ (f2 2 3) }}
                edge [dashed, thick, blue!70!black] (stack1)
                edge [<-, thick, blue!70!black] (f1-1);
        \end{tikzpicture}
    \end{columns}
\end{frame}


\begin{frame}[fragile]{1.2 Хвостовая рекурсия: Шаг 12. Ещё одна хвостовая оптимизация}
    \begin{columns}[c]

        % ======== левая колонка: код ========
        \column{0.4\textwidth}
        \centering
        \usebox{\codeFactorialTailRec}
        \vspace{0.4cm}

        После очередной итерации внутри того же
        единственного фрейма стека значения снова обновились

        \vfill

        % ======= правая колонка: схема ======
        \column{0.6\textwidth}
        \centering
        \begin{tikzpicture}[
            node distance=0.6cm,
            box/.style={
                draw=blue!70!black, fill=blue!5,
                rounded corners, thick, align=center,
                minimum width=3.8cm, minimum height=0.9cm,
                font=\footnotesize
            }
        ]
            \node[box] (repl) {\alert{REPL}};
            \node[box, below=of repl] (f1-1) {\mintinline{scheme}{(f2 3 1)}}
                edge [<-, thick, blue!70!black] (repl);
            \node[env-node=2, right=1.5cm of f1-1] (stack1) { n $\mid$ 2 \nodepart{two} acc $\mid$ 2 };
            \node[box, below=of f1-1] (mul-1) {\mintinline{scheme}{ (f2 2 3) }}
                edge [dashed, thick, blue!70!black] (stack1)
                edge [<-, thick, blue!70!black] (f1-1);
            \node[box, below=of mul-1] (mul-2) {\mintinline{scheme}{ (f2 1 6) }}
                edge [dashed, thick, blue!70!black, bend right=30] (stack1)
                edge [<-, thick, blue!70!black] (mul-1);
        \end{tikzpicture}
    \end{columns}
\end{frame}


\begin{frame}[fragile]{1.2 Хвостовая рекурсия: Шаг 13. Третья итерация}
    \begin{columns}[c]

        % ======== левая колонка: код ========
        \column{0.4\textwidth}
        \centering
        \usebox{\codeFactorialTailRec}
        \vspace{0.4cm}

        Очередная хвостовая оптимизация выполнена

        \vfill

        % ======= правая колонка: схема ======
        \column{0.6\textwidth}
        \centering
        \begin{tikzpicture}[
            node distance=0.6cm,
            box/.style={
                draw=blue!70!black, fill=blue!5,
                rounded corners, thick, align=center,
                minimum width=3.8cm, minimum height=0.9cm,
                font=\footnotesize
            }
        ]
            \node[box] (repl) {\alert{REPL}};
            \node[box, below=of repl] (f1-1) {\mintinline{scheme}{(f2 3 1)}}
                edge [<-, thick, blue!70!black] (repl);
            \node[env-node=2, right=1.5cm of f1-1] (stack1) { n $\mid$ 1 \nodepart{two} acc $\mid$ 6 };
            \node[box, below=of f1-1] (mul-1) {\mintinline{scheme}{ (f2 2 3) }}
                edge [<-, thick, blue!70!black] (f1-1);
            \node[box, below=of mul-1] (mul-2) {\mintinline{scheme}{ (f2 1 6) }}
                edge [dashed, thick, blue!70!black, bend right=30] (stack1)
                edge [<-, thick, blue!70!black] (mul-1);
        \end{tikzpicture}
    \end{columns}
\end{frame}


\begin{frame}[fragile]{1.2 Хвостовая рекурсия: Шаг 14. Финальная хвостовая оптимизация}
    \begin{columns}[c]

        % ======== левая колонка: код ========
        \column{0.4\textwidth}
        \centering
        \usebox{\codeFactorialTailRec}
        \vspace{0.4cm}

        Ещё одна (уже четвёртая) хвостовая оптимизация выполнена

        \vfill

        % ======= правая колонка: схема ======
        \column{0.6\textwidth}
        \centering
        \begin{tikzpicture}[
            node distance=0.5cm,
            box/.style={
                draw=blue!70!black, fill=blue!5,
                rounded corners, thick, align=center,
                minimum width=3.8cm, minimum height=0.9cm,
                font=\footnotesize
            }
        ]
            \node[box] (repl) {\alert{REPL}};
            \node[box, below=of repl] (f1-1) {\mintinline{scheme}{(f2 3 1)}}
                edge [<-, thick, blue!70!black] (repl);
            \node[env-node=2, right=1.5cm of f1-1] (stack1) { n $\mid$ 1 \nodepart{two} acc $\mid$ 6 };
            \node[box, below=of f1-1] (mul-1) {\mintinline{scheme}{ (f2 2 3) }}
                edge [<-, thick, blue!70!black] (f1-1);
            \node[box, below=of mul-1] (mul-2) {\mintinline{scheme}{ (f2 1 6) }}
                edge [dashed, thick, blue!70!black, bend right=30] (stack1)
                edge [<-, thick, blue!70!black] (mul-1);
            \node[box, below=of mul-2] (mul-3) {\mintinline{scheme}{ (f2 0 6) }}
                edge [dashed, thick, blue!70!black, bend right=30] (stack1)
                edge [<-, thick, blue!70!black] (mul-2);
        \end{tikzpicture}
    \end{columns}
\end{frame}


\begin{frame}[fragile]{1.2 Хвостовая рекурсия: Шаг 15. Итерация заканчивается}
    \begin{columns}[c]

        % ======== левая колонка: код ========
        \column{0.4\textwidth}
        \centering
        \usebox{\codeFactorialTailRec}
        \vspace{0.4cm}

        После последней хвостовой оптимизации в единственном
        фрейме стека теперь \mintinline{scheme}{n = 0},
        \mintinline{scheme}{acc = 6}.

        \vfill

        % ======= правая колонка: схема ======
        \column{0.6\textwidth}
        \centering
        \begin{tikzpicture}[
            node distance=0.5cm,
            box/.style={
                draw=blue!70!black, fill=blue!5,
                rounded corners, thick, align=center,
                minimum width=3.8cm, minimum height=0.9cm,
                font=\footnotesize
            }
        ]
            \node[box] (repl) {\alert{REPL}};
            \node[box, below=of repl] (f1-1) {\mintinline{scheme}{(f2 3 1)}}
                edge [<-, thick, blue!70!black] (repl);
            \node[env-node=2, right=1.5cm of f1-1] (stack1) { n $\mid$ 0 \nodepart{two} acc $\mid$ 6 };
            \node[box, below=of f1-1] (mul-1) {\mintinline{scheme}{ (f2 2 3) }}
                edge [<-, thick, blue!70!black] (f1-1);
            \node[box, below=of mul-1] (mul-2) {\mintinline{scheme}{ (f2 1 6) }}
                edge [<-, thick, blue!70!black] (mul-1);
            \node[box, below=of mul-2] (mul-3) {\mintinline{scheme}{ (f2 0 6) }}
                edge [dashed, thick, blue!70!black, bend right=30] (stack1)
                edge [<-, thick, blue!70!black] (mul-2);
        \end{tikzpicture}
    \end{columns}
\end{frame}


\begin{frame}[fragile]{1.2 Хвостовая рекурсия: Шаг 16. Подготовка к возврату результата}
    \begin{columns}[c]

        % ======== левая колонка: код ========
        \column{0.4\textwidth}
        \centering
        \usebox{\codeFactorialTailRec}
        \vspace{0.4cm}

        В единственном фрейме стека теперь \mintinline{scheme}{n = 0}.

        \vfill

        % ======= правая колонка: схема ======
        \column{0.6\textwidth}
        \centering
        \begin{tikzpicture}[
            node distance=0.5cm,
            box/.style={
                draw=blue!70!black, fill=blue!5,
                rounded corners, thick, align=center,
                minimum width=3.8cm, minimum height=0.9cm,
                font=\footnotesize
            }
        ]
            \node[box] (repl) {\alert{REPL}};
            \node[box, below=of repl] (f1-1) {\mintinline{scheme}{(f2 3 1)}}
                edge [<-, thick, blue!70!black] (repl);
            \node[env-node=2, right=1.5cm of f1-1] (stack1) { n $\mid$ 0 \nodepart{two} acc $\mid$ 6 };
            \node[box, below=of f1-1] (mul-1) {\mintinline{scheme}{ (f2 2 3) }}
                edge [<-, thick, blue!70!black] (f1-1);
            \node[box, below=of mul-1] (mul-2) {\mintinline{scheme}{ (f2 1 6) }}
                edge [<-, thick, blue!70!black] (mul-1);
            \node[box, below=of mul-2] (mul-3) {\mintinline{scheme}{ (f2 0 6) }}
                edge [<-, thick, blue!70!black] (mul-2)
                edge [dashed, thick, blue!70!black, bend right=30] (stack1);
        \end{tikzpicture}
    \end{columns}
\end{frame}


\begin{frame}[fragile]{1.2 Хвостовая рекурсия: Шаг 17. Возврат аккумулятора}
    \begin{columns}[c]

        % ======== левая колонка: код ========
        \column{0.4\textwidth}
        \centering
        \usebox{\codeFactorialTailRec}
        \vspace{0.4cm}

        Условие \mintinline{scheme}{(= n 0)} истинно,
        поэтому тело процедуры свелось к простому
        возврату значения переменной \mintinline{scheme}{acc}.

        \vfill

        % ======= правая колонка: схема ======
        \column{0.6\textwidth}
        \centering
        \begin{tikzpicture}[
            node distance=0.5cm,
            box/.style={
                draw=blue!70!black, fill=blue!5,
                rounded corners, thick, align=center,
                minimum width=3.8cm, minimum height=0.9cm,
                font=\footnotesize
            }
        ]
            \node[box] (repl) {\alert{REPL}};
            \node[box, below=of repl] (f1-1) {\mintinline{scheme}{(f2 3 1)}}
                edge [<-, thick, blue!70!black] (repl);
            \node[env-node=2, right=1.5cm of f1-1] (stack1) { n $\mid$ 0 \nodepart{two} acc $\mid$ 6 };
            \node[box, below=of f1-1] (mul-1) {\mintinline{scheme}{ (f2 2 3) }}
                edge [<-, thick, blue!70!black] (f1-1);
            \node[box, below=of mul-1] (mul-2) {\mintinline{scheme}{ (f2 1 6) }}
                edge [<-, thick, blue!70!black] (mul-1);
            \node[box, below=of mul-2] (mul-3) {\mintinline{scheme}{ (acc) }}
                edge [<-, thick, blue!70!black] (mul-2)
                edge [dashed, thick, blue!70!black, bend right=30] (stack1);
        \end{tikzpicture}
    \end{columns}
\end{frame}


\begin{frame}[fragile]{1.2 Хвостовая рекурсия: Шаг 18. Подъём результата}
    \begin{columns}[c]

        % ======== левая колонка: код ========
        \column{0.4\textwidth}
        \centering
        \usebox{\codeFactorialTailRec}
        \vspace{0.4cm}

        \mintinline{scheme}{6} поднимается через перезаписанные
        уровни (\mintinline{scheme}{n} $\to 1 \to 2 \to 3$),
        но без реального стека

        \vfill

        % ======= правая колонка: схема ======
        \column{0.6\textwidth}
        \centering
        \begin{tikzpicture}[
            node distance=0.5cm,
            box/.style={
                draw=blue!70!black, fill=blue!5,
                rounded corners, thick, align=center,
                minimum width=3.8cm, minimum height=0.9cm,
                font=\footnotesize
            }
        ]
            \node[box] (repl) {\alert{REPL}};
            \node[box, below=of repl] (f1-1) {\mintinline{scheme}{(f2 3 1)}}
                edge [<-, thick, blue!70!black] (repl);
            \node[env-node=2, right=1.5cm of f1-1] (stack1) { n $\mid$ 0 \nodepart{two} acc $\mid$ 6 };
            \node[box, below=of f1-1] (mul-1) {\mintinline{scheme}{ (f2 2 3) }}
                edge [<-, thick, blue!70!black] (f1-1);
            \node[box, below=of mul-1] (mul-2) {\mintinline{scheme}{ (f2 1 6) }}
                edge [<-, thick, blue!70!black] (mul-1);
            \node[box, below=of mul-2] (mul-3) {\mintinline{scheme}{ 6 }}
                edge [<-, thick, blue!70!black] (mul-2)
                edge [->, thick, blue!70!black, bend left=30] (mul-2)
                edge [dashed, thick, blue!70!black, bend right=30] (stack1);
        \end{tikzpicture}
    \end{columns}
\end{frame}


\begin{frame}[fragile]{1.2 Хвостовая рекурсия: Шаг 19. Подъём результата}
    \begin{columns}[c]

        % ======== левая колонка: код ========
        \column{0.4\textwidth}
        \centering
        \usebox{\codeFactorialTailRec}
        \vspace{0.4cm}

        За...

        \vfill

        % ======= правая колонка: схема ======
        \column{0.6\textwidth}
        \centering
        \begin{tikzpicture}[
            node distance=0.5cm,
            box/.style={
                draw=blue!70!black, fill=blue!5,
                rounded corners, thick, align=center,
                minimum width=3.8cm, minimum height=0.9cm,
                font=\footnotesize
            }
        ]
            \node[box] (repl) {\alert{REPL}};
            \node[box, below=of repl] (f1-1) {\mintinline{scheme}{(f2 3 1)}}
            edge [<-, thick, blue!70!black] (repl);
            \node[env-node=2, right=1.5cm of f1-1] (stack1) { n $\mid$ 0 \nodepart{two} acc $\mid$ 6 };
            \node[box, below=of f1-1] (mul-1) {\mintinline{scheme}{ (f2 2 3) }}
            edge [<-, thick, blue!70!black] (f1-1);
            \node[box, below=of mul-1] (mul-2) {\mintinline{scheme}{ 6 }}
            edge [<-, thick, blue!70!black] (mul-1)
            edge [->, thick, blue!70!black, bend left=30] (mul-1);
        \end{tikzpicture}

    \end{columns}
\end{frame}



\begin{frame}[fragile]{1.2 Хвостовая рекурсия: Шаг 20. Финальный результат в REPL — рекурсия завершена без роста стека}
    \begin{columns}[c]

        % ======== левая колонка: код ========
        \column{0.4\textwidth}
        \centering
        \usebox{\codeFactorialTailRec}
        \vspace{0.4cm}

        Значение 6 поднялось по всем «виртуальным» уровням.

        \vfill

        % ======= правая колонка: схема ======
        \column{0.6\textwidth}
        \centering
        \begin{tikzpicture}[
            node distance=0.5cm,
            box/.style={
                draw=blue!70!black, fill=blue!5,
                rounded corners, thick, align=center,
                minimum width=3.8cm, minimum height=0.9cm,
                font=\footnotesize
            }
        ]
            \node[box] (repl) {\alert{REPL}};
            \node[box, below=of repl] (f1-1) {\mintinline{scheme}{(f2 3 1)}}
            edge [<-, thick, blue!70!black] (repl);
            \node[env-node=2, right=1.5cm of f1-1] (stack1) { n $\mid$ 0 \nodepart{two} acc $\mid$ 6 };
            \node[box, below=of f1-1] (mul-1) {\mintinline{scheme}{ 6 }}
            edge [<-, thick, blue!70!black] (f1-1)
            edge [->, thick, blue!70!black, bend left=30] (f1-1);
        \end{tikzpicture}

    \end{columns}
\end{frame}

\begin{frame}[fragile]{1.2 Хвостовая рекурсия: Шаг 21. Всё завершено — результат в REPL}
    \begin{columns}[c]

        % ======== левая колонка: код ========
        \column{0.4\textwidth}
        \centering
        \usebox{\codeFactorialTailRec}
        \vspace{0.4cm}

        Значение 6 окончательно возвращено в \alert{REPL}.

        \vfill

        % ======= правая колонка: схема ======
        \column{0.6\textwidth}
        \centering
        \begin{tikzpicture}[
            node distance=0.5cm,
            box/.style={
                draw=blue!70!black, fill=blue!5,
                rounded corners, thick, align=center,
                minimum width=3.8cm, minimum height=0.9cm,
                font=\footnotesize
            }
        ]
            \node[box] (repl) {\alert{REPL}};
            \node[box, below=of repl] (f1-1) {\mintinline{scheme}{ 6 }}
            edge [<-, thick, blue!70!black] (repl);
        \end{tikzpicture}

    \end{columns}
\end{frame}