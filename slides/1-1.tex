\begin{frame}[fragile]{Глава 1. Оптимизация хвостовой рекурсии}
\centering

Рекурсивный вызов является \textbf{хвостовым}, если он — \alert{последнее действие} в выполняемой функции, и после возврата из рекурсивного вызова функции \alert{ничего больше делать не нужно}.

\end{frame}


\begin{frame}{1.1 Обычная рекурсия}
\centering
Начнём анализ рекурсии в Scheme с процедуры \mintinline{scheme}{f1}.
На её примере мы увидим, как \alert{REPL} вызывает
функцию,как формируются новые фреймы в стеке и как
происходит возврат результатов при
рекурсивных вызовах.

\usebox{\codeFactorialSimple}
\end{frame}

\begin{frame}[fragile]{1.1 Обычная рекурсия: Шаг 1. Начальный вызов}
\begin{columns}[c]

  % ======== левая колонка: код ========
  \column{0.4\textwidth}
  \centering
  \usebox{\codeFactorialSimple}
  \vspace{0.4cm}

  Среда \alert{REPL} вызывает процедуру \mintinline{scheme}{f1} с аргументом \mintinline{scheme}{3}

  \vfill

  % ======= правая колонка: схема ======
  \column{0.6\textwidth}
  \centering
  \begin{tikzpicture}[
      node distance=1.1cm,
      box/.style={
        draw=blue!70!black, fill=blue!5,
        rounded corners, thick, align=center,
        minimum width=3.8cm, minimum height=0.9cm,
        font=\footnotesize
      }
    ]
    \node[box] (repl) {\alert{REPL}};
    \node[box, below=of repl] (f1-1) {\mintinline{scheme}{(f1 n)}}
      edge [<-, thick, blue!70!black] (repl);
    \node[env-node=1, right=of f1-1] (stack1) { n $\mid$ 3 }
      edge [dashed, thick, blue!70!black] (f1-1);
    \end{tikzpicture}

\end{columns}
\end{frame}


\begin{frame}[fragile]{1.1 Обычная рекурсия: Шаг 2. Вызов \mintinline{scheme}{if}}
\begin{columns}[c]

  % ======== левая колонка: код ========
  \column{0.4\textwidth}
  \centering
  \usebox{\codeFactorialSimple}
  \vspace{0.4cm}

  Интерпретатор вызывает особую форму \mintinline{scheme}{if}

  \vfill

  % ======= правая колонка: схема ======
  \column{0.6\textwidth}
  \centering
  \begin{tikzpicture}[
      node distance=0.8cm,
      box/.style={
        draw=blue!70!black, fill=blue!5,
        rounded corners, thick, align=center,
        minimum width=3.8cm, minimum height=0.9cm,
        font=\footnotesize
      }
    ]
    \node[box] (repl) {\alert{REPL}};
    \node[box, below=of repl] (f1-1) {\mintinline{scheme}{(f1 3)}}
      edge [<-, thick, blue!70!black] (repl);
    \node[env-node=1, right=of f1-1] (stack1) { n $\mid$ 3 }
      edge [dashed, thick, blue!70!black] (f1-1);

    \node[box, below=of f1-1] (if-1) {\mintinline{scheme}{(if (= n 0))}}
      edge [<-, thick, blue!70!black] (f1-1);
    \end{tikzpicture}

\end{columns}
\end{frame}


\begin{frame}[fragile]{1.1 Обычная рекурсия: Шаг 3. Вычисление условия}
\begin{columns}[c]

  % ======== левая колонка: код ========
  \column{0.4\textwidth}
  \centering
  \usebox{\codeFactorialSimple}
  \vspace{0.4cm}

Интерпретатор спустился внутрь формы \mintinline{scheme}{if} и теперь вычисляет предикат \mintinline{scheme}{(= n 0)}.

\vfill

  % ======= правая колонка: схема ======
  \column{0.6\textwidth}
  \centering
  \begin{tikzpicture}[
      node distance=0.8cm,
      box/.style={
        draw=blue!70!black, fill=blue!5,
        rounded corners, thick, align=center,
        minimum width=3.8cm, minimum height=0.9cm,
        font=\footnotesize
      }
    ]
    \node[box] (repl) {\alert{REPL}};
    \node[box, below=of repl] (f1-1) {\mintinline{scheme}{(f1 3)}}
      edge [<-, thick, blue!70!black] (repl);
    \node[env-node=1, right=of f1-1] (stack1) { n $\mid$ 3 }
      edge [dashed, thick, blue!70!black] (f1-1);
    \node[box, below=of f1-1] (if-1) {\mintinline{scheme}{(if (= n 0))}}
      edge [<-, thick, blue!70!black] (f1-1);
    \node[box, below=of if-1] (eq-1) {\mintinline{scheme}{(= n 0)}}
      edge [<-, thick, blue!70!black] (if-1)
      edge [dashed, thick, blue!70!black, bend right=30] (stack1);
    \end{tikzpicture}

\end{columns}
\end{frame}


\begin{frame}[fragile]{1.1 Обычная рекурсия: Шаг 4. Условие ложно}
\begin{columns}[c]

  % ======== левая колонка: код ========
  \column{0.4\textwidth}
  \centering
  \usebox{\codeFactorialSimple}
  \vspace{0.4cm}

Вычисление предиката \mintinline{scheme}{(= n 0)} завершилось и вернуло значение \mintinline{scheme}{#f} (ложь)

  \vfill

  % ======= правая колонка: схема ======
  \column{0.6\textwidth}
  \centering
  \begin{tikzpicture}[
      node distance=0.8cm,
      box/.style={
        draw=blue!70!black, fill=blue!5,
        rounded corners, thick, align=center,
        minimum width=3.8cm, minimum height=0.9cm,
        font=\footnotesize
      }
    ]
    \node[box] (repl) {\alert{REPL}};
    \node[box, below=of repl] (f1-1) {\mintinline{scheme}{(f1 3)}}
      edge [<-, thick, blue!70!black] (repl);
    \node[env-node=1, right=of f1-1] (stack1) { n $\mid$ 3 }
      edge [dashed, thick, blue!70!black] (f1-1);
    \node[box, below=of f1-1] (if-1) {\mintinline{scheme}{(if (= n 0))}}
      edge [<-, thick, blue!70!black] (f1-1);
    \node[box, below=of if-1] (eq-1) {\mintinline{scheme}{#f}}
      edge [<-, thick, blue!70!black] (if-1)
      edge [->, thick, blue!70!black, bend left=30] (if-1);
    \end{tikzpicture}

\end{columns}
\end{frame}


\begin{frame}[fragile]{1.1 Обычная рекурсия: Шаг 5. Выбор ветки «иначе»}
\begin{columns}[c]

  % ======== левая колонка: код ========
  \column{0.4\textwidth}
  \centering
  \usebox{\codeFactorialSimple}
  \vspace{0.4cm}

Форма \mintinline{scheme}{if} получила от предиката значение \mintinline{scheme}{#f}.
Управление передаётся ветке c рекурсивным вызовом \mintinline{scheme}{(f1 (- n 1))}.

  \vfill

  % ======= правая колонка: схема ======
  \column{0.6\textwidth}
  \centering
  \begin{tikzpicture}[
      node distance=0.8cm,
      box/.style={
        draw=blue!70!black, fill=blue!5,
        rounded corners, thick, align=center,
        minimum width=3.8cm, minimum height=0.9cm,
        font=\footnotesize
      }
    ]
    \node[box] (repl) {\alert{REPL}};
    \node[box, below=of repl] (f1-1) {\mintinline{scheme}{(f1 3)}}
      edge [<-, thick, blue!70!black] (repl);
    \node[env-node=1, right=of f1-1] (stack1) { n $\mid$ 3 }
      edge [dashed, thick, blue!70!black] (f1-1);
    \node[box, below=of f1-1] (if-1) {\mintinline{scheme}{(if #f)}}
      edge [<-, thick, blue!70!black] (f1-1);
    \end{tikzpicture}

\end{columns}
\end{frame}


\begin{frame}[fragile]{1.1 Обычная рекурсия: Шаг 6. Тело обычной (нехвостовой) рекурсии}
\begin{columns}[c]

  % ======== левая колонка: код ========
  \column{0.4\textwidth}
  \centering
  \usebox{\codeFactorialSimple}
  \vspace{0.4cm}

После выбора ветки «иначе» интерпретатор начал вычислять тело процедуры — выражение \mintinline{scheme}{(* (f1 (- n 1)) n)}.

  \vfill

  % ======= правая колонка: схема ======
  \column{0.6\textwidth}
  \centering
  \begin{tikzpicture}[
      node distance=0.8cm,
      box/.style={
        draw=blue!70!black, fill=blue!5,
        rounded corners, thick, align=center,
        minimum width=3.8cm, minimum height=0.9cm,
        font=\footnotesize
      }
    ]
    \node[box] (repl) {\alert{REPL}};
    \node[box, below=of repl] (f1-1) {\mintinline{scheme}{(f1 3)}}
      edge [<-, thick, blue!70!black] (repl);
    \node[env-node=1, right=of f1-1] (stack1) { n $\mid$ 3 }
      edge [dashed, thick, blue!70!black] (f1-1);
    \node[box, below=of f1-1] (mul-1) {\mintinline{scheme}{(* (f1 (- n 1)) n)}}
      edge [<-, thick, blue!70!black] (f1-1)
      edge [dashed, thick, blue!70!black, bend right=10] (stack1);
    \end{tikzpicture}

\end{columns}
\end{frame}


\begin{frame}[fragile]{1.1 Обычная рекурсия: Шаг 7. Подготовка к рекурсивному вызову}
\begin{columns}[c]

  % ======== левая колонка: код ========
  \column{0.4\textwidth}
  \centering
  \usebox{\codeFactorialSimple}
  \vspace{0.4cm}

  Интерпретатор начал вычислять первый операнд умножения — выражение \mintinline{scheme}{(f1 (- n 1))}.

  \vfill

  % ======= правая колонка: схема ======
  \column{0.6\textwidth}
  \centering
  \begin{tikzpicture}[
      node distance=0.8cm,
      box/.style={
        draw=blue!70!black, fill=blue!5,
        rounded corners, thick, align=center,
        minimum width=3.8cm, minimum height=0.9cm,
        font=\footnotesize
      }
    ]
    \node[box] (repl) {\alert{REPL}};
    \node[box, below=of repl] (f1-1) {\mintinline{scheme}{(f1 3)}}
      edge [<-, thick, blue!70!black] (repl);
    \node[env-node=1, right=of f1-1] (stack1) { n $\mid$ 3 }
      edge [dashed, thick, blue!70!black] (f1-1);
    \node[box, below=of f1-1] (mul-1) {\mintinline{scheme}{(* (f1 (- n 1)) n)}}
      edge [<-, thick, blue!70!black] (f1-1)
      edge [dashed, thick, blue!70!black, bend right=10] (stack1);
    \node[box, below=of mul-1] (f1-2) {\mintinline{scheme}{(f1 (- n 1))}}
      edge [<-, thick, blue!70!black] (mul-1);
    \end{tikzpicture}
\end{columns}
\end{frame}

\begin{frame}[fragile]{1.1 Обычная рекурсия: Шаг 8. Вычисление фактического аргумента}
\begin{columns}[c]

  % ======== левая колонка: код ========
  \column{0.4\textwidth}
  \centering
  \usebox{\codeFactorialSimple}
  \vspace{0.4cm}

В новом вызове \mintinline{scheme}{f1} интерпретатор начал вычислять фактический параметр — выражение \mintinline{scheme}{(- n 1)}.
  \vfill

  % ======= правая колонка: схема ======
  \column{0.6\textwidth}
  \centering
  \begin{tikzpicture}[
      node distance=0.5cm,
      box/.style={
        draw=blue!70!black, fill=blue!5,
        rounded corners, thick, align=center,
        minimum width=3.8cm, minimum height=0.9cm,
        font=\footnotesize
      }
    ]
    \node[box] (repl) {\alert{REPL}};
    \node[box, below=of repl] (f1-1) {\mintinline{scheme}{(f1 3)}}
      edge [<-, thick, blue!70!black] (repl);
    \node[env-node=1, right=of f1-1] (stack1) { n $\mid$ 3 }
      edge [dashed, thick, blue!70!black] (f1-1);
    \node[box, below=of f1-1] (mul-1) {\mintinline{scheme}{(* (f1 (- n 1)) n)}}
      edge [<-, thick, blue!70!black] (f1-1)
      edge [dashed, thick, blue!70!black, bend right=10] (stack1);
    \node[box, below=of mul-1] (f1-2) {\mintinline{scheme}{(f1 (- n 1))}}
      edge [<-, thick, blue!70!black] (mul-1);
    \node[box, below=of f1-2] (sub-2) {\mintinline{scheme}{(- n 1)}}
      edge [<-, thick, blue!70!black] (f1-2)
      edge [dashed, thick, blue!70!black, bend right=40] (stack1);
    \end{tikzpicture}

\end{columns}
\end{frame}

  \begin{frame}[fragile]{1.1 Обычная рекурсия: Шаг 9. Аргумент вычислен — появляется второй кадр}
\begin{columns}[c]

  % ======== левая колонка: код ========
  \column{0.4\textwidth}
  \centering
  \usebox{\codeFactorialSimple}
  \vspace{0.4cm}

Выражение \mintinline{scheme}{(- n 1)} вернуло значение \mintinline{scheme}{  2}.
Интерпретатор связал его с параметром \mintinline{scheme}{ n } во втором вызове процедуры \mintinline{scheme}{ f1}.

  \vfill

  % ======= правая колонка: схема ======
  \column{0.6\textwidth}
  \centering
  \begin{tikzpicture}[
      node distance=0.5cm,
      box/.style={
        draw=blue!70!black, fill=blue!5,
        rounded corners, thick, align=center,
        minimum width=3.8cm, minimum height=0.9cm,
        font=\footnotesize
      }
    ]
    \node[box] (repl) {\alert{REPL}};
    \node[box, below=of repl] (f1-1) {\mintinline{scheme}{(f1 3)}}
      edge [<-, thick, blue!70!black] (repl);
    \node[env-node=1, right=of f1-1] (stack1) { n $\mid$ 3 }
      edge [dashed, thick, blue!70!black] (f1-1);
    \node[box, below=of f1-1] (mul-1) {\mintinline{scheme}{(* (f1 (- n 1)) n)}}
      edge [<-, thick, blue!70!black] (f1-1)
      edge [dashed, thick, blue!70!black, bend right=10] (stack1);
    \node[box, below=of mul-1] (f1-2) {\mintinline{scheme}{(f1 (- n 1))}}
      edge [<-, thick, blue!70!black] (mul-1);
    \node[box, below=of f1-2] (sub-2) {\mintinline{scheme}{2}}
      edge [<-, thick, blue!70!black] (f1-2)
      edge [dashed, thick, blue!70!black, bend right=40] (stack1)
      edge [->, thick, blue!70!black, bend left=30] (f1-2);
    \end{tikzpicture}

\end{columns}
\end{frame}

\begin{frame}[fragile]{1.1 Обычная рекурсия: Шаг 10. Второй уровень рекурсии активирован}
\begin{columns}[c]

  % ======== левая колонка: код ========
  \column{0.4\textwidth}
  \centering
  \usebox{\codeFactorialSimple}
  \vspace{0.4cm}

  Значение аргумента \mintinline{scheme}{  2} подставлено, второй кадр процедуры \mintinline{scheme}{ f1 } стал полноценным: теперь выполняется вызов \mintinline{scheme}{  (f1 2) }.

  \vfill

  % ======= правая колонка: схема ======
  \column{0.6\textwidth}
  \centering
  \begin{tikzpicture}[
      node distance=0.6cm,
      box/.style={
        draw=blue!70!black, fill=blue!5,
        rounded corners, thick, align=center,
        minimum width=3.8cm, minimum height=0.9cm,
        font=\footnotesize
      }
    ]
    \node[box] (repl) {\alert{REPL}};
    \node[box, below=of repl] (f1-1) {\mintinline{scheme}{(f1 3)}}
      edge [<-, thick, blue!70!black] (repl);
    \node[env-node=1, right=of f1-1] (stack1) { n $\mid$ 3 }
      edge [dashed, thick, blue!70!black] (f1-1);
    \node[box, below=of f1-1] (mul-1) {\mintinline{scheme}{(* (f1 (- n 1)) n)}}
      edge [<-, thick, blue!70!black] (f1-1)
      edge [dashed, thick, blue!70!black, bend right=10] (stack1);
    \node[box, below=of mul-1] (f1-2) {\mintinline{scheme}{(f1 2)}}
      edge [<-, thick, blue!70!black] (mul-1);
    \end{tikzpicture}

\end{columns}
\end{frame}

  \begin{frame}[fragile]{1.1 Обычная рекурсия: Шаг 11. Стек растёт}
\begin{columns}[c]

  % ======== левая колонка: код ========
  \column{0.4\textwidth}
  \centering
  \usebox{\codeFactorialSimple}
  \vspace{0.4cm}

Второй вызов \mintinline{scheme}{(f1 2)} полностью активирован — в стеке появился новый фрейм окружения с локальной переменной \mintinline{scheme}{n} $\mapsto$ \mintinline{scheme}{2}

  \vfill

  % ======= правая колонка: схема ======
  \column{0.6\textwidth}
  \centering
  \begin{tikzpicture}[
      node distance=0.6cm,
      box/.style={
        draw=blue!70!black, fill=blue!5,
        rounded corners, thick, align=center,
        minimum width=3.8cm, minimum height=0.9cm,
        font=\footnotesize
      }
    ]
    \node[box] (repl) {\alert{REPL}};
    \node[box, below=of repl] (f1-1) {\mintinline{scheme}{(f1 3)}}
      edge [<-, thick, blue!70!black] (repl);
    \node[env-node=1, right=of f1-1] (stack1) { n $\mid$ 3 }
      edge [dashed, thick, blue!70!black] (f1-1);
    \node[box, below=of f1-1] (mul-1) {\mintinline{scheme}{(* (f1 (- n 1)) n)}}
      edge [<-, thick, blue!70!black] (f1-1)
      edge [dashed, thick, blue!70!black, bend right=10] (stack1);
    \node[box, below=of mul-1] (f1-2) {\mintinline{scheme}{(f1 2)}}
      edge [<-, thick, blue!70!black] (mul-1);
    \node[env-node=1, right=of f1-2] (stack2) { n $\mid$ 2 }
      edge [dashed, thick, blue!70!black] (f1-2);
    \end{tikzpicture}

\end{columns}
\end{frame}

    \begin{frame}[fragile]{1.1 Обычная рекурсия: Шаг 12. Провал нехвостовой рекурсии}
\begin{columns}[c]

  % ======== левая колонка: код ========
  \column{0.4\textwidth}
  \centering
  \usebox{\codeFactorialSimple}
  \vspace{0.4cm}

  Всего за три шага рекурсии (от \mintinline{scheme}{ n = 3 }  до \mintinline{scheme}{ n = 0 }) стек вырос до восьми активных кадров

  \vfill

  % ======= правая колонка: схема ======
  \column{0.6\textwidth}
  \centering
  \begin{tikzpicture}[
      node distance=0.28cm,
      box/.style={
        draw=blue!70!black, fill=blue!5,
        rounded corners, thick, align=center,
        minimum width=3.8cm, minimum height=0.5cm,
        font=\footnotesize
      }
    ]
    \node[box] (repl) {\alert{REPL}};
    \node[box, below=of repl] (f1-1) {\mintinline{scheme}{(f1 3)}}
      edge [<-, thick, blue!70!black] (repl);
    \node[env-node=1, right=of f1-1] (stack1) { n $\mid$ 3 }
      edge [dashed, thick, blue!70!black] (f1-1);
    \node[box, below=of f1-1] (mul-1) {\mintinline{scheme}{(* (f1 (- n 1)) n)}}
      edge [<-, thick, blue!70!black] (f1-1)
      edge [dashed, thick, blue!70!black, bend right=10] (stack1);
    \node[box, below=of mul-1] (f1-2) {\mintinline{scheme}{(f1 2)}}
      edge [<-, thick, blue!70!black] (mul-1);
    \node[env-node=1, right=of f1-2] (stack2) { n $\mid$ 2 }
      edge [dashed, thick, blue!70!black] (f1-2);
    \node[box, below=of f1-2] (mul-2) {\mintinline{scheme}{(* (f1 (- n 1)) n)}}
      edge [<-, thick, blue!70!black] (f1-2)
      edge [dashed, thick, blue!70!black, bend right=10] (stack2);

    \node[box, below=of mul-2] (f1-3) {\mintinline{scheme}{(f1 1)}}
      edge [<-, thick, blue!70!black] (mul-2);
    \node[env-node=1, right=of f1-3] (stack3) { n $\mid$ 1 }
      edge [dashed, thick, blue!70!black] (f1-3);
    \node[box, below=of f1-3] (mul-3) {\mintinline{scheme}{(* (f1 (- n 1)) n)}}
      edge [<-, thick, blue!70!black] (f1-3)
      edge [dashed, thick, blue!70!black, bend right=10] (stack3);

    \node[box, below=of mul-3] (f1-4) {\mintinline{scheme}{(f1 0)}}
      edge [<-, thick, blue!70!black] (mul-3);
    \node[env-node=1, right=of f1-4] (stack3) { n $\mid$ 0 }
      edge [dashed, thick, blue!70!black] (f1-4);
    \end{tikzpicture}

\end{columns}
\end{frame}

    \begin{frame}[fragile]{1.1 Обычная рекурсия: Шаг 13. Начало сворачивания}
\begin{columns}[c]

  % ======== левая колонка: код ========
  \column{0.4\textwidth}
  \centering
  \usebox{\codeFactorialSimple}
  \vspace{0.4cm}

  Самый глубокий вызов \mintinline{scheme}{  (f1 0) } попал в ветку \mintinline{scheme}{  (if (= n 0) 1 …) }  и вернул значение \mintinline{scheme}{ 1  }.
Это первый результат, который пошёл вверх по цепочке.

  \vfill

  % ======= правая колонка: схема ======
  \column{0.6\textwidth}
  \centering
  \begin{tikzpicture}[
      node distance=0.28cm,
      box/.style={
        draw=blue!70!black, fill=blue!5,
        rounded corners, thick, align=center,
        minimum width=3.8cm, minimum height=0.5cm,
        font=\footnotesize
      }
    ]
    \node[box] (repl) {\alert{REPL}};
    \node[box, below=of repl] (f1-1) {\mintinline{scheme}{(f1 3)}}
      edge [<-, thick, blue!70!black] (repl);
    \node[env-node=1, right=of f1-1] (stack1) { n $\mid$ 3 }
      edge [dashed, thick, blue!70!black] (f1-1);
    \node[box, below=of f1-1] (mul-1) {\mintinline{scheme}{(* (f1 (- n 1)) n)}}
      edge [<-, thick, blue!70!black] (f1-1)
      edge [dashed, thick, blue!70!black, bend right=10] (stack1);
    \node[box, below=of mul-1] (f1-2) {\mintinline{scheme}{(f1 2)}}
      edge [<-, thick, blue!70!black] (mul-1);
    \node[env-node=1, right=of f1-2] (stack2) { n $\mid$ 2 }
      edge [dashed, thick, blue!70!black] (f1-2);
    \node[box, below=of f1-2] (mul-2) {\mintinline{scheme}{(* (f1 (- n 1)) n)}}
      edge [<-, thick, blue!70!black] (f1-2)
      edge [dashed, thick, blue!70!black, bend right=10] (stack2);

    \node[box, below=of mul-2] (f1-3) {\mintinline{scheme}{(f1 1)}}
      edge [<-, thick, blue!70!black] (mul-2);
    \node[env-node=1, right=of f1-3] (stack3) { n $\mid$ 1 }
      edge [dashed, thick, blue!70!black] (f1-3);
    \node[box, below=of f1-3] (mul-3) {\mintinline{scheme}{(* (f1 (- n 1)) n)}}
      edge [<-, thick, blue!70!black] (f1-3)
      edge [dashed, thick, blue!70!black, bend right=10] (stack3);

    \node[box, below=of mul-3] (f1-4) {\mintinline{scheme}{1}}
      edge [<-, thick, blue!70!black] (mul-3)
      edge [->, thick, blue!70!black, bend left=40] (mul-3);
    \node[env-node=1, right=of f1-4] (stack3) { n $\mid$ 0 }
      edge [dashed, thick, blue!70!black] (f1-4);
    \end{tikzpicture}

\end{columns}
\end{frame}


    \begin{frame}[fragile]{1.1 Обычная рекурсия: Шаг 14. Сворачивание стека: первое умножение выполнено}
\begin{columns}[c]

  % ======== левая колонка: код ========
  \column{0.4\textwidth}
  \centering
  \usebox{\codeFactorialSimple}
  \vspace{0.4cm}

  Самый глубокий кадр вернул \mintinline{scheme}{ 1 }.

  \vfill

  % ======= правая колонка: схема ======
  \column{0.6\textwidth}
  \centering
  \begin{tikzpicture}[
      node distance=0.32cm,
      box/.style={
        draw=blue!70!black, fill=blue!5,
        rounded corners, thick, align=center,
        minimum width=3.8cm, minimum height=0.5cm,
        font=\footnotesize
      }
    ]
    \node[box] (repl) {\alert{REPL}};
    \node[box, below=of repl] (f1-1) {\mintinline{scheme}{(f1 3)}}
      edge [<-, thick, blue!70!black] (repl);
    \node[env-node=1, right=of f1-1] (stack1) { n $\mid$ 3 }
      edge [dashed, thick, blue!70!black] (f1-1);
    \node[box, below=of f1-1] (mul-1) {\mintinline{scheme}{(* (f1 (- n 1)) n)}}
      edge [<-, thick, blue!70!black] (f1-1)
      edge [dashed, thick, blue!70!black, bend right=10] (stack1);
    \node[box, below=of mul-1] (f1-2) {\mintinline{scheme}{(f1 2)}}
      edge [<-, thick, blue!70!black] (mul-1);
    \node[env-node=1, right=of f1-2] (stack2) { n $\mid$ 2 }
      edge [dashed, thick, blue!70!black] (f1-2);
    \node[box, below=of f1-2] (mul-2) {\mintinline{scheme}{(* (f1 (- n 1)) n)}}
      edge [<-, thick, blue!70!black] (f1-2)
      edge [dashed, thick, blue!70!black, bend right=10] (stack2);

    \node[box, below=of mul-2] (f1-3) {\mintinline{scheme}{(f1 1)}}
      edge [<-, thick, blue!70!black] (mul-2);
    \node[env-node=1, right=of f1-3] (stack3) { n $\mid$ 1 }
      edge [dashed, thick, blue!70!black] (f1-3);
    \node[box, below=of f1-3] (mul-3) {\mintinline{scheme}{(* 1 1)}}
      edge [<-, thick, blue!70!black] (f1-3)
      edge [dashed, thick, blue!70!black, bend right=10] (stack3);

    \end{tikzpicture}

\end{columns}
\end{frame}


    \begin{frame}[fragile]{1.1 Обычная рекурсия: Шаг 15. Умножение на самом глубоком уровне завершено}
\begin{columns}[c]

  % ======== левая колонка: код ========
  \column{0.4\textwidth}
  \centering
  \usebox{\codeFactorialSimple}
  \vspace{0.4cm}

  Выражение \mintinline{scheme}{(* 1 1)} вычислено и вернуло \mintinline{scheme}{1}.
Кадр самого нижнего умножения полностью отработал и передал результат \mintinline{scheme}{1} вверх — в кадр процедуры \mintinline{scheme}{(f1 1)}.

  \vfill

  % ======= правая колонка: схема ======
  \column{0.6\textwidth}
  \centering
  \begin{tikzpicture}[
      node distance=0.32cm,
      box/.style={
        draw=blue!70!black, fill=blue!5,
        rounded corners, thick, align=center,
        minimum width=3.8cm, minimum height=0.5cm,
        font=\footnotesize
      }
    ]
    \node[box] (repl) {\alert{REPL}};
    \node[box, below=of repl] (f1-1) {\mintinline{scheme}{(f1 3)}}
      edge [<-, thick, blue!70!black] (repl);
    \node[env-node=1, right=of f1-1] (stack1) { n $\mid$ 3 }
      edge [dashed, thick, blue!70!black] (f1-1);
    \node[box, below=of f1-1] (mul-1) {\mintinline{scheme}{(* (f1 (- n 1)) n)}}
      edge [<-, thick, blue!70!black] (f1-1)
      edge [dashed, thick, blue!70!black, bend right=10] (stack1);
    \node[box, below=of mul-1] (f1-2) {\mintinline{scheme}{(f1 2)}}
      edge [<-, thick, blue!70!black] (mul-1);
    \node[env-node=1, right=of f1-2] (stack2) { n $\mid$ 2 }
      edge [dashed, thick, blue!70!black] (f1-2);
    \node[box, below=of f1-2] (mul-2) {\mintinline{scheme}{(* (f1 (- n 1)) n)}}
      edge [<-, thick, blue!70!black] (f1-2)
      edge [dashed, thick, blue!70!black, bend right=10] (stack2);

    \node[box, below=of mul-2] (f1-3) {\mintinline{scheme}{(f1 1)}}
      edge [<-, thick, blue!70!black] (mul-2);
    \node[env-node=1, right=of f1-3] (stack3) { n $\mid$ 1 }
      edge [dashed, thick, blue!70!black] (f1-3);
    \node[box, below=of f1-3] (mul-3) {\mintinline{scheme}{1}}
      edge [<-, thick, blue!70!black] (f1-3)
      edge [->, thick, blue!70!black, bend left=30] (f1-3)
      edge [dashed, thick, blue!70!black, bend right=10] (stack3);

    \end{tikzpicture}

\end{columns}
\end{frame}


    \begin{frame}[fragile]{1.1 Обычная рекурсия: Шаг 16. Второй уровень рекурсии завершён}
\begin{columns}[c]

  % ======== левая колонка: код ========
  \column{0.4\textwidth}
  \centering
  \usebox{\codeFactorialSimple}
  \vspace{0.4cm}

  Вызов \mintinline{scheme}{  (f1 1) } полностью отработал и вернул значение \mintinline{scheme}{  1 }.

  \vfill

  % ======= правая колонка: схема ======
  \column{0.6\textwidth}
  \centering
  \begin{tikzpicture}[
      node distance=0.34cm,
      box/.style={
        draw=blue!70!black, fill=blue!5,
        rounded corners, thick, align=center,
        minimum width=3.8cm, minimum height=0.5cm,
        font=\footnotesize
      }
    ]
    \node[box] (repl) {\alert{REPL}};
    \node[box, below=of repl] (f1-1) {\mintinline{scheme}{(f1 3)}}
      edge [<-, thick, blue!70!black] (repl);
    \node[env-node=1, right=of f1-1] (stack1) { n $\mid$ 3 }
      edge [dashed, thick, blue!70!black] (f1-1);
    \node[box, below=of f1-1] (mul-1) {\mintinline{scheme}{(* (f1 (- n 1)) n)}}
      edge [<-, thick, blue!70!black] (f1-1)
      edge [dashed, thick, blue!70!black, bend right=10] (stack1);
    \node[box, below=of mul-1] (f1-2) {\mintinline{scheme}{(f1 2)}}
      edge [<-, thick, blue!70!black] (mul-1);
    \node[env-node=1, right=of f1-2] (stack2) { n $\mid$ 2 }
      edge [dashed, thick, blue!70!black] (f1-2);
    \node[box, below=of f1-2] (mul-2) {\mintinline{scheme}{(* (f1 (- n 1)) n)}}
      edge [<-, thick, blue!70!black] (f1-2)
      edge [dashed, thick, blue!70!black, bend right=10] (stack2);

    \node[box, below=of mul-2] (f1-3) {\mintinline{scheme}{1}}
      edge [<-, thick, blue!70!black] (mul-2)
      edge [->, thick, blue!70!black, bend left=30] (mul-2);
    \end{tikzpicture}

\end{columns}
\end{frame}


    \begin{frame}[fragile]{1.1 Обычная рекурсия: Шаг 17. Предпоследний шаг сворачивания}
\begin{columns}[c]

  % ======== левая колонка: код ========
  \column{0.4\textwidth}
  \centering
  \usebox{\codeFactorialSimple}
  \vspace{0.4cm}

  Значение \mintinline{scheme}{ 1 } от \mintinline{scheme}{(f1 1)} уже превратилось во втором уровне в \mintinline{scheme}{2}.

  \vfill

  % ======= правая колонка: схема ======
  \column{0.6\textwidth}
  \centering
  \begin{tikzpicture}[
      node distance=0.34cm,
      box/.style={
        draw=blue!70!black, fill=blue!5,
        rounded corners, thick, align=center,
        minimum width=3.8cm, minimum height=0.5cm,
        font=\footnotesize
      }
    ]
    \node[box] (repl) {\alert{REPL}};
    \node[box, below=of repl] (f1-1) {\mintinline{scheme}{(f1 3)}}
      edge [<-, thick, blue!70!black] (repl);
    \node[env-node=1, right=of f1-1] (stack1) { n $\mid$ 3 }
      edge [dashed, thick, blue!70!black] (f1-1);
    \node[box, below=of f1-1] (mul-1) {\mintinline{scheme}{(* (f1 (- n 1)) n)}}
      edge [<-, thick, blue!70!black] (f1-1)
      edge [dashed, thick, blue!70!black, bend right=10] (stack1);
    \node[box, below=of mul-1] (f1-2) {\mintinline{scheme}{(f1 2)}}
      edge [<-, thick, blue!70!black] (mul-1);
    \node[env-node=1, right=of f1-2] (stack2) { n $\mid$ 2 }
      edge [dashed, thick, blue!70!black] (f1-2);
    \node[box, below=of f1-2] (mul-2) {\mintinline{scheme}{(* 1 2)}}
      edge [<-, thick, blue!70!black] (f1-2)
      edge [dashed, thick, blue!70!black, bend right=10] (stack2);
    \end{tikzpicture}

\end{columns}
\end{frame}


    \begin{frame}[fragile]{1.1 Обычная рекурсия: Шаг 18. Последнее умножение перед завершением}
\begin{columns}[c]

  % ======== левая колонка: код ========
  \column{0.4\textwidth}
  \centering
  \usebox{\codeFactorialSimple}
  \vspace{0.4cm}

  Умножение второго уровня завершено и вернуло \mintinline{scheme}{2}  .
Этот результат подставлен в самое первое, «самое долго ждавшее» умножение: \mintinline{scheme}{  (* 2 n) } $\to$ \mintinline{scheme}{  (* 2 3) }.

  \vfill

  % ======= правая колонка: схема ======
  \column{0.6\textwidth}
  \centering
  \begin{tikzpicture}[
      node distance=0.36cm,
      box/.style={
        draw=blue!70!black, fill=blue!5,
        rounded corners, thick, align=center,
        minimum width=3.8cm, minimum height=0.5cm,
        font=\footnotesize
      }
    ]
    \node[box] (repl) {\alert{REPL}};
    \node[box, below=of repl] (f1-1) {\mintinline{scheme}{(f1 3)}}
      edge [<-, thick, blue!70!black] (repl);
    \node[env-node=1, right=of f1-1] (stack1) { n $\mid$ 3 }
      edge [dashed, thick, blue!70!black] (f1-1);
    \node[box, below=of f1-1] (mul-1) {\mintinline{scheme}{(* (f1 (- n 1)) n)}}
      edge [<-, thick, blue!70!black] (f1-1)
      edge [dashed, thick, blue!70!black, bend right=10] (stack1);
    \node[box, below=of mul-1] (f1-2) {\mintinline{scheme}{(f1 2)}}
      edge [<-, thick, blue!70!black] (mul-1);
    \node[env-node=1, right=of f1-2] (stack2) { n $\mid$ 2 }
      edge [dashed, thick, blue!70!black] (f1-2);
    \node[box, below=of f1-2] (mul-2) {\mintinline{scheme}{2}}
      edge [<-, thick, blue!70!black] (f1-2)
      edge [->, thick, blue!70!black, bend left=30] (f1-2)
      edge [dashed, thick, blue!70!black, bend right=10] (stack2);
    \end{tikzpicture}

\end{columns}
\end{frame}


    \begin{frame}[fragile]{1.1 Обычная рекурсия: Шаг 19. Самое последнее действие нехвостовой рекурсии}
\begin{columns}[c]

  % ======== левая колонка: код ========
  \column{0.4\textwidth}
  \centering
  \usebox{\codeFactorialSimple}
  \vspace{0.4cm}

  Выполняется финальное умножение \mintinline{scheme}{  (* 2 3) }.

  \vfill

  % ======= правая колонка: схема ======
  \column{0.6\textwidth}
  \centering
  \begin{tikzpicture}[
      node distance=0.36cm,
      box/.style={
        draw=blue!70!black, fill=blue!5,
        rounded corners, thick, align=center,
        minimum width=3.8cm, minimum height=0.5cm,
        font=\footnotesize
      }
    ]
    \node[box] (repl) {\alert{REPL}};
    \node[box, below=of repl] (f1-1) {\mintinline{scheme}{(f1 3)}}
      edge [<-, thick, blue!70!black] (repl);
    \node[env-node=1, right=of f1-1] (stack1) { n $\mid$ 3 }
      edge [dashed, thick, blue!70!black] (f1-1);
    \node[box, below=of f1-1] (mul-1) {\mintinline{scheme}{(* (f1 (- n 1)) n)}}
      edge [<-, thick, blue!70!black] (f1-1)
      edge [dashed, thick, blue!70!black, bend right=10] (stack1);
    \node[box, below=of mul-1] (f1-2) {\mintinline{scheme}{2}}
      edge [<-, thick, blue!70!black] (mul-1)
      edge [->, thick, blue!70!black, bend left=30] (mul-1);
    \end{tikzpicture}

\end{columns}
\end{frame}

    \begin{frame}[fragile]{1.1 Обычная рекурсия: Шаг 20. Финальный кадр обычной рекурсии}
\begin{columns}[c]

  % ======== левая колонка: код ========
  \column{0.4\textwidth}
  \centering
  \usebox{\codeFactorialSimple}
  \vspace{0.4cm}

  Выполняется последнее умножение \mintinline{scheme}{  (* 2 3) }.

  \vfill

  % ======= правая колонка: схема ======
  \column{0.6\textwidth}
  \centering
  \begin{tikzpicture}[
      node distance=0.38cm,
      box/.style={
        draw=blue!70!black, fill=blue!5,
        rounded corners, thick, align=center,
        minimum width=3.8cm, minimum height=0.5cm,
        font=\footnotesize
      }
    ]
    \node[box] (repl) {\alert{REPL}};
    \node[box, below=of repl] (f1-1) {\mintinline{scheme}{(f1 3)}}
      edge [<-, thick, blue!70!black] (repl);
    \node[env-node=1, right=of f1-1] (stack1) { n $\mid$ 3 }
      edge [dashed, thick, blue!70!black] (f1-1);
    \node[box, below=of f1-1] (mul-1) {\mintinline{scheme}{(* 2 3)}}
      edge [<-, thick, blue!70!black] (f1-1)
      edge [dashed, thick, blue!70!black, bend right=10] (stack1);
    \end{tikzpicture}

\end{columns}
\end{frame}


    \begin{frame}[fragile]{1.1 Обычная рекурсия: Шаг 21. Рекурсия завершена — результат готов}
\begin{columns}[c]

  % ======== левая колонка: код ========
  \column{0.4\textwidth}
  \centering
  \usebox{\codeFactorialSimple}
  \vspace{0.4cm}

  Умножение \mintinline{scheme}{  (* 2 3) } вернуло \mintinline{scheme}{  6}.
Это и есть окончательное значение всего выражения \mintinline{scheme}{  (f1 3)}.

  \vfill

  % ======= правая колонка: схема ======
  \column{0.6\textwidth}
  \centering
  \begin{tikzpicture}[
      node distance=0.38cm,
      box/.style={
        draw=blue!70!black, fill=blue!5,
        rounded corners, thick, align=center,
        minimum width=3.8cm, minimum height=0.5cm,
        font=\footnotesize
      }
    ]
    \node[box] (repl) {\alert{REPL}};
    \node[box, below=of repl] (f1-1) {\mintinline{scheme}{(f1 3)}}
      edge [<-, thick, blue!70!black] (repl);
    \node[env-node=1, right=of f1-1] (stack1) { n $\mid$ 3 }
      edge [dashed, thick, blue!70!black] (f1-1);
    \node[box, below=of f1-1] (mul-1) {\mintinline{scheme}{6}}
      edge [<-, thick, blue!70!black] (f1-1)
      edge [->, thick, blue!70!black, bend left=30] (f1-1)
      edge [dashed, thick, blue!70!black, bend right=10] (stack1);
    \end{tikzpicture}

\end{columns}
\end{frame}


    \begin{frame}[fragile]{1.1 Обычная рекурсия: Шаг 22. Обычная рекурсия завершена}
\begin{columns}[c]

  % ======== левая колонка: код ========
  \column{0.4\textwidth}
  \centering
  \usebox{\codeFactorialSimple}
  \vspace{0.4cm}

  Значение \mintinline{scheme}{  6 } возвращено в \alert{REPL}.

  \vfill

  % ======= правая колонка: схема ======
  \column{0.6\textwidth}
  \centering
  \begin{tikzpicture}[
      node distance=0.38cm,
      box/.style={
        draw=blue!70!black, fill=blue!5,
        rounded corners, thick, align=center,
        minimum width=3.8cm, minimum height=0.5cm,
        font=\footnotesize
      }
    ]
    \node[box] (repl) {\alert{REPL}};
    \node[box, below=of repl] (f1-1) {\mintinline{scheme}{6}}
      edge [<-, thick, blue!70!black] (repl);
    \end{tikzpicture}

\end{columns}
\end{frame}