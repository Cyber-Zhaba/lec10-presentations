\begin{frame}[fragile]{Глава 1: Оптимизация хвостовой рекурсии}
    \centering

    Рекурсивный вызов является \textbf{хвостовым}, если он —
    \alert{последнее действие} в выполняемой процедуре, и после
    возврата из рекурсивного вызова процедуры \alert{ничего больше делать не нужно}.

\end{frame}


\begin{frame}{1.1 Обычная рекурсия}
    \centering

    Начнем с анализа процедуры \mintinline{scheme}{f1} —
    примера обычной рекурсии.
    Мы разберем, как \alert{REPL} создает фреймы стека и возвращает результаты.

    \usebox{\codeFactorialSimple}
\end{frame}


\begin{frame}[fragile]{1.1 Обычная рекурсия: Шаг 1. Начальный вызов}
    \begin{columns}[c]

        % ======== левая колонка: код ========
        \column{0.4\textwidth}
        \centering
        \usebox{\codeFactorialSimple}
        \vspace{0.4cm}

        Среда \alert{REPL} вызывает \mintinline{scheme}{f1}
        с аргументом \mintinline{scheme}{3}.
        Это старт рекурсии — создается первый фрейм стека с \mintinline{scheme}{n = 3}.

        \vfill

        % ======= правая колонка: схема ======
        \column{0.6\textwidth}
        \centering
        \begin{tikzpicture}[
            node distance=1.1cm,
            box/.style={
                draw=blue!70!black, fill=blue!5,
                rounded corners, thick, align=center,
                minimum width=3.8cm, minimum height=0.9cm,
                font=\footnotesize
            }
        ]
            \node[box] (repl) {\alert{REPL}};
            \node[box, below=of repl] (f1-1) {\mintinline{scheme}{(f1 n)}}
            edge [<-, thick, blue!70!black] (repl);
            \node[env-node=1, right=of f1-1] (stack1) { n $\mid$ 3 }
            edge [dashed, thick, blue!70!black] (f1-1);
        \end{tikzpicture}
    \end{columns}
\end{frame}


\begin{frame}[fragile]{1.1 Обычная рекурсия: Шаг 2. Вызов \alert{if}}
    \begin{columns}[c]

        % ======== левая колонка: код ========
        \column{0.4\textwidth}
        \centering
        \usebox{\codeFactorialSimple}
        \vspace{0.4cm}

        После вызова интерпретатор переходит к специальной форме
        \mintinline{scheme}{if}, проверяя условие для выхода из рекурсии.

        \vfill

        % ======= правая колонка: схема ======
        \column{0.6\textwidth}
        \centering
        \begin{tikzpicture}[
            node distance=0.8cm,
            box/.style={
                draw=blue!70!black, fill=blue!5,
                rounded corners, thick, align=center,
                minimum width=3.8cm, minimum height=0.9cm,
                font=\footnotesize
            }
        ]
            \node[box] (repl) {\alert{REPL}};
            \node[box, below=of repl] (f1-1) {\mintinline{scheme}{(f1 3)}}
            edge [<-, thick, blue!70!black] (repl);
            \node[env-node=1, right=of f1-1] (stack1) { n $\mid$ 3 }
            edge [dashed, thick, blue!70!black] (f1-1);

            \node[box, below=of f1-1] (if-1) {\mintinline{scheme}{(if (= n 0))}}
            edge [<-, thick, blue!70!black] (f1-1);
        \end{tikzpicture}
    \end{columns}
\end{frame}


\begin{frame}[fragile]{1.1 Обычная рекурсия: Шаг 3. Вычисление предиката}
    \begin{columns}[c]

        % ======== левая колонка: код ========
        \column{0.4\textwidth}
        \centering
        \usebox{\codeFactorialSimple}
        \vspace{0.4cm}

        Внутри \mintinline{scheme}{if} вычисляется предикат \mintinline{scheme}{(= n 0)}

        \vfill

        % ======= правая колонка: схема ======
        \column{0.6\textwidth}
        \centering
        \begin{tikzpicture}[
            node distance=0.8cm,
            box/.style={
                draw=blue!70!black, fill=blue!5,
                rounded corners, thick, align=center,
                minimum width=3.8cm, minimum height=0.9cm,
                font=\footnotesize
            }
        ]
            \node[box] (repl) {\alert{REPL}};
            \node[box, below=of repl] (f1-1) {\mintinline{scheme}{(f1 3)}}
            edge [<-, thick, blue!70!black] (repl);
            \node[env-node=1, right=of f1-1] (stack1) { n $\mid$ 3 }
            edge [dashed, thick, blue!70!black] (f1-1);
            \node[box, below=of f1-1] (if-1) {\mintinline{scheme}{(if (= n 0))}}
            edge [<-, thick, blue!70!black] (f1-1);
            \node[box, below=of if-1] (eq-1) {\mintinline{scheme}{(= n 0)}}
            edge [<-, thick, blue!70!black] (if-1)
            edge [dashed, thick, blue!70!black, bend right=30] (stack1);
        \end{tikzpicture}
    \end{columns}
\end{frame}


\begin{frame}[fragile]{1.1 Обычная рекурсия: Шаг 4. Предикат ложен}
    \begin{columns}[c]

        % ======== левая колонка: код ========
        \column{0.4\textwidth}
        \centering
        \usebox{\codeFactorialSimple}
        \vspace{0.4cm}

        Предикат \mintinline{scheme}{(= n 0)}
        возвращает \mintinline{scheme}{#f}.

        \vfill

        % ======= правая колонка: схема ======
        \column{0.6\textwidth}
        \centering
        \begin{tikzpicture}[
            node distance=0.8cm,
            box/.style={
                draw=blue!70!black, fill=blue!5,
                rounded corners, thick, align=center,
                minimum width=3.8cm, minimum height=0.9cm,
                font=\footnotesize
            }
        ]
            \node[box] (repl) {\alert{REPL}};
            \node[box, below=of repl] (f1-1) {\mintinline{scheme}{(f1 3)}}
            edge [<-, thick, blue!70!black] (repl);
            \node[env-node=1, right=of f1-1] (stack1) { n $\mid$ 3 }
            edge [dashed, thick, blue!70!black] (f1-1);
            \node[box, below=of f1-1] (if-1) {\mintinline{scheme}{(if (= n 0))}}
            edge [<-, thick, blue!70!black] (f1-1);
            \node[box, below=of if-1] (eq-1) {\mintinline{scheme}{#f}}
            edge [<-, thick, blue!70!black] (if-1)
            edge [->, thick, blue!70!black, bend left=30] (if-1);
        \end{tikzpicture}
    \end{columns}
\end{frame}


\begin{frame}[fragile]{1.1 Обычная рекурсия: Шаг 5. Выбор else-ветки}
    \begin{columns}[c]

        % ======== левая колонка: код ========
        \column{0.4\textwidth}
        \centering
        \usebox{\codeFactorialSimple}
        \vspace{0.4cm}

        После получения \mintinline{scheme}{#f} управление
        передается в ветку else: \mintinline{scheme}{(* (f1 (- n 1)) n)}

        \vfill

        % ======= правая колонка: схема ======
        \column{0.6\textwidth}
        \centering
        \begin{tikzpicture}[
            node distance=0.8cm,
            box/.style={
                draw=blue!70!black, fill=blue!5,
                rounded corners, thick, align=center,
                minimum width=3.8cm, minimum height=0.9cm,
                font=\footnotesize
            }
        ]
            \node[box] (repl) {\alert{REPL}};
            \node[box, below=of repl] (f1-1) {\mintinline{scheme}{(f1 3)}}
            edge [<-, thick, blue!70!black] (repl);
            \node[env-node=1, right=of f1-1] (stack1) { n $\mid$ 3 }
            edge [dashed, thick, blue!70!black] (f1-1);
            \node[box, below=of f1-1] (if-1) {\mintinline{scheme}{(if #f)}}
            edge [<-, thick, blue!70!black] (f1-1);
        \end{tikzpicture}

    \end{columns}
\end{frame}


\begin{frame}[fragile]{1.1 Обычная рекурсия: Шаг 6. Вычисление тела рекурсии}
    \begin{columns}[c]

        % ======== левая колонка: код ========
        \column{0.4\textwidth}
        \centering
        \usebox{\codeFactorialSimple}
        \vspace{0.4cm}

        Интерпретатор начинает вычислять выражение
        \mintinline{scheme}{(* (f1 (- n 1)) n)},
        готовясь к рекурсивному вызову

        \vfill

        % ======= правая колонка: схема ======
        \column{0.6\textwidth}
        \centering
        \begin{tikzpicture}[
            node distance=0.8cm,
            box/.style={
                draw=blue!70!black, fill=blue!5,
                rounded corners, thick, align=center,
                minimum width=3.8cm, minimum height=0.9cm,
                font=\footnotesize
            }
        ]
            \node[box] (repl) {\alert{REPL}};
            \node[box, below=of repl] (f1-1) {\mintinline{scheme}{(f1 3)}}
            edge [<-, thick, blue!70!black] (repl);
            \node[env-node=1, right=of f1-1] (stack1) { n $\mid$ 3 }
            edge [dashed, thick, blue!70!black] (f1-1);
            \node[box, below=of f1-1] (mul-1) {\mintinline{scheme}{(* (f1 (- n 1)) n)}}
            edge [<-, thick, blue!70!black] (f1-1)
            edge [dashed, thick, blue!70!black, bend right=10] (stack1);
        \end{tikzpicture}
    \end{columns}
\end{frame}


\begin{frame}[fragile]{1.1 Обычная рекурсия: Шаг 7. Подготовка рекурсивного вызова}
    \begin{columns}[c]

        % ======== левая колонка: код ========
        \column{0.4\textwidth}
        \centering
        \usebox{\codeFactorialSimple}
        \vspace{0.4cm}

        Интерпретатор начал вычислять первый операнд
        умножения — выражение \mintinline{scheme}{(f1 (- n 1))}.

        \vfill

        % ======= правая колонка: схема ======
        \column{0.6\textwidth}
        \centering
        \begin{tikzpicture}[
            node distance=0.8cm,
            box/.style={
                draw=blue!70!black, fill=blue!5,
                rounded corners, thick, align=center,
                minimum width=3.8cm, minimum height=0.9cm,
                font=\footnotesize
            }
        ]
            \node[box] (repl) {\alert{REPL}};
            \node[box, below=of repl] (f1-1) {\mintinline{scheme}{(f1 3)}}
            edge [<-, thick, blue!70!black] (repl);
            \node[env-node=1, right=of f1-1] (stack1) { n $\mid$ 3 }
            edge [dashed, thick, blue!70!black] (f1-1);
            \node[box, below=of f1-1] (mul-1) {\mintinline{scheme}{(* (f1 (- n 1)) n)}}
            edge [<-, thick, blue!70!black] (f1-1)
            edge [dashed, thick, blue!70!black, bend right=10] (stack1);
            \node[box, below=of mul-1] (f1-2) {\mintinline{scheme}{(f1 (- n 1))}}
            edge [<-, thick, blue!70!black] (mul-1);
        \end{tikzpicture}
    \end{columns}
\end{frame}


\begin{frame}[fragile]{1.1 Обычная рекурсия: Шаг 8. Вычисление аргумента}
    \begin{columns}[c]

        % ======== левая колонка: код ========
        \column{0.4\textwidth}
        \centering
        \usebox{\codeFactorialSimple}
        \vspace{0.4cm}

        Вычисляется аргумент для следующего вызова:
        \mintinline{scheme}{(- n 1)} = 2.
        Это подготавливает новый уровень рекурсии.

        \vfill

        % ======= правая колонка: схема ======
        \column{0.6\textwidth}
        \centering
        \begin{tikzpicture}[
            node distance=0.5cm,
            box/.style={
                draw=blue!70!black, fill=blue!5,
                rounded corners, thick, align=center,
                minimum width=3.8cm, minimum height=0.9cm,
                font=\footnotesize
            }
        ]
            \node[box] (repl) {\alert{REPL}};
            \node[box, below=of repl] (f1-1) {\mintinline{scheme}{(f1 3)}}
            edge [<-, thick, blue!70!black] (repl);
            \node[env-node=1, right=of f1-1] (stack1) { n $\mid$ 3 }
            edge [dashed, thick, blue!70!black] (f1-1);
            \node[box, below=of f1-1] (mul-1) {\mintinline{scheme}{(* (f1 (- n 1)) n)}}
            edge [<-, thick, blue!70!black] (f1-1)
            edge [dashed, thick, blue!70!black, bend right=10] (stack1);
            \node[box, below=of mul-1] (f1-2) {\mintinline{scheme}{(f1 (- n 1))}}
            edge [<-, thick, blue!70!black] (mul-1);
            \node[box, below=of f1-2] (sub-2) {\mintinline{scheme}{(- n 1)}}
            edge [<-, thick, blue!70!black] (f1-2)
            edge [dashed, thick, blue!70!black, bend right=40] (stack1);
        \end{tikzpicture}
    \end{columns}
\end{frame}


\begin{frame}[fragile]{1.1 Обычная рекурсия: Шаг 9. Вычисление аргумента}
    \begin{columns}[c]

        % ======== левая колонка: код ========
        \column{0.4\textwidth}
        \centering
        \usebox{\codeFactorialSimple}
        \vspace{0.4cm}

        Выражение \mintinline{scheme}{(- n 1)} вернуло значение \mintinline{scheme}{2}.
        Интерпретатор связал его с параметром \mintinline{scheme}{n} во втором
        вызове процедуры \mintinline{scheme}{f1}.

        \vfill

        % ======= правая колонка: схема ======
        \column{0.6\textwidth}
        \centering
        \begin{tikzpicture}[
            node distance=0.5cm,
            box/.style={
                draw=blue!70!black, fill=blue!5,
                rounded corners, thick, align=center,
                minimum width=3.8cm, minimum height=0.9cm,
                font=\footnotesize
            }
        ]
            \node[box] (repl) {\alert{REPL}};
            \node[box, below=of repl] (f1-1) {\mintinline{scheme}{(f1 3)}}
            edge [<-, thick, blue!70!black] (repl);
            \node[env-node=1, right=of f1-1] (stack1) { n $\mid$ 3 }
            edge [dashed, thick, blue!70!black] (f1-1);
            \node[box, below=of f1-1] (mul-1) {\mintinline{scheme}{(* (f1 (- n 1)) n)}}
            edge [<-, thick, blue!70!black] (f1-1)
            edge [dashed, thick, blue!70!black, bend right=10] (stack1);
            \node[box, below=of mul-1] (f1-2) {\mintinline{scheme}{(f1 (- n 1))}}
            edge [<-, thick, blue!70!black] (mul-1);
            \node[box, below=of f1-2] (sub-2) {\mintinline{scheme}{2}}
            edge [<-, thick, blue!70!black] (f1-2)
            edge [dashed, thick, blue!70!black, bend right=40] (stack1)
            edge [->, thick, blue!70!black, bend left=30] (f1-2);
        \end{tikzpicture}
    \end{columns}
\end{frame}


\begin{frame}[fragile]{1.1 Обычная рекурсия: Шаг 10. Рекурсивный вызов}
    \begin{columns}[c]

        % ======== левая колонка: код ========
        \column{0.4\textwidth}
        \centering
        \usebox{\codeFactorialSimple}
        \vspace{0.4cm}

        Второй вызов \mintinline{scheme}{(f1 2)} активирован.
        Стек продолжает расти с новым фреймом.

        \vfill

        % ======= правая колонка: схема ======
        \column{0.6\textwidth}
        \centering
        \begin{tikzpicture}[
            node distance=0.6cm,
            box/.style={
                draw=blue!70!black, fill=blue!5,
                rounded corners, thick, align=center,
                minimum width=3.8cm, minimum height=0.9cm,
                font=\footnotesize
            }
        ]
            \node[box] (repl) {\alert{REPL}};
            \node[box, below=of repl] (f1-1) {\mintinline{scheme}{(f1 3)}}
            edge [<-, thick, blue!70!black] (repl);
            \node[env-node=1, right=of f1-1] (stack1) { n $\mid$ 3 }
            edge [dashed, thick, blue!70!black] (f1-1);
            \node[box, below=of f1-1] (mul-1) {\mintinline{scheme}{(* (f1 (- n 1)) n)}}
            edge [<-, thick, blue!70!black] (f1-1)
            edge [dashed, thick, blue!70!black, bend right=10] (stack1);
            \node[box, below=of mul-1] (f1-2) {\mintinline{scheme}{(f1 2)}}
            edge [<-, thick, blue!70!black] (mul-1);
        \end{tikzpicture}
    \end{columns}
\end{frame}


\begin{frame}[fragile]{1.1 Обычная рекурсия: Шаг 11. Стек растёт}
    \begin{columns}[c]

        % ======== левая колонка: код ========
        \column{0.4\textwidth}
        \centering
        \usebox{\codeFactorialSimple}
        \vspace{0.4cm}

        В стеке появился новый фрейм
        окружения с локальной переменной \mintinline{scheme}{n}
        $\mapsto$ \mintinline{scheme}{2}

        \vfill

        % ======= правая колонка: схема ======
        \column{0.6\textwidth}
        \centering
        \begin{tikzpicture}[
            node distance=0.6cm,
            box/.style={
                draw=blue!70!black, fill=blue!5,
                rounded corners, thick, align=center,
                minimum width=3.8cm, minimum height=0.9cm,
                font=\footnotesize
            }
        ]
            \node[box] (repl) {\alert{REPL}};
            \node[box, below=of repl] (f1-1) {\mintinline{scheme}{(f1 3)}}
            edge [<-, thick, blue!70!black] (repl);
            \node[env-node=1, right=of f1-1] (stack1) { n $\mid$ 3 }
            edge [dashed, thick, blue!70!black] (f1-1);
            \node[box, below=of f1-1] (mul-1) {\mintinline{scheme}{(* (f1 (- n 1)) n)}}
            edge [<-, thick, blue!70!black] (f1-1)
            edge [dashed, thick, blue!70!black, bend right=10] (stack1);
            \node[box, below=of mul-1] (f1-2) {\mintinline{scheme}{(f1 2)}}
            edge [<-, thick, blue!70!black] (mul-1);
            \node[env-node=1, right=of f1-2] (stack2) { n $\mid$ 2 }
            edge [dashed, thick, blue!70!black] (f1-2);
        \end{tikzpicture}
    \end{columns}
\end{frame}


\begin{frame}[fragile]{1.1 Обычная рекурсия: Шаг 12. Максимальная глубина стека}
    \begin{columns}[c]

        % ======== левая колонка: код ========
        \column{0.4\textwidth}
        \centering
        \usebox{\codeFactorialSimple}
        \vspace{0.4cm}

        Всего за три шага рекурсии

        (от \mintinline{scheme}{n = 3}
        до \mintinline{scheme}{n = 0})
        стек вырос до четырёх активных фреймов.

        \vfill

        % ======= правая колонка: схема ======
        \column{0.6\textwidth}
        \centering
        \begin{tikzpicture}[
            node distance=0.28cm,
            box/.style={
                draw=blue!70!black, fill=blue!5,
                rounded corners, thick, align=center,
                minimum width=3.8cm, minimum height=0.5cm,
                font=\footnotesize
            }
        ]
            \node[box] (repl) {\alert{REPL}};
            \node[box, below=of repl] (f1-1) {\mintinline{scheme}{(f1 3)}}
            edge [<-, thick, blue!70!black] (repl);
            \node[env-node=1, right=of f1-1] (stack1) { n $\mid$ 3 }
            edge [dashed, thick, blue!70!black] (f1-1);
            \node[box, below=of f1-1] (mul-1) {\mintinline{scheme}{(* (f1 (- n 1)) n)}}
            edge [<-, thick, blue!70!black] (f1-1)
            edge [dashed, thick, blue!70!black, bend right=10] (stack1);
            \node[box, below=of mul-1] (f1-2) {\mintinline{scheme}{(f1 2)}}
            edge [<-, thick, blue!70!black] (mul-1);
            \node[env-node=1, right=of f1-2] (stack2) { n $\mid$ 2 }
            edge [dashed, thick, blue!70!black] (f1-2);
            \node[box, below=of f1-2] (mul-2) {\mintinline{scheme}{(* (f1 (- n 1)) n)}}
            edge [<-, thick, blue!70!black] (f1-2)
            edge [dashed, thick, blue!70!black, bend right=10] (stack2);

            \node[box, below=of mul-2] (f1-3) {\mintinline{scheme}{(f1 1)}}
            edge [<-, thick, blue!70!black] (mul-2);
            \node[env-node=1, right=of f1-3] (stack3) { n $\mid$ 1 }
            edge [dashed, thick, blue!70!black] (f1-3);
            \node[box, below=of f1-3] (mul-3) {\mintinline{scheme}{(* (f1 (- n 1)) n)}}
            edge [<-, thick, blue!70!black] (f1-3)
            edge [dashed, thick, blue!70!black, bend right=10] (stack3);

            \node[box, below=of mul-3] (f1-4) {\mintinline{scheme}{(f1 0)}}
            edge [<-, thick, blue!70!black] (mul-3);
            \node[env-node=1, right=of f1-4] (stack3) { n $\mid$ 0 }
            edge [dashed, thick, blue!70!black] (f1-4);
        \end{tikzpicture}
    \end{columns}
\end{frame}


\begin{frame}[fragile]{1.1 Обычная рекурсия: Шаг 13. Начало сворачивания стека}
    \begin{columns}[c]

        % ======== левая колонка: код ========
        \column{0.4\textwidth}
        \centering
        \usebox{\codeFactorialSimple}
        \vspace{0.4cm}

        Самый глубокий вызов \mintinline{scheme}{(f1 0)} попал в ветку
        \mintinline{scheme}{(if (= n 0) 1 ...)} и
        вернул значение \mintinline{scheme}{1}.
        Это первый результат, который пошёл вверх по цепочке.

        \vfill

        % ======= правая колонка: схема ======
        \column{0.6\textwidth}
        \centering
        \begin{tikzpicture}[
            node distance=0.28cm,
            box/.style={
                draw=blue!70!black, fill=blue!5,
                rounded corners, thick, align=center,
                minimum width=3.8cm, minimum height=0.5cm,
                font=\footnotesize
            }
        ]
            \node[box] (repl) {\alert{REPL}};
            \node[box, below=of repl] (f1-1) {\mintinline{scheme}{(f1 3)}}
            edge [<-, thick, blue!70!black] (repl);
            \node[env-node=1, right=of f1-1] (stack1) { n $\mid$ 3 }
            edge [dashed, thick, blue!70!black] (f1-1);
            \node[box, below=of f1-1] (mul-1) {\mintinline{scheme}{(* (f1 (- n 1)) n)}}
            edge [<-, thick, blue!70!black] (f1-1)
            edge [dashed, thick, blue!70!black, bend right=10] (stack1);
            \node[box, below=of mul-1] (f1-2) {\mintinline{scheme}{(f1 2)}}
            edge [<-, thick, blue!70!black] (mul-1);
            \node[env-node=1, right=of f1-2] (stack2) { n $\mid$ 2 }
            edge [dashed, thick, blue!70!black] (f1-2);
            \node[box, below=of f1-2] (mul-2) {\mintinline{scheme}{(* (f1 (- n 1)) n)}}
            edge [<-, thick, blue!70!black] (f1-2)
            edge [dashed, thick, blue!70!black, bend right=10] (stack2);

            \node[box, below=of mul-2] (f1-3) {\mintinline{scheme}{(f1 1)}}
            edge [<-, thick, blue!70!black] (mul-2);
            \node[env-node=1, right=of f1-3] (stack3) { n $\mid$ 1 }
            edge [dashed, thick, blue!70!black] (f1-3);
            \node[box, below=of f1-3] (mul-3) {\mintinline{scheme}{(* (f1 (- n 1)) n)}}
            edge [<-, thick, blue!70!black] (f1-3)
            edge [dashed, thick, blue!70!black, bend right=10] (stack3);

            \node[box, below=of mul-3] (f1-4) {\mintinline{scheme}{1}}
            edge [<-, thick, blue!70!black] (mul-3)
            edge [->, thick, blue!70!black, bend left=40] (mul-3);
            \node[env-node=1, right=of f1-4] (stack3) { n $\mid$ 0 }
            edge [dashed, thick, blue!70!black] (f1-4);
        \end{tikzpicture}
    \end{columns}
\end{frame}


\begin{frame}[fragile]{1.1 Обычная рекурсия: Шаг 14. Первое умножение в сворачивании}
    \begin{columns}[c]

        % ======== левая колонка: код ========
        \column{0.4\textwidth}
        \centering
        \usebox{\codeFactorialSimple}
        \vspace{0.4cm}

        Значение \mintinline{scheme}{1} подставлено в умножение на уровне
        \mintinline{scheme}{n = 1}: \mintinline{scheme}{(* 1 1) = 1}.
        Фрейм начинает сворачиваться.

        \vfill

        % ======= правая колонка: схема ======
        \column{0.6\textwidth}
        \centering
        \begin{tikzpicture}[
            node distance=0.32cm,
            box/.style={
                draw=blue!70!black, fill=blue!5,
                rounded corners, thick, align=center,
                minimum width=3.8cm, minimum height=0.5cm,
                font=\footnotesize
            }
        ]
            \node[box] (repl) {\alert{REPL}};
            \node[box, below=of repl] (f1-1) {\mintinline{scheme}{(f1 3)}}
            edge [<-, thick, blue!70!black] (repl);
            \node[env-node=1, right=of f1-1] (stack1) { n $\mid$ 3 }
            edge [dashed, thick, blue!70!black] (f1-1);
            \node[box, below=of f1-1] (mul-1) {\mintinline{scheme}{(* (f1 (- n 1)) n)}}
            edge [<-, thick, blue!70!black] (f1-1)
            edge [dashed, thick, blue!70!black, bend right=10] (stack1);
            \node[box, below=of mul-1] (f1-2) {\mintinline{scheme}{(f1 2)}}
            edge [<-, thick, blue!70!black] (mul-1);
            \node[env-node=1, right=of f1-2] (stack2) { n $\mid$ 2 }
            edge [dashed, thick, blue!70!black] (f1-2);
            \node[box, below=of f1-2] (mul-2) {\mintinline{scheme}{(* (f1 (- n 1)) n)}}
            edge [<-, thick, blue!70!black] (f1-2)
            edge [dashed, thick, blue!70!black, bend right=10] (stack2);

            \node[box, below=of mul-2] (f1-3) {\mintinline{scheme}{(f1 1)}}
            edge [<-, thick, blue!70!black] (mul-2);
            \node[env-node=1, right=of f1-3] (stack3) { n $\mid$ 1 }
            edge [dashed, thick, blue!70!black] (f1-3);
            \node[box, below=of f1-3] (mul-3) {\mintinline{scheme}{(* 1 1)}}
            edge [<-, thick, blue!70!black] (f1-3)
            edge [dashed, thick, blue!70!black, bend right=10] (stack3);
        \end{tikzpicture}
    \end{columns}
\end{frame}


\begin{frame}[fragile]{1.1 Обычная рекурсия: Шаг 15. Передача результата вверх}
    \begin{columns}[c]

        % ======== левая колонка: код ========
        \column{0.4\textwidth}
        \centering
        \usebox{\codeFactorialSimple}
        \vspace{0.4cm}

        Умножение завершено и возвращает 1 в вызов \mintinline{scheme}{(f1 1)}.
        Один фрейм удален.

        \vfill

        % ======= правая колонка: схема ======
        \column{0.6\textwidth}
        \centering
        \begin{tikzpicture}[
            node distance=0.32cm,
            box/.style={
                draw=blue!70!black, fill=blue!5,
                rounded corners, thick, align=center,
                minimum width=3.8cm, minimum height=0.5cm,
                font=\footnotesize
            }
        ]
            \node[box] (repl) {\alert{REPL}};
            \node[box, below=of repl] (f1-1) {\mintinline{scheme}{(f1 3)}}
            edge [<-, thick, blue!70!black] (repl);
            \node[env-node=1, right=of f1-1] (stack1) { n $\mid$ 3 }
            edge [dashed, thick, blue!70!black] (f1-1);
            \node[box, below=of f1-1] (mul-1) {\mintinline{scheme}{(* (f1 (- n 1)) n)}}
            edge [<-, thick, blue!70!black] (f1-1)
            edge [dashed, thick, blue!70!black, bend right=10] (stack1);
            \node[box, below=of mul-1] (f1-2) {\mintinline{scheme}{(f1 2)}}
            edge [<-, thick, blue!70!black] (mul-1);
            \node[env-node=1, right=of f1-2] (stack2) { n $\mid$ 2 }
            edge [dashed, thick, blue!70!black] (f1-2);
            \node[box, below=of f1-2] (mul-2) {\mintinline{scheme}{(* (f1 (- n 1)) n)}}
            edge [<-, thick, blue!70!black] (f1-2)
            edge [dashed, thick, blue!70!black, bend right=10] (stack2);

            \node[box, below=of mul-2] (f1-3) {\mintinline{scheme}{(f1 1)}}
            edge [<-, thick, blue!70!black] (mul-2);
            \node[env-node=1, right=of f1-3] (stack3) { n $\mid$ 1 }
            edge [dashed, thick, blue!70!black] (f1-3);
            \node[box, below=of f1-3] (mul-3) {\mintinline{scheme}{1}}
            edge [<-, thick, blue!70!black] (f1-3)
            edge [->, thick, blue!70!black, bend left=30] (f1-3)
            edge [dashed, thick, blue!70!black, bend right=10] (stack3);
        \end{tikzpicture}
    \end{columns}
\end{frame}


\begin{frame}[fragile]{1.1 Обычная рекурсия: Шаг 16. Завершение первого уровня}
    \begin{columns}[c]

        % ======== левая колонка: код ========
        \column{0.4\textwidth}
        \centering
        \usebox{\codeFactorialSimple}
        \vspace{0.4cm}

        Вызов \mintinline{scheme}{(f1 1)} возвращает 1.
        Результат передается на уровень выше для дальнейшего умножения.

        \vfill

        % ======= правая колонка: схема ======
        \column{0.6\textwidth}
        \centering
        \begin{tikzpicture}[
            node distance=0.34cm,
            box/.style={
                draw=blue!70!black, fill=blue!5,
                rounded corners, thick, align=center,
                minimum width=3.8cm, minimum height=0.5cm,
                font=\footnotesize
            }
        ]
            \node[box] (repl) {\alert{REPL}};
            \node[box, below=of repl] (f1-1) {\mintinline{scheme}{(f1 3)}}
            edge [<-, thick, blue!70!black] (repl);
            \node[env-node=1, right=of f1-1] (stack1) { n $\mid$ 3 }
            edge [dashed, thick, blue!70!black] (f1-1);
            \node[box, below=of f1-1] (mul-1) {\mintinline{scheme}{(* (f1 (- n 1)) n)}}
            edge [<-, thick, blue!70!black] (f1-1)
            edge [dashed, thick, blue!70!black, bend right=10] (stack1);
            \node[box, below=of mul-1] (f1-2) {\mintinline{scheme}{(f1 2)}}
            edge [<-, thick, blue!70!black] (mul-1);
            \node[env-node=1, right=of f1-2] (stack2) { n $\mid$ 2 }
            edge [dashed, thick, blue!70!black] (f1-2);
            \node[box, below=of f1-2] (mul-2) {\mintinline{scheme}{(* (f1 (- n 1)) n)}}
            edge [<-, thick, blue!70!black] (f1-2)
            edge [dashed, thick, blue!70!black, bend right=10] (stack2);
            \node[box, below=of mul-2] (f1-3) {\mintinline{scheme}{1}}
            edge [<-, thick, blue!70!black] (mul-2)
            edge [->, thick, blue!70!black, bend left=30] (mul-2);
        \end{tikzpicture}
    \end{columns}
\end{frame}


\begin{frame}[fragile]{1.1 Обычная рекурсия: Шаг 17. Умножение на втором уровне}
    \begin{columns}[c]

        % ======== левая колонка: код ========
        \column{0.4\textwidth}
        \centering
        \usebox{\codeFactorialSimple}
        \vspace{0.4cm}

        Значение 1 превращается в 2 через \mintinline{scheme}{(* 1 2)}.
        Стек продолжает сворачиваться.

        \vfill

        % ======= правая колонка: схема ======
        \column{0.6\textwidth}
        \centering
        \begin{tikzpicture}[
            node distance=0.34cm,
            box/.style={
                draw=blue!70!black, fill=blue!5,
                rounded corners, thick, align=center,
                minimum width=3.8cm, minimum height=0.5cm,
                font=\footnotesize
            }
        ]
            \node[box] (repl) {\alert{REPL}};
            \node[box, below=of repl] (f1-1) {\mintinline{scheme}{(f1 3)}}
            edge [<-, thick, blue!70!black] (repl);
            \node[env-node=1, right=of f1-1] (stack1) { n $\mid$ 3 }
            edge [dashed, thick, blue!70!black] (f1-1);
            \node[box, below=of f1-1] (mul-1) {\mintinline{scheme}{(* (f1 (- n 1)) n)}}
            edge [<-, thick, blue!70!black] (f1-1)
            edge [dashed, thick, blue!70!black, bend right=10] (stack1);
            \node[box, below=of mul-1] (f1-2) {\mintinline{scheme}{(f1 2)}}
            edge [<-, thick, blue!70!black] (mul-1);
            \node[env-node=1, right=of f1-2] (stack2) { n $\mid$ 2 }
            edge [dashed, thick, blue!70!black] (f1-2);
            \node[box, below=of f1-2] (mul-2) {\mintinline{scheme}{(* 1 2)}}
            edge [<-, thick, blue!70!black] (f1-2)
            edge [dashed, thick, blue!70!black, bend right=10] (stack2);
        \end{tikzpicture}

    \end{columns}
\end{frame}


\begin{frame}[fragile]{1.1 Обычная рекурсия: Шаг 18. Завершение второго уровня}
    \begin{columns}[c]

        % ======== левая колонка: код ========
        \column{0.4\textwidth}
        \centering
        \usebox{\codeFactorialSimple}
        \vspace{0.4cm}

        Результат умножения передаётся в вызов \mintinline{scheme}{(f1 2)}.

        \vfill

        % ======= правая колонка: схема ======
        \column{0.6\textwidth}
        \centering
        \begin{tikzpicture}[
            node distance=0.36cm,
            box/.style={
                draw=blue!70!black, fill=blue!5,
                rounded corners, thick, align=center,
                minimum width=3.8cm, minimum height=0.5cm,
                font=\footnotesize
            }
        ]
            \node[box] (repl) {\alert{REPL}};
            \node[box, below=of repl] (f1-1) {\mintinline{scheme}{(f1 3)}}
            edge [<-, thick, blue!70!black] (repl);
            \node[env-node=1, right=of f1-1] (stack1) { n $\mid$ 3 }
            edge [dashed, thick, blue!70!black] (f1-1);
            \node[box, below=of f1-1] (mul-1) {\mintinline{scheme}{(* (f1 (- n 1)) n)}}
            edge [<-, thick, blue!70!black] (f1-1)
            edge [dashed, thick, blue!70!black, bend right=10] (stack1);
            \node[box, below=of mul-1] (f1-2) {\mintinline{scheme}{(f1 2)}}
            edge [<-, thick, blue!70!black] (mul-1);
            \node[env-node=1, right=of f1-2] (stack2) { n $\mid$ 2 }
            edge [dashed, thick, blue!70!black] (f1-2);
            \node[box, below=of f1-2] (mul-2) {\mintinline{scheme}{2}}
            edge [<-, thick, blue!70!black] (f1-2)
            edge [->, thick, blue!70!black, bend left=30] (f1-2)
            edge [dashed, thick, blue!70!black, bend right=10] (stack2);
        \end{tikzpicture}

    \end{columns}
\end{frame}


\begin{frame}[fragile]{1.1 Обычная рекурсия: Шаг 19. Передача результата вверх}
    \begin{columns}[c]

        % ======== левая колонка: код ========
        \column{0.4\textwidth}
        \centering
        \usebox{\codeFactorialSimple}
        \vspace{0.4cm}

        Значение \mintinline{scheme}{2} передаётся в умножение на уровне

        \vfill

        % ======= правая колонка: схема ======
        \column{0.6\textwidth}
        \centering
        \begin{tikzpicture}[
            node distance=0.36cm,
            box/.style={
                draw=blue!70!black, fill=blue!5,
                rounded corners, thick, align=center,
                minimum width=3.8cm, minimum height=0.5cm,
                font=\footnotesize
            }
        ]
            \node[box] (repl) {\alert{REPL}};
            \node[box, below=of repl] (f1-1) {\mintinline{scheme}{(f1 3)}}
            edge [<-, thick, blue!70!black] (repl);
            \node[env-node=1, right=of f1-1] (stack1) { n $\mid$ 3 }
            edge [dashed, thick, blue!70!black] (f1-1);
            \node[box, below=of f1-1] (mul-1) {\mintinline{scheme}{(* (f1 (- n 1)) n)}}
            edge [<-, thick, blue!70!black] (f1-1)
            edge [dashed, thick, blue!70!black, bend right=10] (stack1);
            \node[box, below=of mul-1] (f1-2) {\mintinline{scheme}{2}}
            edge [<-, thick, blue!70!black] (mul-1)
            edge [->, thick, blue!70!black, bend left=30] (mul-1);
        \end{tikzpicture}

    \end{columns}
\end{frame}

\begin{frame}[fragile]{1.1 Обычная рекурсия: Шаг 20. Умножение на 4 уровне}
    \begin{columns}[c]

        % ======== левая колонка: код ========
        \column{0.4\textwidth}
        \centering
        \usebox{\codeFactorialSimple}
        \vspace{0.4cm}

        Значение \mintinline{scheme}{2} передаётся в \mintinline{scheme}{(* 2 3)}.

        \vfill

        % ======= правая колонка: схема ======
        \column{0.6\textwidth}
        \centering
        \begin{tikzpicture}[
            node distance=0.38cm,
            box/.style={
                draw=blue!70!black, fill=blue!5,
                rounded corners, thick, align=center,
                minimum width=3.8cm, minimum height=0.5cm,
                font=\footnotesize
            }
        ]
            \node[box] (repl) {\alert{REPL}};
            \node[box, below=of repl] (f1-1) {\mintinline{scheme}{(f1 3)}}
            edge [<-, thick, blue!70!black] (repl);
            \node[env-node=1, right=of f1-1] (stack1) { n $\mid$ 3 }
            edge [dashed, thick, blue!70!black] (f1-1);
            \node[box, below=of f1-1] (mul-1) {\mintinline{scheme}{(* 2 3)}}
            edge [<-, thick, blue!70!black] (f1-1)
            edge [dashed, thick, blue!70!black, bend right=10] (stack1);
        \end{tikzpicture}

    \end{columns}
\end{frame}


\begin{frame}[fragile]{1.1 Обычная рекурсия: Шаг 21. Передача результата вверх}
    \begin{columns}[c]

        % ======== левая колонка: код ========
        \column{0.4\textwidth}
        \centering
        \usebox{\codeFactorialSimple}
        \vspace{0.4cm}

        Умножение \mintinline{scheme}{(* 2 3)}
        вернуло \mintinline{scheme}{6}.
        Это и есть окончательное значение всего выражения
        \mintinline{scheme}{(f1 3)}.

        \vfill

        % ======= правая колонка: схема ======
        \column{0.6\textwidth}
        \centering
        \begin{tikzpicture}[
            node distance=0.38cm,
            box/.style={
                draw=blue!70!black, fill=blue!5,
                rounded corners, thick, align=center,
                minimum width=3.8cm, minimum height=0.5cm,
                font=\footnotesize
            }
        ]
            \node[box] (repl) {\alert{REPL}};
            \node[box, below=of repl] (f1-1) {\mintinline{scheme}{(f1 3)}}
            edge [<-, thick, blue!70!black] (repl);
            \node[env-node=1, right=of f1-1] (stack1) { n $\mid$ 3 }
            edge [dashed, thick, blue!70!black] (f1-1);
            \node[box, below=of f1-1] (mul-1) {\mintinline{scheme}{6}}
            edge [<-, thick, blue!70!black] (f1-1)
            edge [->, thick, blue!70!black, bend left=30] (f1-1)
            edge [dashed, thick, blue!70!black, bend right=10] (stack1);
        \end{tikzpicture}

    \end{columns}
\end{frame}


\begin{frame}[fragile]{1.1 Обычная рекурсия: Шаг 22. Завершение рекурсии}
    \begin{columns}[c]

        % ======== левая колонка: код ========
        \column{0.4\textwidth}
        \centering
        \usebox{\codeFactorialSimple}
        \vspace{0.4cm}

        Значение \mintinline{scheme}{6} возвращено в \alert{REPL}.

        \vfill

        % ======= правая колонка: схема ======
        \column{0.6\textwidth}
        \centering
        \begin{tikzpicture}[
            node distance=0.38cm,
            box/.style={
                draw=blue!70!black, fill=blue!5,
                rounded corners, thick, align=center,
                minimum width=3.8cm, minimum height=0.5cm,
                font=\footnotesize
            }
        ]
            \node[box] (repl) {\alert{REPL}};
            \node[box, below=of repl] (f1-1) {\mintinline{scheme}{6}}
            edge [<-, thick, blue!70!black] (repl);
        \end{tikzpicture}

    \end{columns}
\end{frame}
