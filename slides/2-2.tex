\begin{frame}{2.2 Замыкание для хранения состояния}
    \centering

    Разберём по шагам, как обычная лямбда-функция «захватывает»
    переменные из места своего рождения, сохраняет их в куче и
    продолжает к ним обращаться даже после завершения всех вызовов

    \usebox{\codeClosureCounter}
\end{frame}


\begin{frame}[fragile]{2.2 Замыкание лямбды: Шаг 1. Определение \alert{counter}}
    \begin{columns}[c]

        % ======== левая колонка: код ========
        \column{0.4\textwidth}
        \centering
        \usebox{\codeClosureCounter}
        \vspace{0.4cm}

        В \alert{REPL} вводится определение переменной \mintinline{scheme}{counter}.
        Процедура создаётся с помощью немедленного вызова
        \mintinline{scheme}{(lambda (n) ...)} с начальным значением \mintinline{scheme}{n = 0}.

        \vfill

        % ======= правая колонка: схема ======
        \column{0.6\textwidth}
        \centering
        \begin{tikzpicture}[
            node distance=0.5cm,
            box/.style={
                draw=blue!70!black, fill=blue!5,
                rounded corners, thick, align=center,
                minimum width=2cm, minimum height=0.9cm,
                font=\footnotesize
            }
        ]
            \node[box] (repl) {\alert{REPL}};
            \node[box, above=of repl, align=left] (f1-1) {
                \mintinline{scheme}{(define counter } \\
                \phantom{\ \ (}\mintinline{scheme}{((lambda (n) } \\
                \phantom{\ \ \ \ (}\mintinline{scheme}{(lambda () } \\
                \phantom{\ \ \ \ \ \ (}\phantom{(}\mintinline{scheme}{(set! n (+ n 1)) } \\
                \phantom{\ \ \ \ \ \ (}\phantom{(}\mintinline{scheme}{n } \\
                \phantom{\ \ \ \ (}\mintinline{scheme}{0)) }
            }
                edge [<-, thick, blue!70!black] (repl);
        \end{tikzpicture}
    \end{columns}
\end{frame}


\begin{frame}[fragile]{2.2 Замыкание лямбды: Шаг 2. Завершение внешней лямбды}
    \begin{columns}[c]

        % ======== левая колонка: код ========
        \column{0.4\textwidth}
        \centering
        \usebox{\codeClosureCounter}
        \vspace{0.4cm}

        Вычисление \mintinline{scheme}{(define counter ...)} завершено.

        \vfill

        % ======= правая колонка: схема ======
        \column{0.6\textwidth}
        \centering
        \begin{tikzpicture}[
            node distance=0.5cm,
            box/.style={
                draw=blue!70!black, fill=blue!5,
                rounded corners, thick, align=center,
                minimum width=2cm, minimum height=0.9cm,
                font=\footnotesize
            }
        ]
            \node[box] (repl) {\alert{REPL}};
            \node[box, above=of repl, align=left] (f1-1) {
                \mintinline{scheme}{(define counter } \\
                \phantom{\ \ (}\mintinline{scheme}{((lambda (n) } \\
                \phantom{\ \ \ \ (}\mintinline{scheme}{(lambda () } \\
                \phantom{\ \ \ \ \ \ (}\phantom{(}\mintinline{scheme}{(set! n (+ n 1)) } \\
                \phantom{\ \ \ \ \ \ (}\phantom{(}\mintinline{scheme}{n } \\
                \phantom{\ \ \ \ (}\mintinline{scheme}{0)) }
            }
                edge [<-, thick, blue!70!black] (repl);
            \node[env-node=2, rectangle split, right=of repl] (stack1) { \mintinline{scheme}{counter} \nodepart{two} \mintinline{scheme}{((lambda (n) ...) 0)} }
                edge [dashed, thick, blue!70!black] (repl);
        \end{tikzpicture}
    \end{columns}
\end{frame}


\begin{frame}[fragile]{2.2 Замыкание лямбды: Шаг 3. Глобальная среда после определения}
    \begin{columns}[c]

        % ======== левая колонка: код ========
        \column{0.4\textwidth}
        \centering
        \usebox{\codeClosureCounter}
        \vspace{0.4cm}

        Определение \mintinline{scheme}{counter} полностью вычислено и
        записано в глобальную среду.

        \vfill

        % ======= правая колонка: схема ======
        \column{0.6\textwidth}
        \centering
        \begin{tikzpicture}[
            node distance=0.5cm,
            box/.style={
                draw=blue!70!black, fill=blue!5,
                rounded corners, thick, align=center,
                minimum width=2cm, minimum height=0.9cm,
                font=\footnotesize
            }
        ]
            \node[box] (repl) {\alert{REPL}};
            \node[box, above=of repl, align=left] (f1-1) {
                \mintinline{scheme}{(define counter } \\
                \phantom{\ \ (}\mintinline{scheme}{((lambda (n) } \\
                \phantom{\ \ \ \ (}\mintinline{scheme}{(lambda () } \\
                \phantom{\ \ \ \ \ \ (}\phantom{(}\mintinline{scheme}{(set! n (+ n 1)) } \\
                \phantom{\ \ \ \ \ \ (}\phantom{(}\mintinline{scheme}{n } \\
                \phantom{\ \ \ \ (}\mintinline{scheme}{0)) }
            }
                edge [<-, thick, blue!70!black] (repl);
            \node[env-node=2, rectangle split, right=of repl] (stack1) { \mintinline{scheme}{counter} \nodepart{two} \mintinline{scheme}{((lambda (n) ...) 0)} }
                edge [dashed, thick, blue!70!black] (repl);
            \node[env-node=1, right=of f1-1] (stack2) {n $\mid$ 0}
                edge [dashed, thick, blue!70!black] (f1-1);
        \end{tikzpicture}
    \end{columns}
\end{frame}


\begin{frame}[fragile]{2.2 Замыкание лямбды: Шаг 4. Замыкание полностью сформировано и «привязано» к своей среде}
    \begin{columns}[c]

        % ======== левая колонка: код ========
        \column{0.4\textwidth}
        \centering
        \usebox{\codeClosureCounter}
        \vspace{0.2cm}

        Замыкание полностью сформировано: \mintinline{scheme}{counter} ссылается на
        внутреннюю лямбду, которая «помнит» \mintinline{scheme}{n=0} — глобальная
        среда обновлена без изменений.

        \vfill

        % ======= правая колонка: схема ======
        \column{0.6\textwidth}
        \centering
        \begin{tikzpicture}[
            node distance=0.5cm,
            box/.style={
                draw=blue!70!black, fill=blue!5,
                rounded corners, thick, align=center,
                minimum width=2cm, minimum height=0.9cm,
                font=\footnotesize
            }
        ]
            \node[box] (repl) {\alert{REPL}};
            \node[box, above=of repl, align=left] (f1-1) {
                \mintinline{scheme}{(define counter } \\
                \phantom{\ \ (}\mintinline{scheme}{((lambda (n) } \\
                \phantom{\ \ \ \ (}\mintinline{scheme}{(lambda () } \\
                \phantom{\ \ \ \ \ \ (}\phantom{(}\mintinline{scheme}{(set! n (+ n 1)) } \\
                \phantom{\ \ \ \ \ \ (}\phantom{(}\mintinline{scheme}{n } \\
                \phantom{\ \ \ \ (}\mintinline{scheme}{0)) }
            }
                edge [<-, thick, blue!70!black] (repl);
            \node[env-node=2, rectangle split, right=of repl] (stack1) { \mintinline{scheme}{counter} \nodepart{two} \mintinline{scheme}{((lambda (n) ...) 0)} }
                edge [dashed, thick, blue!70!black] (repl);
            \node[env-node=1, right=of f1-1] (stack2) {n $\mid$ 0}
                edge [dashed, thick, blue!70!black] (f1-1)
                edge [dashed, thick, red!70!black] (stack1);
        \end{tikzpicture}
    \end{columns}
\end{frame}


\begin{frame}[fragile]{2.2 Замыкание лямбды: Шаг 5. Готовность к первому вызову}
    \begin{columns}[c]

        % ======== левая колонка: код ========
        \column{0.4\textwidth}
        \centering
        \usebox{\codeClosureCounter}
        \vspace{0.4cm}

        Выражение определения \mintinline{scheme}{(define counter ...)}
        полностью отработало и исчезло из вида.
        В стеке вызовов ничего нет — мы снова на верхнем уровне REPL.

        \vfill

        % ======= правая колонка: схема ======
        \column{0.6\textwidth}
        \centering
        \begin{tikzpicture}[
            node distance=0.5cm,
            box/.style={
                draw=blue!70!black, fill=blue!5,
                rounded corners, thick, align=center,
                minimum width=2cm, minimum height=0.9cm,
                font=\footnotesize
            }
        ]
            \node[box] (repl) {\alert{REPL}};
            \node[env-node=2, rectangle split, right=of repl] (stack1) { \mintinline{scheme}{counter} \nodepart{two} \mintinline{scheme}{((lambda (n) ...) 0)} }
                edge [dashed, thick, blue!70!black] (repl);
            \node[env-node=1, above=of stack1] (stack2) {n $\mid$ 0}
                edge [dashed, thick, red!70!black] (stack1);
        \end{tikzpicture}
    \end{columns}
\end{frame}


\begin{frame}[fragile]{2.2 Замыкание лямбды: Шаг 6. Вызов счётчика в первый раз}
    \begin{columns}[c]

        % ======== левая колонка: код ========
        \column{0.4\textwidth}
        \centering
        \usebox{\codeClosureCounter}
        \vspace{0.4cm}

        Среда \alert{REPL} выполняет выражение \mintinline{scheme}{(counter)}.

        \vfill

        % ======= правая колонка: схема ======
        \column{0.6\textwidth}
        \centering
        \begin{tikzpicture}[
            node distance=0.5cm,
            box/.style={
                draw=blue!70!black, fill=blue!5,
                rounded corners, thick, align=center,
                minimum width=2cm, minimum height=0.9cm,
                font=\footnotesize
            }
        ]
            \node[box] (repl) {\alert{REPL}};
            \node[env-node=2, rectangle split, right=of repl] (stack1) { \mintinline{scheme}{counter} \nodepart{two} \mintinline{scheme}{((lambda (n) ...) 0)} }
                edge [dashed, thick, blue!70!black] (repl);
            \node[env-node=1, above=of stack1] (stack2) {n $\mid$ 0}
                edge [dashed, thick, red!70!black] (stack1);
            \node[box, above=of repl] (c1) { \mintinline{scheme}{ (counter) } }
                edge [dashed, thick, blue!70!black] (stack1)
                edge [<-, thick, blue!70!black] (repl);
        \end{tikzpicture}
    \end{columns}
\end{frame}


\begin{frame}[fragile]{2.2 Замыкание лямбды: Шаг 7. Вход в тело замыкания}
    \begin{columns}[c]

        % ======== левая колонка: код ========
        \column{0.4\textwidth}
        \centering
        \usebox{\codeClosureCounter}
        \vspace{0.4cm}

        Вызов перенаправлен во внешнюю лямбду с \mintinline{scheme}{n=0}; тело
        начинает вычисляться — добавлен фрейм, ссылающийся на захваченную \mintinline{scheme}{n}.

        \vfill

        % ======= правая колонка: схема ======
        \column{0.6\textwidth}
        \centering
        \begin{tikzpicture}[
            node distance=0.5cm,
            box/.style={
                draw=blue!70!black, fill=blue!5,
                rounded corners, thick, align=center,
                minimum width=2cm, minimum height=0.9cm,
                font=\footnotesize
            }
        ]
            \node[box] (repl) {\alert{REPL}};
            \node[env-node=2, rectangle split, right=of repl] (stack1) { \mintinline{scheme}{counter} \nodepart{two} \mintinline{scheme}{((lambda (n) ...) 0)} }
                edge [dashed, thick, blue!70!black] (repl);
            \node[env-node=1, above=of stack1] (stack2) {n $\mid$ 0}
                edge [dashed, thick, red!70!black] (stack1);
            \node[box, above=of repl, align=left] (c1) {
                \mintinline{scheme}{ ((lambda (n) } \\
                \phantom{\ \ (}\mintinline{scheme}{(lambda () } \\
                \phantom{\ \ \ \ (}\phantom{(}\mintinline{scheme}{(set! n (+ n 1)) } \\
                \phantom{\ \ \ \ (}\phantom{(}\mintinline{scheme}{n)) } \\
                \phantom{\ \ (}\mintinline{scheme}{0) }
            }
                edge [dashed, thick, blue!70!black] (stack1)
                edge [dashed, thick, blue!70!black] (stack2)
                edge [<-, thick, blue!70!black] (repl);
        \end{tikzpicture}
    \end{columns}
\end{frame}


\begin{frame}[fragile]{2.2 Замыкание лямбды: Шаг 8. Создание внутренней лямбды}
    \begin{columns}[c]

        % ======== левая колонка: код ========
        \column{0.4\textwidth}
        \centering
        \usebox{\codeClosureCounter}
        \vspace{0.4cm}

        Интерпретатор дошёл до тела внешней лямбды и теперь вычисляет её
        единственное выражение — внутреннюю лямбду

        \vfill

        % ======= правая колонка: схема ======
        \column{0.6\textwidth}
        \centering
        \begin{tikzpicture}[
            node distance=0.6cm,
            box/.style={
                draw=blue!70!black, fill=blue!5,
                rounded corners, thick, align=center,
                minimum width=2cm, minimum height=0.9cm,
                font=\footnotesize
            }
        ]
            \node[box] (repl) {\alert{REPL}};
            \node[env-node=2, rectangle split, right=of repl] (stack1) { \mintinline{scheme}{counter} \nodepart{two} \mintinline{scheme}{((lambda (n) ...) 0)} }
                edge [dashed, thick, blue!70!black] (repl);
            \node[env-node=1, above=1cm of stack1] (stack2) {n $\mid$ 0}
                edge [dashed, thick, red!70!black] (stack1);
            \node[box, above=of repl, align=left] (c1) {
                \mintinline{scheme}{ ((lambda (n) } \\
                \phantom{\ \ (}\mintinline{scheme}{(lambda () } \\
                \phantom{\ \ \ \ (}\phantom{(}\mintinline{scheme}{(set! n (+ n 1)) } \\
                \phantom{\ \ \ \ (}\phantom{(}\mintinline{scheme}{n)) } \\
                \phantom{\ \ (}\mintinline{scheme}{0) }
            }
                edge [dashed, thick, blue!70!black] (stack1)
                edge [dashed, thick, blue!70!black] (stack2)
                edge [<-, thick, blue!70!black] (repl);
            \node[box, above=of c1, align=left] (c2) {
                \mintinline{scheme}{ (lambda () } \\
                \phantom{\ \ (}\phantom{(}\mintinline{scheme}{(set! n (+ n 1)) } \\
                \phantom{\ \ (}\phantom{(}\mintinline{scheme}{n)) } \\
            }
                edge [dashed, thick, blue!70!black, bend left=40] (stack2)
                edge [<-, thick, blue!70!black] (c1);
        \end{tikzpicture}
    \end{columns}
\end{frame}


\begin{frame}[fragile]{2.2 Замыкание лямбды: Шаг 9. Вызов внутренней лямбды}
    \begin{columns}[c]

        % ======== левая колонка: код ========
        \column{0.4\textwidth}
        \centering
        \usebox{\codeClosureCounter}
        \vspace{0.4cm}

        Внутренняя лямбда вызвана: интерпретатор входит в её тело и начинает
        \mintinline{scheme}{set!} — фрейм ссылается на
        \mintinline{scheme}{n=0}, подготавливая инкремент

        \vfill

        % ======= правая колонка: схема ======
        \column{0.6\textwidth}
        \centering
        \begin{tikzpicture}[
            node distance=0.6cm,
            box/.style={
                draw=blue!70!black, fill=blue!5,
                rounded corners, thick, align=center,
                minimum width=2cm, minimum height=0.9cm,
                font=\footnotesize
            }
        ]
            \node[box] (repl) {\alert{REPL}};
            \node[env-node=2, rectangle split, right=of repl] (stack1) { \mintinline{scheme}{counter} \nodepart{two} \mintinline{scheme}{((lambda (n) ...) 0)} }
                edge [dashed, thick, blue!70!black] (repl);
            \node[env-node=1, above=1cm of stack1] (stack2) {n $\mid$ 0}
                edge [dashed, thick, red!70!black] (stack1);
            \node[box, above=of repl, align=left] (c1) {
                \mintinline{scheme}{ ((lambda (n) } \\
                \phantom{\ \ (}\mintinline{scheme}{(lambda () } \\
                \phantom{\ \ \ \ (}\phantom{(}\mintinline{scheme}{(set! n (+ n 1)) } \\
                \phantom{\ \ \ \ (}\phantom{(}\mintinline{scheme}{n)) } \\
                \phantom{\ \ (}\mintinline{scheme}{0) }
            }
                edge [dashed, thick, blue!70!black] (stack1)
                edge [dashed, thick, blue!70!black] (stack2)
                edge [<-, thick, blue!70!black] (repl);
            \node[box, above=of c1, align=left] (c2) {
                \mintinline{scheme}{ (lambda () } \\
                \phantom{\ \ (}\phantom{(}\mintinline{scheme}{(set! n (+ n 1)) } \\
                \phantom{\ \ (}\phantom{(}\mintinline{scheme}{n)) } \\
            }
                edge [dashed, thick, blue!70!black, bend left=40] (stack2)
                edge [<-, thick, blue!70!black] (c1);
            \node[box, right=of c2, yshift=0.5cm] (c3) {
                \mintinline{scheme}{ (set! n (+ n 1)) } }
                edge [dashed, thick, blue!70!black] (stack2)
                edge [<-, thick, blue!70!black] (c2);
        \end{tikzpicture}
    \end{columns}
\end{frame}


\begin{frame}[fragile]{2.2 Замыкание лямбды: Шаг 10. Изменение захваченной переменной}
    \begin{columns}[c]

        % ======== левая колонка: код ========
        \column{0.4\textwidth}
        \centering
        \usebox{\codeClosureCounter}
        \vspace{0.4cm}

        Выражение \mintinline{scheme}{(set! n (+ n 1))} уже вычислено и
        превратилось в \mintinline{scheme}{(set! n 1)}.

        \vfill

        % ======= правая колонка: схема ======
        \column{0.6\textwidth}
        \centering
        \begin{tikzpicture}[
            node distance=0.6cm,
            box/.style={
                draw=blue!70!black, fill=blue!5,
                rounded corners, thick, align=center,
                minimum width=2cm, minimum height=0.9cm,
                font=\footnotesize
            }
        ]
            \node[box] (repl) {\alert{REPL}};
            \node[env-node=2, rectangle split, right=of repl] (stack1) { \mintinline{scheme}{counter} \nodepart{two} \mintinline{scheme}{((lambda (n) ...) 0)} }
                edge [dashed, thick, blue!70!black] (repl);
            \node[env-node=1, above=1cm of stack1] (stack2) {n $\mid$ 0}
                edge [dashed, thick, red!70!black] (stack1);
            \node[box, above=of repl, align=left] (c1) {
                \mintinline{scheme}{ ((lambda (n) } \\
                \phantom{\ \ (}\mintinline{scheme}{(lambda () } \\
                \phantom{\ \ \ \ (}\phantom{(}\mintinline{scheme}{(set! n (+ n 1)) } \\
                \phantom{\ \ \ \ (}\phantom{(}\mintinline{scheme}{n)) } \\
                \phantom{\ \ (}\mintinline{scheme}{0) }
            }
                edge [dashed, thick, blue!70!black] (stack1)
                edge [dashed, thick, blue!70!black] (stack2)
                edge [<-, thick, blue!70!black] (repl);
            \node[box, above=of c1, align=left] (c2) {
                \mintinline{scheme}{ (lambda () } \\
                \phantom{\ \ (}\phantom{(}\mintinline{scheme}{(set! n (+ n 1)) } \\
                \phantom{\ \ (}\phantom{(}\mintinline{scheme}{n)) } \\
            }
                edge [dashed, thick, blue!70!black, bend left=40] (stack2)
                edge [<-, thick, blue!70!black] (c1);
            \node[box, right=of c2, yshift=0.5cm] (c3) {
                \mintinline{scheme}{ (set! n 1) } }
                edge [dashed, thick, blue!70!black] (stack2)
                edge [<-, thick, blue!70!black] (c2);
        \end{tikzpicture}
    \end{columns}
\end{frame}


\begin{frame}[fragile]{2.2 Замыкание лямбды: Шаг 11. Возврат значения из замыкания}
    \begin{columns}[c]

        % ======== левая колонка: код ========
        \column{0.4\textwidth}
        \centering
        \usebox{\codeClosureCounter}
        \vspace{0.4cm}

        Присваивание \mintinline{scheme}{(set! n (+ n 1))} завершено — в
        захваченной ячейке теперь лежит \mintinline{scheme}{n=1}.

        \vfill

        % ======= правая колонка: схема ======
        \column{0.6\textwidth}
        \centering
        \begin{tikzpicture}[
            node distance=0.6cm,
            box/.style={
                draw=blue!70!black, fill=blue!5,
                rounded corners, thick, align=center,
                minimum width=2cm, minimum height=0.9cm,
                font=\footnotesize
            }
        ]
            \node[box] (repl) {\alert{REPL}};
            \node[env-node=2, rectangle split, right=of repl] (stack1) { \mintinline{scheme}{counter} \nodepart{two} \mintinline{scheme}{((lambda (n) ...) 0)} }
                edge [dashed, thick, blue!70!black] (repl);
            \node[env-node=1, above=1cm of stack1] (stack2) {n $\mid$ 1}
                edge [dashed, thick, red!70!black] (stack1);
            \node[box, above=of repl, align=left] (c1) {
                \mintinline{scheme}{ ((lambda (n) } \\
                \phantom{\ \ (}\mintinline{scheme}{(lambda () } \\
                \phantom{\ \ \ \ (}\phantom{(}\mintinline{scheme}{(set! n (+ n 1)) } \\
                \phantom{\ \ \ \ (}\phantom{(}\mintinline{scheme}{n)) } \\
                \phantom{\ \ (}\mintinline{scheme}{0) }
            }
                edge [dashed, thick, blue!70!black] (stack1)
                edge [dashed, thick, blue!70!black] (stack2)
                edge [<-, thick, blue!70!black] (repl);
            \node[box, above=of c1] (c2) {
                \mintinline{scheme}{ n }
            }
                edge [dashed, thick, blue!70!black, bend left=40] (stack2)
                edge [<-, thick, blue!70!black] (c1);
        \end{tikzpicture}
    \end{columns}
\end{frame}


\begin{frame}[fragile]{2.2 Замыкание лямбды: Шаг 12. Завершение первого вызова}
    \begin{columns}[c]

        % ======== левая колонка: код ========
        \column{0.4\textwidth}
        \centering
        \usebox{\codeClosureCounter}
        \vspace{0.4cm}

        Вызов \mintinline{scheme}{(counter)} полностью
        завершён и вернул значение \mintinline{scheme}{1}.

        \vfill

        % ======= правая колонка: схема ======
        \column{0.6\textwidth}
        \centering
        \begin{tikzpicture}[
            node distance=0.6cm,
            box/.style={
                draw=blue!70!black, fill=blue!5,
                rounded corners, thick, align=center,
                minimum width=2cm, minimum height=0.9cm,
                font=\footnotesize
            }
        ]
            \node[box] (repl) {\alert{REPL}};
            \node[env-node=2, rectangle split, right=of repl] (stack1) { \mintinline{scheme}{counter} \nodepart{two} \mintinline{scheme}{((lambda (n) ...) 0)} }
                edge [dashed, thick, blue!70!black] (repl);
            \node[env-node=1, above=1cm of stack1] (stack2) {n $\mid$ 1}
                edge [dashed, thick, red!70!black] (stack1);
            \node[box, above=of repl] (c1) { 1 }
                edge [<-, thick, blue!70!black] (repl);
        \end{tikzpicture}
    \end{columns}
\end{frame}
